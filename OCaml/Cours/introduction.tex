%%-*-latex-*-

%--------------------------------------------------------------------
%
\begin{frame}
\frametitle{Objectifs didactiques}

Découverte d'un nouveau paradigme de programmation: après les langages
de script, la programmation structurée et à objets, présenter la
programmation fonctionnelle illustrée par le langage OCaml.

\bigskip

Le temps imparti étant très court, on n'abordera que les traits
fonctionnels et distinctifs d'OCaml.

\bigskip

L'approche sera très différente, voire déroutante, par rapport à
celles que vous avez déjà pratiquées.

\bigskip

À cause de la nouveauté et du peu de temps, il sera donc essentiel de
travailler en dehors des heures de classe.

\end{frame}

% ------------------------------------------------------------------------
%
\begin{frame}
\frametitle{Domaines d'usage de OCaml}

\textcolor{blue}{Usage général}

\bigskip

\textcolor{blue}{Domaines de prédilection}

\begin{itemize}

  \item calcul symbolique: preuves mathématiques, compilation,
        interprétation, analyses de programmes;

  \item prototypage rapide et langages dédiés (\emph{Domain Specific
        Languages}).

\end{itemize}

\textcolor{blue}{Enseignement et recherche} 

Classes préparatoires, grandes universités (Europe, USA, Japon).

\bigskip

\textcolor{blue}{Industrie} CEA, EDF, France Télécom, Simulog.

\bigskip

\textcolor{blue}{Gros logiciels} Coq, Ensemble, Unison.

\end{frame}

% ------------------------------------------------------------------------
%
\begin{frame}
\frametitle{Quelques mots-clés}

Le langage OCaml est dit:
\begin{itemize}

  \item \emph{fonctionnel}: les fonctions peuvent être directement
  passées à d'autre fonctions et être retournées;

  \item \emph{à gestion de mémoire automatique} (comme \textsf{Java});

  \item \emph{fortement typé}: le système de types garanti l'absence
  d'erreurs à l'exécution dues à des incohérences sur les données;

  \item \emph{avec inférence de types statique}: les annotations de
  types sont facultatives car inférées par le compilateur;

  \item \emph{compilé} ou \emph{interactif} (interactif comme une
  calculette);

  \item \emph{à objets} et \emph{à modules}.

\end{itemize}

\end{frame}

% ------------------------------------------------------------------------
%
\begin{frame}
\frametitle{Mini-ML}

\noindent Commençons pas présenter un sous-ensemble du noyau
fonctionnel de OCaml: mini-ML.

\bigskip

Une \emph{phrase} est définie par les cas suivants, où $e$ dénote une
expression, $x$ et $f$ sont des variables et \Xlet et \Xrec sont des
mots-clés:

\medskip

\begin{tabular}{rll}
    $\bullet$ 
  & \textcolor{blue}{définition globale}
  & \phrase{$e$}\\
  &
  & \phrase{$\Xlet \,\, x = e$}\\
    $\bullet$
  & \textcolor{blue}{définition globale récursive}
  & \phrase{$\Xlet \,\, \Xrec \,\, f = e$}
\end{tabular}

Les phrases sont terminées par «~\textsf{;;}~» et un \emph{programme}
est une suite de phrases.

\end{frame}

% ------------------------------------------------------------------------
%
\begin{frame}
\frametitle{Méta-variables et variables}

Lorsque l'on écrit que $e$ est une expression, on veut dire que $e$
désigne une portion de phrase OCaml qu'on classe syntaxiquement
(c.-à-d. d'après leur forme) dans les expressions.

\bigskip

On dit que $e$ est une \emph{méta-variable} car, étant un nom c'est
une variable, mais cette variable n'existe pas dans le langage décrit
(OCaml): elle existe dans le langage de \emph{description}. Autrement
dit, ce n'est pas du OCaml mais une notation pour décrire des
fragments de OCaml (éventuellement une infinité).

\bigskip

Ainsi la méta-variable $x$ désigne un ensemble (peut-être infini) de
variables OCaml, et il ne faut pas la confondre avec la variable
OCaml~\ident{x}. De même, la méta-variable $e_1$ désigne une infinité
d'expressions.

\end{frame}

% ------------------------------------------------------------------------
%
\begin{frame}[containsverbatim]
\frametitle{Les expressions}

Une expression $e$ est définie \emph{récursivement} par les cas
suivants:

\medskip

\begin{tabular}{rll}
    $\bullet$
  & \textcolor{blue}{variable}
  & $f, g, h$ (fonctions); $x, y, z$ (autres). \\
    $\bullet$
  & \textcolor{blue}{fonction} (ou \textcolor{blue}{abstraction})
  & $\Xfun \,\, x \rightarrow e$\\
    $\bullet$ 
  & \textcolor{blue}{appel} (ou \textcolor{blue}{application})
  & $e_1 \,\, e_2$ \\ 
    $\bullet$
  & \textcolor{blue}{opérateur arithmétique}
  & \lpar\texttt{+}\rpar \ \lpar\texttt{-}\rpar \ \lpar\texttt{/}\rpar
    \ \lpar\texttt{*}\rpar\\
    $\bullet$
  & \textcolor{blue}{opération arithmétique}
  & $e_1$ \texttt{+} $e_2$ ou $e_1$ \texttt{-} $e_2$\\
 && ou $e_1$ \texttt{/} $e_2$ ou $e_1$ \texttt{*} $e_2$\\
    $\bullet$
  & \textcolor{blue}{unité ou constante entière}
  & \unit \ ou \textsf{0} ou \textsf{1} ou \textsf{2} etc.\\
    $\bullet$
  & \textcolor{blue}{parenthèse}
  & $\lpar{e}\rpar$\\
    $\bullet$
  & \textcolor{blue}{définition locale}
  & $\Xlet \,\, x = e_1 \,\, \Xin \,\, e_2$
\end{tabular}

\bigskip

\remarques \ Ce que nous écrivons joliment $\rightarrow$ s'écrit
\verb+->+ en \textsc{ascii}.

%% Le symbole = a deux sens: définition et comparaison (égalité
%% structurelle).

\end{frame}

% ------------------------------------------------------------------------
%
\begin{frame}
\frametitle{Un programme correct}

{\small
\phrase{let x = 0}

\phrase{let id = fun x -> x}

\phrase{let y = 2 in id (y)}

\phrase{let x = (fun x -> fun y -> x + y) 1 2}

\phrase{x+1}
}

\bigskip

\remarques

\begin{itemize}

  \item Les variables doivent débuter par une minuscule.

  \item La flèche est associative à \emph{droite}: l'expression\\
  $\Xfun \,\, x_1 \rightarrow \Xfun \,\, x_2 \rightarrow \ldots
  \rightarrow \Xfun \,\, x_n \rightarrow e$ \ \ \ est équivalente à\\
  $\Xfun \,\, x_1 \rightarrow \textcolor{blue}{(}\Xfun \,\, x_2 \rightarrow
  \textcolor{blue}{(}\ldots \rightarrow \textcolor{blue}{(}\Xfun \,\, x_n \rightarrow
  e\textcolor{blue}{)) \ldots)}$

\end{itemize}

\end{frame}

% ------------------------------------------------------------------------
%
\begin{frame}
\frametitle{Un programme correct (suite)}

\begin{itemize}

  \item L'appel de fonction est associatif à \emph{gauche}:
  l'expression\\ $e_1 \, e_2 \, e_3 \, \ldots \, e_n$ est équivalente
  à $\textcolor{blue}{((\ldots (}e_1 \, e_2\textcolor{blue}{)} \, e_3\textcolor{blue}{)} \ldots
  \textcolor{blue}{)} \, e_n\textcolor{blue}{)}$

  \item L'application des fonctions est prioritaire par rapport à
  celle des opérateurs:
  \begin{center}
    \texttt{f 3 + 4} \ équivaut à \ \texttt{\textcolor{blue}{(}f 3\textcolor{blue}{)} +
    4}
  \end{center}

  \item L'application des opérateurs est prioritaire par rapport à
  l'abstraction:
  \begin{center} 
    \texttt{fun x -> x + y} \ équivaut à \ \texttt{fun x ->
    \textcolor{blue}{(}x + y\textcolor{blue}{)}}
  \end{center}

\end{itemize}

\end{frame}

% ------------------------------------------------------------------------
%
\begin{frame}[containsverbatim]
\frametitle{Phrase unique}

Un programme, c.-à-d. une suite de définitions globales, peut toujours
se réécrire en une seule phrase à l'aide de définitions locales
imbriquées. 

\bigskip

L'exemple précédent est équivalent à
\begin{semiverbatim}
let x = 0 \textcolor{blue}{in}
  let id = fun x -> x \textcolor{blue}{in}
  \textcolor{blue}{let _ =} let y = 2 in id (y) \textcolor{blue}{in}
  let x = (fun x -> fun y -> x + y) 1 2 
\textcolor{blue}{in} x+1;;
\end{semiverbatim}
Le symbole \verb|_| désigne une variable dont le nom est unique et
différent de toutes les autres. Sans perte de généralité, nous
étudierons donc les programmes réduits à une seule expression. Le
résultat du programme est l'\emph{évaluation} de \verb|x+1|.

\end{frame}

% ------------------------------------------------------------------------
%
\begin{frame}[containsverbatim]
\frametitle{Un autre programme correct}
\label{un_autre_programme}

OCaml est dit fonctionnel car ses programmes sont bâtis sur des
fonctions qui modélisent les fonctions mathématiques calculables.

\bigskip

Par exemple:
{\small
\begin{verbatim}
let compose = fun f -> fun g -> fun x -> f (g x) in
  let square = fun f -> compose f f in
  let double = fun x -> x + x in
  let quad = square double 
in square quad;;
\end{verbatim}
}

\end{frame}

% ------------------------------------------------------------------------
%
\begin{frame}[containsverbatim]
\frametitle{Interprétation mathématique}
\label{maths}

{\small
\begin{alignat*}{1}
  \texttt{compose} &\equiv f \mapsto (g \mapsto (x \mapsto f(g(x))))
  \equiv f \mapsto (g \mapsto (x \mapsto f \circ g (x)))\\ 
&\equiv f \mapsto (g \mapsto f \circ g) \ \ \textnormal{car} \ \
  \forall h.(x \mapsto h(x) \equiv h)\\
  \texttt{square} &\equiv f \mapsto f \circ f \equiv f \mapsto f^2\\
  \texttt{double} &\equiv x \mapsto x + x \equiv x \mapsto 2x\\
  \texttt{quad}   &\equiv (f \mapsto f^2)(x \mapsto 2x) \equiv (x
  \mapsto 2x)^2 \equiv x \mapsto 4x\\
  \texttt{square} \, (\texttt{quad}) &\equiv (f \mapsto f^2)(x \mapsto
  4x) \equiv (x \mapsto 4x)^2 \equiv x \mapsto 16x
  \end{alignat*}
}
Donc le résultat du programme (\verb+square quad+) est une fonction.

\end{frame}

