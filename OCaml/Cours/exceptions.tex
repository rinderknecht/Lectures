%-*-latex-*-

% ------------------------------------------------------------------------
%
\begin{frame}[containsverbatim]
\frametitle{Exceptions}

Les exceptions de ML ont servi de modèle pour celles du langage \cpp. 
{\small
\begin{verbatim}
exception Perdu

let rec cherche_la_clé k = function
  (h,v)::t -> if h = k then v else cherche_la_clé k t
| [] -> raise Perdu
  
let k = 
  try 
    cherche_la_clé "Louis" [("Georges",14); ("Louis",5)]
  with Perdu -> 10
\end{verbatim}
}

\bigskip

\textbf{Exercice~5} \ Réécrire le programme sans user d'exceptions.

\end{frame}

% ------------------------------------------------------------------------
%
\begin{frame}
\frametitle{Exceptions (suite)}

\textcolor{blue}{Syntaxe}
\begin{center}
\begin{tabular}{ll}
  Définition (phrase) & \phrase{exception $C$ [of $t$]}\\
  \hline
  Lancement (expression) & \phrase{raise $e$}\\
  Filtrage (expression) & \phrase{try $e$ with $p_1 \rightarrow e_1
  \mid \ldots \mid p_n \rightarrow e_n$}
\end{tabular}
\end{center}

\bigskip

Remarquez l'analogie avec le filtrage des valeurs.

\bigskip

\textcolor{blue}{Typage}

Les exceptions sont toutes de type \textsf{exn}, qui peut être
considéré comme un type somme ouvert (de nouveaux constructeurs
peuvent être ajoutés avec des déclarations \texttt{exception}).

\end{frame}

% ------------------------------------------------------------------------
%
\begin{frame}
\frametitle{Exceptions prédéfinies}

\textcolor{blue}{\emph{The Core Library}}
\begin{center}
\small
\begin{tabular}{l|l}
Constructeur & Usage\\
\hline
\excerpt{Invalid\_argument of string} & Argument hors bornes\\
\excerpt{Failure of string} & Fonction indéfinie pour un argument\\
\excerpt{Not\_found} & Échec de fonctions de recherche\\
\excerpt{Match\_failure of ...} & Échec de filtrage\\
\excerpt{End\_of\_file} & Fin de fichier
\end{tabular}
\end{center}

\end{frame}

% ------------------------------------------------------------------------
%
\begin{frame}
\frametitle{Sémantique des exceptions}

\begin{itemize}

  \item Le type \textsf{exn} est le seul type somme \emph{extensible}.

  \item Le lancement d'une exception arrête l'évaluation et retourne
        une valeur exceptionnelle (c-à-d. de type \textsf{exn}).

  \item Une exception ne peut être éventuellement filtrée que si
        l'expression a été encadrée par un bloc \textsf{try $e$ with
        $m$}:
        \begin{itemize}
 
          \item Si l'évaluation de $e$ retourne une valeur normale,
                celle-ci est retournée sans passer par le filtre $m$.

          \item Sinon, l'exception est passée au filtre $m$. Si un des
                motifs $p_i$ filtre l'exception, alors $e_i$ est
                évaluée, sinon l'exception est propagée (\textbf{Les
                filtres d'exceptions ne sont pas forcément
                complets.}).

        \end{itemize}

   \item On peut observer une exception (c-à-d. la filtrer puis la
         relancer):\\
         \excerpt{try f x with Failure s as x -> prerr\_string s; raise x}

\end{itemize}

\end{frame}

