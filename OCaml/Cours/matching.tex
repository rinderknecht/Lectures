%-*-latex-*-

% ------------------------------------------------------------------------
%
\begin{frame}
\frametitle{Le filtre à motifs}

Étendons encore les expressions de mini-ML par les \emph{filtres} et
les \emph{motifs}:
\begin{itemize}
 
  \item \Xmatch{} $e$ \Xwith{} $p_1 \rightarrow e_1 \mid \ldots \mid
    p_n \rightarrow e_n$
 
  \item où les $p_i$ sont des motifs.

\end{itemize}

\bigskip

Les motifs sont définis récursivement par les cas suivants:

\bigskip

\begin{tabular}{rll}
    $\bullet$
  & \textcolor{blue}{variable}
  & $f, g, h$ (fonctions) et $x, y, z$ (autres). \\
    $\bullet$
  & \textcolor{blue}{unité ou constante}
  & \unit \ ou \textsf{0} ou \Xtrue{} etc.\\
    $\bullet$
  & \textcolor{blue}{$n$-uplet}
  & $p_1, \ldots, p_n$\\
    $\bullet$
  & \textcolor{blue}{parenthèse}
  & $\lpar{p}\rpar$\\
    $\bullet$
  & \textcolor{blue}{joker}
  & {\Large \_}
\end{tabular}

\bigskip

\remarque \quad Les motifs irréfutables sont des motifs particuliers.

\end{frame}

% ------------------------------------------------------------------------
%
\begin{frame}[containsverbatim]
\frametitle{Le filtre à motifs (suite)} 

Typiquement, on se sert des filtres pour faire des définitions par cas
de fonctions (mais pas uniquement), comme en mathématiques. Par exemple:
{\small
\begin{verbatim}
let rec fib n =
  match n with
    0 -> 1
  | 1 -> 1
  | _ -> fib(n-1) + fib(n-2)
\end{verbatim}
}Comme en mathématiques, les cas sont ordonnés par l'écriture et la
définition précédente se lit donc: «~Si la valeur de \texttt{n}
  \emph{a la forme} de \texttt{0} alors \texttt{fib(n)} est la valeur
  de \texttt{1}, si la valeur de \texttt{n} \emph{a la forme} de
  \texttt{1} alors \texttt{fib(n)} est la valeur de \texttt{1}, sinon
  \texttt{fib(n)} vaut la valeur de \texttt{fib(n-1) + fib(n-2)}~».

\end{frame}

% ------------------------------------------------------------------------
%
\begin{frame}
\frametitle{Le filtre à motifs (suite)} 

Que signifie la relation «~$v$ a la forme de $p$~» (ou «~$p$ filtre
$v$~»)?
\begin{itemize}

  \item Une constante a la forme de la constante identique dans un
  motif.

  \item L'unité \unit{} a la forme de \unit{} dans un motif.

  \item Un $n$-uplet a la forme d'un $n$-uplet dans un motif.

  \item \textbf{Toute valeur a la forme d'une variable de motif ou du
  joker «{\Large \_}».}

\end{itemize}

\bigskip

\remarques 

\begin{itemize}

  \item Les motifs ne filtrent aucune fermeture, c.-à-d. que $e$ dans
    \Xmatch{} $e$ \Xwith{} ne doit pas s'évaluer en une fermeture.

  \item Dans le cas des constantes et \unit, la relation se confond
  avec l'égalité.

\end{itemize}

\end{frame}

% ------------------------------------------------------------------------
%
\begin{frame}[containsverbatim]
\frametitle{Sémantique du filtrage}

Le \emph{filtrage} est l'évaluation d'un filtre. Informellement,
l'évaluation de
 
\centerline{\Xmatch{} $e$ \Xwith{} $p_1 \rightarrow e_1 \mid \ldots
  \mid p_n \rightarrow e_n$} commence par celle de $e$ en $v$.

\bigskip

Ensuite $v$ est confrontée aux différents motifs $p_i$ dans l'ordre
d'écriture. Si $p_i$ est le premier motif à filtrer $v$, alors la
valeur du filtre est la valeur de $e_i$.

\bigskip

Par exemple, voici la définition curryfiée de la disjonction logique:
{\small
\begin{verbatim}
let or = fun (x,y) ->
  match (x,y) with
   (false, false) -> false
  | _ -> true
\end{verbatim}
}

\end{frame}

% ------------------------------------------------------------------------
%
\begin{frame}[containsverbatim]
\frametitle{Les types sommes et le filtrage des leurs valeurs}

\textcolor{blue}{Les définitions de types sommes}

\bigskip

Les types sommes s'apparentent aux énumérations et aux \textsf{union}
du langage C. Par exemple, les valeurs booléennes peuvent être
définies ainsi: {\small
\begin{verbatim}
type booléen = Vrai | Faux
let v = Vrai and f = Faux
\end{verbatim}
}
\textbf{Les constructeurs ont pour première lettre une majuscule.}

\bigskip

Le \textcolor{blue}{filtrage} permet d'examiner les valeurs d'un type
somme:
{\small
\begin{verbatim}
let int_of_booléen = fun b ->
  match b with
     Vrai -> 1 
   | Faux -> 0
\end{verbatim}
}

\end{frame}

% ------------------------------------------------------------------------
%
\begin{frame}[containsverbatim]
\frametitle{Les types sommes et le filtrage de leurs valeurs (suite)}

Les constructeurs peuvent aussi transporter de l'information, dont
l'interprétation complète alors celle du constructeur lui-même. Par
exemple, un jeu de cartes peut être défini par
{\small
\begin{verbatim}
type carte = Carte of ordinaire | Joker
and ordinaire = couleur * figure
and couleur = Coeur | Carreau | Pique | Trefle
and figure = As | Roi | Reine | Valet | Simple of int
\end{verbatim}
}

\end{frame}

% ------------------------------------------------------------------------
%
\begin{frame}
\frametitle{Le jeu de cartes}

Définissons des cartes et des fonction construisant des cartes:

\bigskip

\topin{let valet\_de\_pique = Carte (Pique,Valet)}

\topout{val valet\_de\_pique : carte = Carte (Pique,Valet)}

\topin{let carte f c = Carte (c,f)}

\topout{val carte : figure $\rightarrow$ couleur $\rightarrow$ carte =
  $\topclos$}

\topin{let roi = carte Roi}

\topout{val roi : couleur $\rightarrow$ carte = $\topclos$}

\end{frame}

% ------------------------------------------------------------------------
%
\begin{frame}[containsverbatim]
\frametitle{Filtrage des cartes}

{\small
\begin{verbatim}
let valeur c = match c with
  Carte (As)       -> 14
| Carte (Roi)      -> 13
| Carte (Reine)    -> 12
| Carte (Valet)    -> 11
| Carte (Simple k) -> k
| Joker            -> 0
\end{verbatim}
}

\bigskip

Un motif peut capturer plusieurs cas:
{\small
\begin{verbatim}
let est_petite c = match c with
  Carte (Simple _) -> true
| _ -> false
\end{verbatim}
}

\end{frame}

% ------------------------------------------------------------------------
%
\begin{frame}
\frametitle{Filtres incomplets}

\textcolor{blue}{Rappel} Lorsqu'une valeur $v$ est filtrée par $p_1
\rightarrow e_1 \mid \ldots \mid p_n \rightarrow e_n$, l'expression
$e_i$ associée au premier motif $p_i$ qui filtre $v$ est évaluée dans
l'environnement courant étendu par les liaisons éventuelles créées par
le filtrage de $v$ par $p_i$.

\bigskip

Le filtre est \emph{incomplet} s'il existe au moins une valeur qui n'est
filtrée par aucun motif. Dans ce cas, un message d'avertissement est
indiqué à la compilation:

\topin{let simple c = match c with Carte (\_,Simple k) -> k}

\error{Characters 15-51}

\error{Warning: this pattern-matching is not exhaustive.}

\error{Here is an example of a value that is not matched:
  Joker}

\topout{val simple : carte $\rightarrow$ int = $\topclos$}

\textcolor{blue}{Il est préférable d'éviter les définitions incomplètes.}

\end{frame}

% ------------------------------------------------------------------------
%
\begin{frame}
\frametitle{Les motifs non linéaires}

\textcolor{blue}{Une variable ne peut être liée deux fois dans le même
  motif.}

\bigskip

\topin{fun (x,y) -> match (x,y) with  (Carte z, z) -> true}

\error{Characters 41-42}

\error{This variable is bound several times in this matching.}

\bigskip

Un tel motif est dit \emph{non linéaire}.

\end{frame}
