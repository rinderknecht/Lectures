%-*-latex-*-

% ------------------------------------------------------------------------
%
\begin{frame}
\frametitle{Évaluation des programmes mini-ML}

L'évaluation d'une expression (donc d'un programme) est une fonction
partielle des expressions vers les \emph{valeurs}. 

\bigskip

Le fait que la fonction soit partielle modélise le fait qu'une
évaluation peut ne pas terminer ou s'interrompre pour cause d'erreur.

\bigskip

Les valeurs $v$ de mini-ML sont presqu'un sous-ensemble strict des
expressions défini récursivement par les cas suivants:

\begin{tabular}{rll}
    $\bullet$
  & \textcolor{blue}{unité ou constante entière}
  & \unit \ ou \textsf{0} ou \textsf{1} ou \textsf{2} etc.\\
    $\bullet$
  & \textcolor{blue}{fermeture}
  & $\langle\Xfun \,\, x \rightarrow e, \rho\rangle$\\
 && où $\rho$ est un environnement.\\
  & 
  & Pour les opérateurs: $\langle\lpar{\texttt{+}}\rpar, \rho\rangle$ etc.
\end{tabular}

\end{frame}

% ------------------------------------------------------------------------
%
\begin{frame}
\frametitle{Fermeture, liaison et environnement (le retour)}

Une \emph{fermeture} est une sorte de paire faite d'une fonction et
d'un environnement. Cela signifie en particulier que les fonctions
font partie des valeurs, c.-à-d. qu'elles peuvent être le résultat de
l'évaluation d'un programme mini-ML (cf. exemple
page~\pageref{maths}): \emph{c'est la caractéristique d'un langage
fonctionnel}.

\bigskip

Pour définir l'évaluation nous devons modifier le concept de liaison:
une liaison associe maintenant une variable à une \emph{valeur} (et
non plus une expression).

\bigskip

L'évaluation (donc peut-être la valeur) dépendra de l'environnement au
lieu où se trouve l'expression.

Notons $(x \mapsto v) \oplus \rho$ l'ajout d'une liaison $x \mapsto v$
dans $\rho$ (en tête).

\end{frame}

% ------------------------------------------------------------------------
%
\begin{frame}
\frametitle{Évaluation des programmes mini-ML (suite)}

Pour chaque expression $e$ nous définissons une règle d'évaluation
dans un environnement $\rho$ qui donne la valeur $v$:

\begin{tabular}{rll}
    $\bullet$
  & $x$ 
  & Chercher la \emph{première} valeur associée à $x$\\
 && dans $\rho$.\\
    $\bullet$
  & $\Xfun \,\, x \rightarrow e$
  & La valeur est $\langle\Xfun \,\, x \rightarrow e, \rho\rangle$.\\
    $\bullet$
  & \texttt{+} \ \texttt{-} \ \texttt{/} \ \texttt{*}
  & La valeur est $\langle\lpar\texttt{+}\rpar, \rho\rangle$ etc.\\
    $\bullet$
  & $e_1$ \texttt{+} $e_2$ \ etc.
  & Évaluer $e_1$ et $e_2$ dans $\rho$ et sommer, etc.\\
    $\bullet$
  & \unit \ ou \textsf{0} ou \textsf{1} ou \textsf{2} etc.
  & La valeur est \unit{} ou \textsf{0} ou \textsf{1} etc.\\
    $\bullet$
  & $\lpar{e}\rpar$
  & Évaluer $e$ dans $\rho$.
\end{tabular}

\end{frame}

% ------------------------------------------------------------------------
%
\begin{frame}
\frametitle{Évaluation des programmes mini-ML (suite)}
\label{semantique_let_in}

\begin{tabular}{rll}
    $\bullet$
  & $\Xlet \,\, x = e_1 \,\, \Xin \,\, e_2$
  & Évaluer $e_1$ en $v_1$ dans $\rho$;\\
  & 
  & évaluer $e_2$ en $v$ dans $(x \mapsto v_1) \oplus \rho$.\\
    $\bullet$
  & $e_1 \,\, e_2$
  & Évaluer $e_1$ et $e_2$ en $v_1$ et $v_2$ dans $\rho$\\
  &
  & (l'ordre n'est pas spécifié);\\
  &
  & $v_1$ doit être de la forme $\langle\Xfun \,\, x \rightarrow
    e,\rho'\rangle$;\\
  &
  & évaluer $e$ dans $(x \mapsto v_2) \oplus \rho'$: la valeur est $v$.
\end{tabular}

\end{frame}

% ------------------------------------------------------------------------
%
\begin{frame}[containsverbatim]
\frametitle{Exemple d'évaluation}
\label{exemple_eval_informelle}

L'environnement est initialement vide: $\rho = \{\}$. Évaluons
{\small
\begin{verbatim}
let x = 0 in
  let id = fun x -> x in
  let y = id (x) in
  let x = (fun x -> fun y -> x + y) 1 2 
in x+1;;
\end{verbatim}
}
\begin{itemize}

  \item L'évaluation de {\small \verb|let x = 0 in ...|} impose
    d'évaluer d'abord \texttt{0} qui vaut $0$. Ensuite on crée la
    liaison $\ident{x} \mapsto 0$, on l'ajoute à~$\rho$, ce qui donne
    $\{\ident{x} \mapsto 0\}$, et on évalue la sous-expression élidée
    dans ce nouvel environnement.

  \item L'évaluation de {\small \verb|let id = fun x -> x in ...|} se
    fait dans l'environnement $\{\ident{x} \mapsto 0\}$. La valeur $v$
    est alors la fermeture $\langle\Xfun \,\, \ident{x} \rightarrow
    \ident{x}, \{\ident{x} \mapsto 0\}\rangle$. On étend
    l'environnement courant avec $\ident{id} \mapsto v$ et on évalue
    la sous-expression avec.

\end{itemize}

\end{frame}

% ------------------------------------------------------------------------
%
\begin{frame}[containsverbatim]
\frametitle{Exemple d'évaluation (suite)}

\begin{itemize}

  \item L'évaluation de {\small \verb|let y = id (x) in ...|} se fait
    dans l'environnement $\{\ident{id} \mapsto \langle\Xfun \,\,
    \ident{x} \rightarrow \ident{x}, \{\ident{x} \mapsto 0\}\rangle;
    \ident{x} \mapsto 0\}$.
  \begin{itemize}

    \item On évalue d'abord \verb|id(x)| dans l'environnement
      courant. Pour cela on évalue \texttt{id} et \texttt{x}
      séparément. Ce sont tous deux des variables, donc on cherche
      dans l'environnement la première liaison de même nom. La valeur
      de \texttt{id} est donc $\ident{id} \mapsto \langle\Xfun \,\,
      \ident{x} \rightarrow \ident{x}, \{\ident{x} \mapsto 0\}\rangle$
      et celle de \texttt{x} est 0. Il faut évaluer \texttt{x} dans
      l'environnement $(\ident{x} \mapsto 0) \oplus \{\ident{x}
      \mapsto 0\}$, ce qui donne 0.

    \item On crée la liaison $\ident{y} \mapsto 0$, on l'ajoute à
    l'environnement courant et on évalue la sous-expression élidée
    avec le nouvel environnement.
 
  \end{itemize}

  \item L'évaluation de {\small
    \verb|let x = (fun x -> fun y -> x + y) 1 2 in ...|} se fait dans
    l'environnement $\{\ident{y} \mapsto 0; \ident{id} \mapsto
    \langle\Xfun \,\, \ident{x} \rightarrow \ident{x}, \{\ident{x}
    \mapsto 0\}\rangle; \ident{x} \mapsto 0\}$.

\end{itemize}

\end{frame}

% ------------------------------------------------------------------------
%
\begin{frame}
\frametitle{Évaluation formelle des programmes mini-ML}

Si on note $\evalf{e}{\rho}$ la valeur obtenue en évaluant
l'expression $e$ dans l'environnement $\rho$, alors nous pouvons
résumer l'évaluation des programmes mini-ML ainsi:
\begin{align*}
\evalf{\overline{n}}{\rho} &= \dot{n} \ \ \textnormal{où $\overline{n}$
  est un entier mini-ML et $\dot{n} \in \mathbb{N}$}\\
\evalf{e_1 \, \texttt{+} \, e_2}{\rho} &= \evalf{e_1}{\rho} +
\evalf{e_2}{\rho} \ \ \textnormal{etc.}\\ 
\evalf{\lpar{e}\rpar}{\rho} &= \evalf{e}{\rho}\\
\evalf{x}{\rho} &= \rho(x) \ \ \textnormal{(la première liaison de $x$ dans $\rho$)}\\
\evalf{\Xfun \,\, x \rightarrow e}{\rho} &= \langle\Xfun \,\, x
\rightarrow e, \rho\rangle\\
\evalf{\Xlet \,\, x = e_1 \,\, \Xin \,\, e_2}{\rho} &= \evalf{e_2}{((x
  \mapsto \evalf{e_1}{\rho}) \oplus \rho)}\\
\evalf{e_1 \,\, e_2}{\rho} &= \evalf{e}{((x \mapsto \evalf{e_2}{\rho})
  \oplus \rho')}\\
&\phantom{=\llbracket} \textnormal{où} \ \evalf{e_1}{\rho} = \langle\Xfun \,\, x
\rightarrow e,\rho'\rangle
\end{align*}
L'évaluation consiste à appliquer les équations de la gauche vers la
droite jusqu'à la terminaison (s'il y a) ou une erreur (à
l'exécution).

\end{frame}

% ------------------------------------------------------------------------
%
\begin{frame}[containsverbatim]
\frametitle{Exemples}

On établit directement que $\evalf{\lpar\Xfun \; x \rightarrow
e_2\rpar \; e_1}{\rho} = \evalf{\Xlet \,\, x = e_1 \,\, \Xin \,\,
e_2}{\rho}$,\\ c.-à-d. que la liaison locale n'est pas nécessaire en
théorie.

\bigskip

Reprenez la description informelle de l'évaluation du programme
page~\pageref{exemple_eval_informelle} et donnez-en une description
formelle.

\bigskip

Pour mieux comprendre la nécessité des fermetures, décrivez
l'évaluation de
{\small
\begin{verbatim}
let x = 1 in
  let f = fun y -> x + y in
  let x = 2
in f(x)
\end{verbatim}
}
et du programme page~\pageref{un_autre_programme}.

\end{frame}

% ------------------------------------------------------------------------
%
\begin{frame}
\frametitle{Exercices}

\begin{itemize}

  \item Quelle différences y a-t-il entre 

  \begin{itemize}
  
    \item $f \, x \, y$

    \item $(f \, x) \, y$

    \item $f \, (x \, y)$

    \item $(f \, (x)) \, (y)$

  \end{itemize}

  \item Quelle différences y a-t-il entre 
  \begin{itemize}

    \item $f$

    \item $\Xfun \,\, x \rightarrow f \, x$

    \item $\Xfun \,\, x \, y \rightarrow f \, x \, y$

  \end{itemize}

\end{itemize}

\end{frame}

% ------------------------------------------------------------------------
%
\begin{frame}
\frametitle{Solution}

La seconde expression retarde l'évaluation jusqu'à ce que le premier
argument soit fourni.

\bigskip

La dernière retarde l'évaluation jusqu'à ce que les deux arguments
soient fournis.

\bigskip

Cela peut délaisser ou dupliquer certains calculs:

\bigskip

\excerpt{let f x = let z = print\_int x in fun y -> x + y}

\excerpt{let test g = let h = g(1) in h(2) + h(3)}

\topin{test f}

\topout{1- : int = 7}

\topin{test (fun x y -> f x y)}

\topout{11- : int = 7}

\end{frame}

% ------------------------------------------------------------------------
% 
\begin{frame}[containsverbatim]
\frametitle{Applications partielles et complètes}
\label{application_partielle}

Si les fonctions sont des valeurs, elles peuvent être retournées en
résultat (c.-à-d. être la valeur d'une application). Par exemple:
{\small
\begin{verbatim}
let add = fun x -> fun y -> x + y in
  let incr = add 1                     (* incr est une fonction *)
in incr 5;;
\end{verbatim}
} On parle d'application \emph{partielle}, par opposition à
application \emph{complète}, qui ne retourne pas de fonction, comme
(\verb|add 1 5|). Les opérateurs peuvent aussi être utilisés en
position préfixe dans une expression en les parenthèsant:
$\lpar\texttt{+}\rpar \; \num{1} \; \num{2}$. Comme toute application,
les opérations peuvent être aussi partiellement évaluées: {\small
\begin{verbatim}
let incr = (+) 1  (* incr est une fonction *)
in incr 5;;
\end{verbatim}
}

\end{frame}

% ------------------------------------------------------------------------
%
\begin{frame}[containsverbatim]
\frametitle{Non-terminaison}

En théorie, nous pouvons d'ores et déjà calculer avec mini-ML
tout ce qui est calculable avec l'ordinateur sous-jacent. Par exemple,
nous avons déjà la récurrence, comme le montre le programme suivant
qui ne termine pas:
\begin{center}
\verb|let omega = fun f -> f f in omega omega|
\end{center}
Cela se manifeste par
\begin{gather*}
\evalf{\texttt{\small let omega = fun f -> f f in omega omega}}{\rho}\\
\begin{align*}
&= \evalf{\texttt{\small omega omega}}{((\texttt{\small omega} \mapsto
\evalf{\texttt{\small fun f -> f f}}{\rho}) \oplus \rho)}\\
&= \evalf{\texttt{\small f f}}{((\texttt{\small f} \mapsto
    \langle\texttt{\small fun f -> f f},\rho\rangle) \oplus \rho)}\\
&= \textnormal{\emph{idem}}
\end{align*}
\end{gather*}

\end{frame}

% ------------------------------------------------------------------------
% 
\begin{frame}[containsverbatim]
\frametitle{Fonctions récursives}

Pour mettre en évidence la puissance de mini-ML, voyons comment
définir des fonctions récursives à l'aide de la fonction
auto-applicative \texttt{\small omega}.

\bigskip

Définissons d'abord une fonction \texttt{\small fix},
traditionnellement appelée le \emph{combinateur Y de point fixe de
Curry}:
\begin{verbatim}
let omega = fun f -> f f;;
let fix = fun g -> omega (fun h -> fun x -> g (h h) x);;
\end{verbatim}

\end{frame}

% ------------------------------------------------------------------------
% 
\begin{frame}
\frametitle{Point fixe d'une fonction}

On démontre (péniblement) que pour toute fonction $f$ et variable $x$:
$$\evalf{\lpar\ident{fix} \; f\rpar{} \, x}{\rho} = \evalf{f \;
  \lpar\ident{fix} \; f\rpar{} \, x}{\rho}$$

En d'autre termes, pour tout \ident{x} on a $\lpar\ident{fix} \; f\rpar{}
\; x \, \equiv f \; \lpar\ident{fix} \; f\rpar{} \; x$, soit
$\lpar\ident{fix} \; f\rpar{} \equiv f \; \lpar\ident{fix} \; f\rpar$

\bigskip

D'autre part, par définition, le point fixe $p$ d'une fonction $f$
vérifie $p = f(p)$. Donc le point fixe d'une fonction \ident{f},
\emph{s'il existe}, est \lpar\ident{fix} \ident{f}\rpar{}.

\bigskip

Il est possible de définir la sémantique (c.-à-d. l'évaluation) d'une
famille d'opérateurs de point fixes (et non d'un seul comme
précédemment) en posant qu'un tel opérateur doit satisfaire
$$\evalf{\ident{fix} \,\, e}{\rho} = \evalf{e_1}{(f \mapsto
  \evalf{\ident{fix} \, \lpar\Xfun \, f \rightarrow e_1\rpar}{\rho}
  \oplus \rho')}$$ où $\evalf{e}{\rho} = \langle\Xfun \, f \rightarrow
e_1,\rho'\rangle$

\end{frame}

% ------------------------------------------------------------------------
% 
\begin{frame}[containsverbatim]
\frametitle{Factorielle et cas général}

% Posons
{\small
\begin{semiverbatim}
let pre_fact =
  \textcolor{blue}{fun f ->} fun n -> if n=1 then 1 else n * \textcolor{blue}{f}(n-1);;
let fact = fix pre_fact;;
\end{semiverbatim}
}
Donc \texttt{\small fact} est le point fixe de \texttt{\small
pre\_fact}, s'il existe, c'est-à-dire
\begin{align*}
\evalf{\texttt{\small fact}}{\rho} &= \evalf{\texttt{\small pre\_fact
    (fact)}}{\rho}\\
&= \evalf{\texttt{\small fun n -> if n=1 then 1 else n * fact(n-1)}}{\rho}
\end{align*}
Donc \texttt{\small fact} est la fonction factorielle (équation de
récurrence). On peut alors prédéfinir un opérateur de point fixe
\texttt{\small fix} (qui n'est pas forcément celui de Curry) et
permettre au programmeur de s'en servir directement, mais nous allons
plutôt doter mini-ML d'une liaison récursive native:
$$\evalf{\Xlet \, \Xrec \; f = e_1 \, \Xin \, e_2}{\rho} =
\evalf{\Xlet \; f = \textsf{fix} \, \lpar\Xfun \, f \rightarrow
e_1\rpar \, \Xin \, e_2}{\rho}$$

\end{frame}
