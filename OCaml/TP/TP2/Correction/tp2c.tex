%%-*-latex-*-

\documentclass[a4paper]{article}

\usepackage[francais]{babel}
\usepackage[T1]{fontenc}
\usepackage[latin1]{inputenc}
\usepackage{amssymb}

/home/rinderkn/LaTeX/TeX/trace.tex

\title{Corrig� du TP 2 de programmation fonctionnelle en Objective Caml}
\author{Christian Rinderknecht}

\begin{document}

\maketitle

\section{Curryfication}

\begin{verbatim}
let curry f = fun x y -> f (x, y);;
let uncurry f = fun (x, y) -> f x y;;
\end{verbatim}


\section{Param�tres fonctionnels}

\begin{verbatim}
let fun_prod f g = fun (x, y) -> (f x, g y);;

let rec iter n f x =
  if n = 0 then x else f (iter (n-1) f x)
;;
\end{verbatim}

\noindent On aurait pu �crire la derni�re ligne:
\verb|iter (n-1) f (f x)|

\medskip

\noindent La version na�ve de Fibonnacci est
\begin{verbatim}
let rec fib n = if n < 2 then 1 else fib (n-1) + fib (n-2);;
\end{verbatim}

\noindent Soit $A_n$ le nombre d'appels r�cursifs pour calculer
\texttt{fib(n)}. Alors nous avons
$$A_0 = A_1 = 1 \ \textnormal{et} \ \forall n > 1, A_n = 1 + A_{n-1} +
A_{n-2}$$
Posons $B_n = A_n + 1$ et il vient:
$$ B_0 = B_1 = 2 \ \textnormal{et} \ \forall n > 1, B_n = B_{n-1} +
B_{n-2} $$
On prouve par r�currence sur $n$ que $\forall n \in \mathbb{N}, B_n =
2 F_n$. Par cons�quent $A_n = 2 F_n - 1$. Le temps d'ex�cution de
\texttt{fib(n)} est donc asymptotiquement une exponentielle de $n$.

\medskip

\noindent
Si l'on pose $f : (x,y) \mapsto (x+y,x)$, alors $(F_{n+2},F_{n+1}) = f
(F_{n+1},F_{n})$.

\medskip

\noindent 
La suite $(F_{n+1},F_{n})$ pour tout $n \in \mathbb{N}$ vaut donc
$(F_{n+1},F_{n}) = f^{n}(F_1,F_0)$, soit $(F_{n+1},F_{n}) =
f^{n}(1,1)$. Nous pouvons alors la programmer directement et
efficacement (le nombre d'appels est lin�aire en $n$ maintenant):

\begin{verbatim}
let fib n = fst (iter n (fun (x,y) -> (x+y,x)) (1,1));;
\end{verbatim}

\noindent La fonction \texttt{fst} est pr�d�finie mais peut se d�finir
trivialement par:
\begin{verbatim}
let fst (x,y) = x;;
\end{verbatim}
\begin{verbatim}
let power m n = iter n (fun x -> m * x) 1;;

let fib n = fst (iter n (fun (x,y) -> (x+y,x)) (1,0));;

let iter_prod f g n = iter n (fun_prod f g);;

let iter_prod_bis f g n = fun_prod (iter n f) (iter n g);;

let rec loop f p z = if (p z) then z else loop f p (f z);;

let rec modulo x y = loop (fun v -> v - y) (fun v -> v < y) x;;
\end{verbatim}
\begin{verbatim}
let iter_bis n f x =
  fst (loop (fun_prod f pred) 
            (fun (_, k) -> k = 0) 
            (x, n))
\end{verbatim}
Dans la fonction {\tt iter\_bis}, on utilise {\tt loop} pour calculer
la suite $(f^k(x), n-k)_{k \geq 0}$ et on s'arr�te lorsque la deuxi�me
composante s'annule. On obtient donc la paire $(f^n(x), 0)$, dont on
extrait la premi�re composante gr�ce � {\tt fst}. Cette fonction est
pr�d�finie mais peut se d�finir trivialement, ainsi que \texttt{pred}
qui calcule le pr�d�cesseur d'un entier par:
\begin{verbatim}
let fst (x,y) = x
let pred x = x - 1
\end{verbatim}

\section{Filtrage}

\noindent Factorielle:
\begin{verbatim}
let rec fact n = match n with
  0 -> 1
| _ -> n * fact(n-1)
;;
\end{verbatim}

\noindent ou bien
\begin{verbatim}
let rec fact n = 
  match n with
    0 -> 1
  | _ -> n * fact(n-1)
;;
\end{verbatim}

\noindent Op�rations sur les matrices:
\begin{verbatim}
let transpose = fun ((x, y), (z, t)) -> ((x, z), (y, t));;

let prod ((x, y), (z, t)) ((x', y'), (z', t')) =
  ((x *. x' +. y *. z', x *. y' +. y *. t'),
   (z *. x' +. t *. z', z *. y' +. t *. t'))
;;

let is_const ((x, y), (z, t)) = (x = y) & (y = z) & (z = t)
;;
\end{verbatim}

\noindent On ne peut pas utiliser un simple filtrage pour d�terminer
si les quatre �l�ments sont �gaux (cf. �nonc�).
\begin{verbatim}
let trig_sup m = 
  match m with
    (_, (0.0, _)) -> true
  | _ -> false
;;
\end{verbatim}


\end{document}
