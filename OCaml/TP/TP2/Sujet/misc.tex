%%-*-latex-*-

\section{Calcul approch� de la d�riv�e d'une fonction}

\'{E}crire une fonction {\tt d�riv�e} � trois arguments:
\begin{itemize}
\item le pas {\tt epsilon} utilis� pour le calcul approch�;
\item la fonction {\tt f} dont on veut calculer la d�riv�e;
\item le point {\tt x} auquel on veut calculer la d�riv�e.
\end{itemize}

\section{Calcul approch� d'une int�grale}

\'{E}crire une fonction {\tt int�grale} � quatre arguments:
\begin{itemize}
\item le nombre de sous-intervalles � utiliser pour le calcul;
\item les bornes de l'intervalle consid�r�;
\item la fonction � int�grer.
\end{itemize}


\section{Calcul du z�ro d'une fonction par la m�thode de Newton}

\'{E}crire une fonction {\tt newton} � trois arguments:
\begin{itemize}
\item la pr�cision requise {\tt epsilon}, de type {\tt float};
\item la fonction {\tt f}, de type $\mbox{\tt float}\rightarrow\mbox{\tt float}$, dont on veut trouver un z�ro;
\item le point de d�part {\tt x}, de type {\tt float}, de la recherche.
\end{itemize}
La m�thode de Newton consiste � r�aliser une suite d'approximations d�finie par:
\begin{itemize}
\item $x_0 = \mbox{\tt x}$
\item $x_{n+1} = x_n - \frac{{\mbox{\tt f}}(x_n)}{{\mbox{\tt f}}'(x_n)}$
\end{itemize}
On s'arr�te lorsque $\mid x_{n+1} - x_n \mid \,\leq \mbox{\tt epsilon}$.

Applications:
\begin{itemize}
\item �crire une fonction � un argument {\tt x} qui calcule une
      approximation de $\sqrt{\mbox{\tt x}}$.  Comparer au r�sultat
      obtenu en utilisant la fonction {\tt sqrt} de Caml.
\item Calculer une valeur approch�e de $x$ tel que
      $$\int_0^x \sin t.dt = 1$$
\end{itemize}



\item R��crire la fonction {\tt newton} en utilisant la fonction {\tt loop}.
