\paragraph{Question.} Consider the following \Lex specification:
{\small
\begin{verbatim}
%{
#include<string.h>
char* keyword[] = {"else", "if", "return", "then"};
%}
num  (\.[0-9]+|[+\-]?[0-9]+(\.[0-9]+)?)(E[+\-]?[0-9]+)?
id   [A-Za-z]([_]*[A-Za-z0-9])*
ws   [ \n\t]+
%%
{num}  { printf ("num<%s>\n",yytext); }
{id}   { int index = 0;
         while (index <= 3 
                && strcmp(keyword[index],yytext)) index++;
         if (index == 4) printf("id<%s>\n",yytext);
         else printf("kwd<%s>\n",yytext);
       }
{ws}   {}
"+"    { printf ("plus<>\n"); }
%%
\end{verbatim}
}

\noindent Some possible transition diagrams corresponding to the
regular expressions \texttt{num}, \texttt{id} and \texttt{ws} are
respectively given in figure~\ref{num_diagram}, \ref{id_diagram}
and~\ref{ws_diagram}.

\begin{figure}
\begin{center}
\includegraphics[bb=48 618 335 756]{dfa_num_ext}
\caption{Transition diagram for regular expression \texttt{num}}
\label{num_diagram}
\end{center}
\end{figure}

\begin{figure}
\begin{center}
\includegraphics[bb=48 701 263 757]{dfa_id_ext}
\caption{Transition diagram for regular expression \texttt{id}}
\label{id_diagram}
\end{center}
\end{figure}

\begin{figure}
\begin{center}
\includegraphics[bb=48 710 205 732]{dfa_ws_opt}
\caption{Transition diagram for regular expression \texttt{ws}}
\label{ws_diagram}
\end{center}
\end{figure}

\bigskip

Define the meaning of the pointers \(\upharpoonleft\),
\(\upharpoonright\) and \(\Uparrow\) presented in class and show
how the input
{\small
\begin{alltt}
return\textvisiblespace\textvisiblespace{}x+\textvisiblespace{}.5E2+y_0
\end{alltt}
}
\noindent where \textvisiblespace{} represents a blank character, is
analysed using the transition diagrams. Make clear of which diagram
you are referring to.
