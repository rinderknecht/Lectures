%%-*-latex-*-

% ------------------------------------------------------------------------
% 
\begin{frame}
\frametitle{Deterministic finite automata}
 
Transition diagrams are useful \emph{graphical} representations of
instances of the mathematical concept of \textbf{deterministic finite
automaton} (\textbf{DFA}).

\bigskip

Formally, a DFA \({\cal D}\) is a 5-tuple \({\cal D} = (Q, \Sigma,
\delta, q_0, F)\) where
\begin{enumerate}

  \item a finite set of \emph{states}, often noted \(Q\);

  \item an \emph{initial state} \(q_0 \in Q\);

  \item a set of \emph{final (\emph{or} accepting)
  states} \(F \subseteq Q\);

  \item a finite set of \emph{input symbols}, often noted \(\Sigma\);

  \item a \emph{transition function} \(\delta\) that takes 
  a state and an input symbol and returns a state: if \(q\) is a state
  with an edge labeled \(a\), the edge leads to state \(\delta(q,
  a)\).

\end{enumerate}

\end{frame}

% ------------------------------------------------------------------------
% 
\begin{frame}
\frametitle{DFA/Recognised words}

Independently of the interpretation of the states, we can define how a
given word is accepted (or recognised) or rejected by a given DFA.

\bigskip

\noindent The word \(a_1 a_2 \cdots a_n\), with \(a_i \in \Sigma\), is
recognised by the DFA \({\cal D} = (Q, \Sigma, \delta, q_0, F)\) if
\begin{itemize}

  \item for all \(0 \leqslant i \leqslant n-1\)

  \item there is a sequence of states \(q_i \in Q\) such as 

  \item \(\delta (q_i, a_{i+1}) = q_{i+1}\) 

  \item and \(q_n \in F\).

\end{itemize}
The language recognised by \(\cal D\), noted \(L({\cal D})\) is the
set of words recognised by \(\cal D\).

\end{frame}

% ------------------------------------------------------------------------
% 
\begin{frame}
\frametitle{DFA/Recognised words/Example}

\begin{columns}
  \column{0.5\textwidth}
  For example, consider the following DFA:
  \begin{center}
    \includegraphics[bb=48 644 218 730,scale=0.93]{trie_then}
  \end{center}

  \column{0.5\textwidth} The word ``\textsf{then}'' is recognised
  because there is a sequence of states\linebreak\((q_0, q_1, q_2, q_4, q_5)\)
  connected by edges which satisfies
  \begin{align*}
    \delta (q_0, \textsf{t}) &= q_1\\
    \delta (q_1, \textsf{h}) &= q_2\\
    \delta (q_2, \textsf{e}) &= q_4\\
    \delta (q_4, \textsf{n}) &= q_5
  \end{align*}
  and \(q_5 \in F\), i.e. \(q_5\) is a final state.
\end{columns}

\end{frame}

% ------------------------------------------------------------------------
% 
\begin{frame}
\frametitle{DFA/Recognised language}

It is easy to define formally \(L({\cal D})\).

\bigskip

Let \({\cal D} = (Q, \Sigma, \delta, q_0, F)\).

\bigskip

First, let us extend \(\delta\) to words and let us call this
extension \(\hat{\delta}\):
\begin{itemize}

  \item for all state \(q \in Q\), let \(\hat{\delta} (q,
  \varepsilon) = q\), where \(\varepsilon\) is the empty string;

  \item for all state \(q \in Q\), all word \(w \in \Sigma^{*}\), all
 input \(a \in \Sigma\), let \(\hat{\delta} (q, wa) = \delta
 (\hat{\delta}(q,w),a)\).

\end{itemize}
Then the word \(w\) is recognised by \(\cal D\) if \(\hat{\delta}(q_0,
w) \in F\).

The language \(L({\cal D})\) recognised by \(\cal D\) is defined as
\[
L({\cal D}) = \{w \in \Sigma^{*} \; \lvert \; \hat{\delta}(q_0, w) \in F\}
\]

\end{frame}

% ------------------------------------------------------------------------
% 
\begin{frame}
\frametitle{DFA/Recognised language/Example}

For example, in our last example:
\[
\begin{aligned}
   \hat{\delta}(q_0, \epsilon)
&= q_0\\
   \hat{\delta}(q_0, \textsf{t})
&= \delta (\hat{\delta}(q_0, \epsilon), \textsf{t})
= \delta (q_0, \textsf{t}) 
= q_1\\
   \hat{\delta}(q_0, \textsf{th})
&= \delta (\hat{\delta}(q_0, \textsf{t}), \textsf{h})
= \delta (q_1, \textsf{h})
= q_2\\
   \hat{\delta}(q_0, \textsf{the})
&= \delta (\hat{\delta}(q_0, \textsf{th}), \textsf{e})
= \delta (q_2, \textsf{e})
= q_4\\
   \hat{\delta}(q_0, \textsf{then}) 
&= \delta (\hat{\delta}(q_0, \textsf{the}), \textsf{n})
= \delta (q_4, \textsf{n})
= q_5 \in F
\end{aligned}
\]

\end{frame}

% ------------------------------------------------------------------------
% 
\begin{frame}
\frametitle{DFA/Transition diagrams}
 
We can also redefine transition diagrams in terms of the concept of
DFA.

\bigskip

A transition diagram for a DFA \({\cal D} = (Q, \Sigma, \delta, q_0,
F)\) is a graph defined as follows:
\begin{enumerate}

  \item for each state \(q\) in \(Q\) there is a \textbf{node}, i.e. a
    single circle with \(q\) inside;

  \item for each state \(q \in Q\) and each input symbol \(a \in
    \Sigma\), if \(\delta (q, a)\) exists, then there is an
    \textbf{edge}, i.e. an arrow, from the node denoting \(q\) to the
    node denoting \(\delta (q, a)\) labeled by \(a\); multiple edges
    can be merged into one and the labels are then separated by
    commas;

  \item there is an edge coming to the node denoting \(q_0\) without
    origin;

  \item nodes corresponding to final states (i.e. in \(F\)) are
    double-circled.

\end{enumerate}

\end{frame}

% ------------------------------------------------------------------------
% 
\begin{frame}
\frametitle{DFA/Transition diagram/Example}

Here is a transition diagram for the language over alphabet \(\{0,
1\}\), called \textbf{binary alphabet}, which contains the string
\(01\):
\begin{center}
\includegraphics[bb=48 710 185 760]{dfa_01}
\end{center} 

\end{frame}

% ------------------------------------------------------------------------
% 
\begin{frame}
\frametitle{DFA/Transition table}

There is a compact textual way to represent the transition function of
a DFA: a \textbf{transition table}.

\bigskip

The rows of the table correspond to the states and the columns
correspond to the inputs (symbols). In other words, the entry for
the row corresponding to state \(q\) and the column corresponding to
input \(a\) is the state \(\delta (q, a)\):
\[
\begin{array}{c||c|c|c}
\delta & \ldots & a & \ldots\\
\hhline{=::===}
\vdots & & &\\
\hline
q & & \delta (q, a)\\
\hline
\vdots & & &
\end{array}
\]

\end{frame}

% ------------------------------------------------------------------------
% 
\begin{frame}
\frametitle{DFA/Transition table/Example}

The transition table corresponding to the function \(\delta\) of our
last example is
\[
\begin{array}{r@{}l||c|c}
\multicolumn{2}{c||}{\cal D} & 0 & 1\\
\hhline{==::==}
\rightarrow & q_0 & q_1 & q_0\\
            & q_1 & q_1 & q_2\\
         \# & q_2 & q_2 & q_2
\end{array}
\]
Actually, we added some extra information in the table: the initial
state is marked with \(\rightarrow\) and the final states are marked
with \(\#\).

\bigskip

Therefore, it is not only \(\delta\) which is defined by means of the
transition table here, but the whole DFA \(\cal D\).

\end{frame}

% ------------------------------------------------------------------------
% 
\begin{frame}
\frametitle{DFA/Example}

We want to define formally a DFA which recognises the language \(L\)
whose words contain an even number of 0's and an even number of 1's
(the alphabet is binary).

\bigskip

We should understand that the role of the states here is to
\textbf{not} to count the exact number of 0's and 1's that have been
recognised before but this number \textbf{modulo 2}.

\bigskip

Therefore, there are four states because there are four cases:
\begin{enumerate}

  \item there has been an even number of 0's and 1's (state \(q_0\));

  \item there has been an even number of 0's and an odd number of 1's
    (state \(q_1\));

  \item there has been an odd number of 0's and an even number of 1's
    (state \(q_2\));

  \item there has been an odd number of 0's and 1's (state \(q_3\)).

\end{enumerate}

\end{frame}

% ------------------------------------------------------------------------
% 
\begin{frame}
\frametitle{DFA/Example (cont)}

What about the initial and final states?
\begin{itemize}

  \item State \(q_0\) is the initial state because before considering
  any input, the number of 0's and 1's is zero and zero is even.

  \item State \(q_0\) is the lone final state because its definition
  matches exactly the characteristic of \(L\) and no other state
  matches.

\end{itemize}
We know now almost how to specify the DFA for language \(L\). It is
\[
{\cal D} = (\{q_0,q_1,q_2,q_3\}, \{0,1\}, \delta,q_0, \{q_0\})
\]
where the transition function \(\delta\) is described by the following
transition diagram.

\end{frame}

% ------------------------------------------------------------------------
% 
\begin{frame}
\frametitle{DFA/Example (cont)}

\begin{columns}
  \column{0.5\textwidth}
  \begin{center}
    \includegraphics[bb=48 653 138 737]{dfa_even01}
  \end{center}

  \column{0.5\textwidth} Notice how each input 0 causes the state to
  cross the horizontal line.

  \bigskip

  Thus, after seeing an even number of 0's we are always above the
  horizontal line, in state \(q_0\) or \(q_1\), and after seeing an
  odd number of 0's we are always below this line, in state \(q_2\) or
  \(q_3\).

  \bigskip

  There is a vertically symmetric situation for transitions on 1.
\end{columns}

\end{frame}

% ------------------------------------------------------------------------
% 
\begin{frame}
\frametitle{DFA/Example (cont)}

We can also represent this DFA by a transition table:
\[
\begin{array}{r@{}l||c|c}
\multicolumn{2}{c||}{\cal D} & 0 & 1\\
\hhline{==::==}
\# \rightarrow & q_0 & q_2 & q_1\\
               & q_1 & q_3 & q_0\\
               & q_2 & q_0 & q_3\\
               & q_3 & q_1 & q_2
\end{array}
\]
We can use this table to illustrate the construction of
\(\hat{\delta}\) from \(delta\). Suppose the input is
\(110101\). Since this string has even numbers of 0's and 1's, it
belongs to \(L\), i.e. we expect \(\hat{\delta}(q_0,110101) = q_0\),
since \(q_0\) is the sole final state.

\end{frame}

% ------------------------------------------------------------------------
% 
\begin{frame}[containsverbatim]
\frametitle{DFA/Example (cont)}

We can check this by computing step by step
\(\hat{\delta}(q_0,\verb+110101+)\), from the shortest prefix to the
longest (which is the word \verb+110101+ itself):
\begin{align*}
  \hat{\delta} (q_0, \varepsilon) 
&= q_0\\
   \hat{\delta} (q_0, \texttt{1}) 
&= \delta (\hat{\delta} (q_0, \varepsilon), \texttt{1})
&= \delta (q_0, \texttt{1})
&= q_1\\
   \hat{\delta} (q_0, \texttt{11}) 
&= \delta (\hat{\delta} (q_0, \texttt{1}), \texttt{1}) 
&= \delta (q_1, \texttt{1}) 
&= q_0\\
   \hat{\delta} (q_0, \texttt{110}) 
&= \delta (\hat{\delta} (q_0, \texttt{11}), \texttt{0}) 
&= \delta (q_0, \texttt{0}) 
&= q_2\\
   \hat{\delta} (q_0, \texttt{1101}) 
&= \delta (\hat{\delta} (q_0, \texttt{110}), \texttt{1}) 
&= \delta (q_2, \texttt{1}) 
&= q_3\\
   \hat{\delta} (q_0, \texttt{11010}) 
&= \delta (\hat{\delta} (q_0, \texttt{1101}), \texttt{0}) 
&= \delta (q_3, \texttt{0}) 
&= q_1\\
   \hat{\delta} (q_0, \texttt{110101}) 
&= \delta (\hat{\delta} (q_0, \texttt{11010}), \texttt{1}) 
&= \delta (q_1, \texttt{1}) 
&= q_0 \in F
\end{align*}

\end{frame}
