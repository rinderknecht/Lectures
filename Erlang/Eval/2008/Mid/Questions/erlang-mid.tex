%%-*-latex-*-

\documentclass[11pt,a4paper]{article}

\usepackage[british]{babel}
\usepackage[T1]{fontenc}
\usepackage[latin1]{inputenc}
\usepackage{amsmath,amssymb}
\usepackage{array}
\usepackage{multirow}
\usepackage{verbatim}

/home/rinderkn/LaTeX/TeX/trace.tex
/home/rinderkn/LaTeX/TeX/erlc.tex

\title{Final examination\\ of Introduction to Networking}

\author{Christian Rinderknecht}
\date{25 April 2008}

\begin{document}

\maketitle

\thispagestyle{empty}

\noindent A \emph{queue} is a like stack where items are pushed on one
end and popped on the other end. Adding an item in a queue is called
\emph{enqueuing}, whereas removing one is called
\emph{dequeuing}. Compare the following figures:
\[
\begin{array}{rr@{\;}cc|c|c|c|c|c|cc@{\;}l}
\cline{4-10}
\text{Queue}: & \textsc{Enqueue} & \rightarrow & & a & b & c & d & e &
& \rightarrow & \textsc{Dequeue}\\
\cline{4-10}\\
\cline{4-9}
\text{Stack}: & \textsc{Push}, \textsc{Pop} & \leftrightarrow & & a &
b & c & d & e &&&\\
\cline{4-9}
\end{array}
\]
\noindent Let \texttt{enqueue(E,Q)} be the queue where \texttt{E} is
the last item in and \texttt{Q} is the remaining queue. Let
\texttt{dequeue(Q)} be the pair \texttt{\{E,R\}} where \texttt{E} is
the first item out of queue \texttt{Q} (if there is none,
\texttt{dequeue(Q)} is undefined) and \texttt{R} is the remaining
queue.

\bigskip

\paragraph{Question.} Design a protocol to be used between an automatic teller
machine (ATM) and a bank's centralised computer. Your protocol should
allow a user's card and password to be verified, the account balance
(which is maintained at the bank) to be queried, and an account
withdrawal to be made (money is given to the
customer). \textbf{1.~Specify your protocol} by listing the messages
exchanged and the action taken by the ATM or the bank's centralised
computer on transmission or receipt of each message:
\begin{center}
\begin{tabular}{@{}p{170pt}p{170pt}@{}}
\toprule
\multicolumn{2}{c}{From ATM to Bank}\\
\cmidrule(r{0pt}){1-2}
Message name & Meaning/Action\\
and arguments &\\
\midrule
 & \\
 & \\
 & \\
 & \\
 & \\
 & \\
 & \\
 & \\
\bottomrule
\end{tabular}

\begin{tabular}{@{}p{170pt}p{170pt}@{}}
\toprule
\multicolumn{2}{c}{From Bank to ATM}\\
\cmidrule(r{0pt}){1-2}
Message name & Meaning/Action\\
and arguments &\\
\midrule
 & \\
 & \\
 & \\
 & \\
 & \\
 & \\
 & \\
 & \\
\bottomrule
\end{tabular}
\end{center}
\noindent\textbf{2.~Sketch the operation of your protocol}, using a
diagram, for the cases of a simple withdrawal \textbf{(a)} with no
errors, \textbf{(b)} with one error.

Give the Morris-Pratt algorithm, assuming you know the supply function.


\end{document}
