%%-*-latex-*-

\documentclass[11pt,a4paper]{article}

\usepackage[british]{babel}
\usepackage[T1]{fontenc}
\usepackage[latin1]{inputenc}
\usepackage{amsmath,amssymb}

\title{Test 1d of \textsf{Erlang}}
\author{Christian Rinderknecht}
\date{7 October 2011}

\newcommand\fun[1]{\textsf{#1}}

\begin{document}

\maketitle

\thispagestyle{empty}

\begin{enumerate}

  \item Write a function \fun{filter/1} such that, if
    \[\fun{filter}(P) \xrightarrow{+} \{R,S\},\]
    then \(R\)~is the list of the non-list items of~\(P\) and \(S\)~is
    the list of the list in~\(P\), in the same order as in~\(P\). For
    example,
    \begin{align*}
      \fun{filter}([\fun{a},1,[\,],2,[\fun{b},[3]]])
    &\xrightarrow{+} \{[\fun{a},1,2],[[\,],[\fun{b},[3]]]\}.\\
      \fun{filter}([\fun{a},1,2])
    &\xrightarrow{+} \{[\fun{a},1,2],[\,]\}.\\
      \fun{filter}([\,])
    &\xrightarrow{+} \{[\,],[\,]\}.
    \end{align*}
    
  \item Strings in \textsf{Erlang} are like in \textsf{Java}. For
    example, \fun{"This is a string"}. Concatenation of strings is
    performed by the \fun{++} operator, as in \textsf{Java}; for
    instance, \fun{"This is a" ++ " fruit"} evaluates in \fun{"This is
      a fruit"}. Write a function \fun{punc/1} (\emph{punctuates})
    such that, if
    \[\fun{punc}(P) \xrightarrow{+} S,\]
    then \(S\)~is the string made of the concatenation of the strings
    in the list~\(P\), separated by commas, except the last two
    strings, which require "and". The resulting string is terminated
    by a period as well, like in English. For example,
    \[
    \fun{punc}([\textnormal{"One"}, \textnormal{"two"},
      \textnormal{"three"}, \textnormal{"four"}]) \xrightarrow{+}
    \textnormal{"One, two, three and four."}\] If there is only one
    string in~\(P\), only the terminal period is required, for
    instance:
    \[\fun{punc}([\textnormal{"One"}]) \xrightarrow{+} \textnormal{"One."}\] If \(P\)~is empty,
    the result is the empty string \fun{""} (no period inside).

\end{enumerate}

\end{document}
