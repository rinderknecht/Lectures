%%-*-latex-*-

\documentclass[11pt,a4paper]{article}

\usepackage[british]{babel}
\usepackage[T1]{fontenc}
\usepackage[latin1]{inputenc}
\usepackage{amsmath,amssymb}

\usepackage{array}
\usepackage{multirow}

\title{Test 1f of \textsf{Erlang}}
\author{Christian Rinderknecht}
\date{7 October 2011}

\newcommand\fun[1]{\textsf{#1}}

\begin{document}

\maketitle

\thispagestyle{empty}


\begin{enumerate}

  \item Write a function \fun{suf/2} (\emph{suffix}) such that, if
    \[\fun{suf}(P,Q) \xrightarrow{+} R,\]
    and list \(P\) is the ending of list~\(Q\) then \(R\)~is
    \fun{true}, otherwise \fun{false}. For example,
    \begin{align*}
      \fun{suf}([1,1,0,0],[0,1,1,0,1,1,1,0,0])
      &\xrightarrow{+} \fun{true},\\
      \fun{suf}([\fun{c},\fun{i},\fun{t},\fun{y}],
      [\fun{a}, \fun{t}, \fun{r}, \fun{o}, \fun{c}, \fun{i}, \fun{t},
        \fun{y}]
      & \xrightarrow{+} \fun{true},\\
      \fun{suf}([[2],5,\fun{a}],[4,\fun{a}])
      &\xrightarrow{+} \fun{false},\\
      \fun{suf}([\,],[1,2,3])
      &\xrightarrow{+} \fun{true}.
    \end{align*}
    Note that the empty list is the suffix of \emph{any} list.

  \item Write a function \fun{allp/1} (\emph{all prefixes}) such that,
    if
    \[\fun{allp}(P) \xrightarrow{+} Q,\] then \(Q\) is the list of all
    the prefixes of \(P\), by increasing length. A prefix is a list
    whose items are found in the beginning of another list. For
    example,
    \[\fun{allp}([1,2,3,4]) \xrightarrow{+}
          [[1],[1,2],[1,2,3],[1,2,3,4]].\]

\end{enumerate}

\end{document}
