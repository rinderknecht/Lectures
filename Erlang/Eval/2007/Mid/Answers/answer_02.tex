\paragraph{Answer.} In the one-stack implementation, every time
we dequeue, we must traverse all the items in the stack. In the
two-stack implementation, we only traverse the second stack, which is
always shorter or equal than the stack in the first case.
\begin{enumerate}

  \item The best case for the one-stack implementation is when the
    queue contains only one item, and the best case for the two-stack
    implementation is when the second stack is not empty: any
    dequeuing then has a constant cost (a pop).

  \item All configurations of the one-stack implementation is the
    worst case: the cost of dequeuing is always the size of the stack,
    whilst the worst case for the two-stack implementation is when the
    second stack is empty, in which case the first stack has to be
    reversed, so the total cost is the number of items in the queue
    (which are then all in the first stack).

\end{enumerate}
As a conclusion, the two-stack implementation is more efficient.

