%%-*-latex-*-

\documentclass[11pt,a4paper]{article}

/home/rinderkn/LaTeX/TeX/trace.tex

\usepackage{xspace}
\newcommand\Erlang{\textsf{Erlang}\xspace}

\title{Homework \#11 of Erlang}
\author{Christian Rinderknecht}
\date{18 April 2007} 

\begin{document}

\maketitle

\thispagestyle{empty}

\begin{enumerate}

  \item Write a function \texttt{drop/2} such that \texttt{drop(L,N)}
    is the list \texttt{L} without every \texttt{N}th elements (first
    element is 1). For example
{\small
\begin{verbatim}
1> homework:drop([1,2,3,4,5,6,7,8,9],3).
[1,2,4,5,7,8]
\end{verbatim}
}

  \item Write a function \texttt{rnd\_select/2} such that
    \texttt{rnd\_select(L,N)} is a list of \texttt{N} items selected
    pseudo-randomly from the list \texttt{L}. For instance
{\small
\begin{verbatim}
1> homework:rnd_select([2,[],4,[[5],6],1,-2,[3,[]]],3).
[-2,[[5],6],2]
2> homework:rnd_select([2,[],4,[[5],6],1,-2,[3,[]]],3).
[[],[[5],6],[3,[]]]
3> homework:rnd_select([2,[],4,[[5],6],1,-2,[3,[]]],3).
[[[5],6],[3,[]],1]
4>
\end{verbatim}
}
\textbf{Hint:} Use \texttt{random:uniform/1}, since
\texttt{random:uniform(P)} is an integer chosen pseudo-randomly by the
\Erlang runtime between \texttt{1} and \texttt{P}. (The distribution
is uniform, i.e., all numbers in the interval have an equal chance to
be chosen.)
\end{enumerate}

Try to make your functions as efficient as possible.

\end{document}



