%%-*-latex-*-

%---------------------------------------------------------------------
%
\begin{frame}
\frametitle{Introduction}

\Erlang was developed at Ericsson. Since 1998, it is distributed
independently with an open source license.

\bigskip

The main web site distributing \Erlang is
\begin{center}
\url{http://www.erlang.org}
\end{center}
The implementation is called \textbf{Erlang/OTP} (Open
Telecommunication Platform).

\end{frame}

% ------------------------------------------------------------------------
%
\begin{frame}
\frametitle{Introduction (cont)}

\textbf{Sequential programming} in \Erlang consists in not using the
concurrence features. This allows to focus on the kernel of \Erlang,
which consists in basic functional features.

\bigskip

\Erlang is \textbf{dynamically typed}, that is, there is no need for
the programmer to annotate the code with type informations. 

\bigskip

The difference with scripting languages, like \Bash, \Perl, \Python or
\Ruby, is that in these languages, in case of a type error in an
operation, there is always a default value which can be implicitly the
result.

\bigskip

In \Erlang, a type error at run-time ends the execution of the
programs, if not handled.

\end{frame}

% ------------------------------------------------------------------------
%
\begin{frame}[containsverbatim]
\frametitle{Introduction/Modules}

Following a top-down approach, an \Erlang program is first
structured in \textbf{modules}.

\bigskip

At first glance, a module is a file containing \Erlang functions. 

\bigskip

The name of the module is given in the file by a special directive
(i.e., it is not part of the language itself, strictly speaking) and
must be the same as the file name on the underlying operating system.

\bigskip

For example, a file \texttt{my\_prog.erl} must include, as a first
statement, the directive
\begin{verbatim}
-module(my_prog).
\end{verbatim}
Note the starting dash and the final period.

\end{frame}

% ------------------------------------------------------------------------
%
\begin{frame}
\frametitle{Introduction/Shell and virtual machine}

Since \Erlang modules only contain functions, they serve as a
\emph{database of functions} and something else is needed to use them,
just like in a relational database one needs a query language.

\bigskip

This role is devoted to the \textbf{\Erlang shell}, similar to the
command line shells on \Unix and \Windows operating systems, like,
respectively, \Bash and \DOS.

\bigskip

From this shell, the user can open modules and call functions exported
by them. 

\end{frame}

% ------------------------------------------------------------------------
%
\begin{frame}
\frametitle{Introduction/Shell and virtual machine (cont)}

In order to run the functions, these must be compiled first
to \textbf{byte-code}, just like \Java for instance. 

\bigskip

Second, this byte-code is loaded into the \textbf{\Erlang virtual
  machine} (called \textbf{\BEAM}), a task which is hidden from the
user and taken care of by the \Erlang shell.

\end{frame}

% ------------------------------------------------------------------------
%
\begin{frame}
\frametitle{Introduction/First module}

Consider the file \texttt{math2.erl}:
\ErlangIn{math2}
This module exports a list of functions consisting of the lone
\texttt{double} function, which takes only one argument. This function
is defined in the body of the module, together with \texttt{mult},
which is not visible from the other modules or the \Erlang shell. 

\bigskip

Note that the order of definition is not meaningful.

\end{frame}

% ------------------------------------------------------------------------
%
\begin{frame}[containsverbatim]
\frametitle{Introduction/First compilation}

There are two ways of compiling \Erlang modules: from the OS shell or
from the \Erlang shell.


From the OS shell (whose prompt is here represented by the symbol
\texttt{\$}), type
{\scriptsize
\begin{verbatim}
$ erl
Erlang (BEAM) emulator version 5.5.2 [source] [async-threads:0] [hipe]

Eshell V5.5.2  (abort with ^G)
1> _
\end{verbatim}
}
The \Erlang shell prompt is here ``\texttt{1>}'', where ``\texttt{1}''
is the command line number (i.e., the first command is expected).

\end{frame}

% ------------------------------------------------------------------------
%
\begin{frame}[containsverbatim]
\frametitle{Introduction/From the Erlang shell}
\label{arity}

Then, the compilation function \texttt{c/1} takes the name of the
file to compile, e.g. \texttt{math2.erl}, launches the compiler and,
if no error is found, the resulting byte-code is output in a file
\texttt{math2.beam}:
\begin{verbatim}
1> c(math2).
{ok,math2}
2> _
\end{verbatim}
This means that the compilation was successful (\texttt{ok}) and the
result is the file \texttt{math2.beam}. The \texttt{/1} in
\texttt{c/1} is the number of arguments (called \textbf{arity}).

\end{frame}

% ------------------------------------------------------------------------
%
\begin{frame}[containsverbatim]
\frametitle{Introduction/From the OS shell}

The other way is just to call the \Erlang compiler from the OS
command-line shell:
\begin{verbatim}
$ erlc math2.erl 
$ _
\end{verbatim}

\end{frame}

% ------------------------------------------------------------------------
%
\begin{frame}[containsverbatim]
\frametitle{Introduction/Running the program}

After the module has been compiled to byte-code, we need an \Erlang
shell to use it.

\bigskip

From the \Erlang shell, a function is called by giving the name of the
module first, then typing in a colon, the function name and finally
its arguments, between parentheses and separated by commas.

\bigskip

For instance
\begin{verbatim}
1> math2:double(5).
10
2> _
\end{verbatim}
If there are no arguments, just type the opening and closing
parentheses.

\end{frame}

% ------------------------------------------------------------------------
%
\begin{frame}
\frametitle{Introduction/A classic}

\label{fact}

\begin{columns}
  \column{0.5\textwidth} By default, all functions can be
  recursive. Take a look at the classic factorial mathematical
  function in \Erlang:
  \ErlangIn{math1}

  \column{0.5\textwidth} Notice how a function can be defined in a
  manner similar to the mathematical notation:
  \begin{align*}
    0! &= 1\\
    n! &= n \times (n-1)!
  \end{align*}
  Note that the function names in \Erlang (e.g. \texttt{fact}) must
  start in lowercase and the variables (e.g. \texttt{N}) in uppercase.

\end{columns}

\end{frame}

% ------------------------------------------------------------------------
%
\begin{frame}[containsverbatim]
\frametitle{Introduction/A classic (cont)}

The factorial function is defined by two cases, called
\textbf{clauses} in \Erlang. Clauses of the same function are
separated by semicolons and the last one is terminated by a period:
\begin{verbatim}
...
fact(0) -> ... ; % First clause
fact(N) -> ... . % Second clause
\end{verbatim}
The part before the arrow is called the \textbf{head} and the part
after it until the semicolon or the period is called the \textbf{body}
of the function:
\[
\underbrace{\texttt{fact(N)}}_{\textrm{\normalsize head}} \;
\texttt{->} \; \underbrace{\texttt{N *
    fact(N-1)}}_{\textrm{\normalsize body}}\texttt{.}
\]

\end{frame}

% ------------------------------------------------------------------------
%
\begin{frame}
\frametitle{Introduction/Clause selection}

The order of the clauses matters. 

\bigskip

Given a function call and the clauses of the corresponding function,
the \Erlang run-time
\begin{enumerate}

  \item computes the arguments in the call;

  \item examines the heads of the clauses, in the order of the
    module, until the head of a clause \textbf{matches} the function
    call;

  \item that matching may bind variables to values in the call (in
    other words: ``the parameters are bound to the arguments'') and
    these variables are replaced by their value in the body of the
    selected clause;

  \item the body is evaluated.

\end{enumerate}

\end{frame}

% ------------------------------------------------------------------------
%
\begin{frame}[containsverbatim]
\frametitle{Introduction/Clause selection (cont)}

For example, consider the function call \texttt{math1:fact(0)}.

\bigskip

We do not need here to evaluate the arguments, since \texttt{0} is
already a value.

\bigskip

The first clause in module \texttt{math1} is
\begin{verbatim}
fact(0) -> 1
\end{verbatim}
Its head is exactly the function call: it is a trivial case of
matching. There are no variable in the head, so there is no binding as
the result of the matching. 

\bigskip

Hence we do not need to replace any variable in the body. Finally, the
body, \texttt{1}, is the result --- there is no need to evaluate it,
since it is already a value.

\end{frame}

% ------------------------------------------------------------------------
%
\begin{frame}
\frametitle{Introduction/Clause selection (cont)}

Consider now the function call \texttt{math1:fact(1)}.
\begin{center}
\textbf{Matching rule \#1}: \emph{Two values match if and only if they
  are equal.}
\end{center}
The head of the first clause is \texttt{fact(0)} and \texttt{0} does
not match \texttt{1}, so the first clause is skipped.

\bigskip

The head of the second clause is \texttt{fact(N)}. 
\begin{center}
\textbf{Matching rule \#2}: \emph{A variable matches any value.}
\end{center}
Therefore \texttt{N} matches \texttt{1} and the body of the second
clause, \texttt{N * fact(N-1)}, is selected. The matching leads to
bind \texttt{N} to \texttt{1}.

\bigskip

Finally, \texttt{N} is replaced by \texttt{1} in the body, which
becomes \texttt{1 * fact(1-1)}, and is further evaluated as the
result (i.e., the value of the function call).

\end{frame}

% ------------------------------------------------------------------------
%
\begin{frame}
\frametitle{Introduction/Tracing computations}

It is possible to summarise the evaluation of a function call by a
series of rewrite steps. For instance
\begin{align*}
\texttt{fact(2)}
  & \stackrel{2}{\longrightarrow} \texttt{2 * fact(2-1)}\\
  & \longrightarrow \texttt{2 * fact(1)}\\
  & \stackrel{2}{\longrightarrow} \texttt{2 * (1 * fact(1-1))}\\
  & \longrightarrow \texttt{2 * (1 * fact(0))}\\
  & \stackrel{1}{\longrightarrow} \texttt{2 * (1 * (1))}\\
  & \longrightarrow \texttt{2 * (1)}\\
  & \longrightarrow \texttt{2}
\end{align*}
Note: The number on the arrow is the clause number. An arrow with no
number is an elementary arithmetic computation.

\end{frame}

% ------------------------------------------------------------------------
%
\begin{frame}
\frametitle{Introduction/Why order matters}

Let us try the following mistake:
\ErlangInUnchecked{math1bis}

\end{frame}

% ------------------------------------------------------------------------
%
\begin{frame}
\frametitle{Introduction/Why order matters (cont)}

Then let us evaluate
\begin{align*}
\texttt{fact(2)}
  & \stackrel{1}{\longrightarrow} \texttt{2 * fact(2-1)}\\
  & \longrightarrow \texttt{2 * fact(1)}\\
  & \stackrel{1}{\longrightarrow} \texttt{2 * (1 * fact(1-1))}\\
  & \longrightarrow \texttt{2 * (1 * fact(0))}\\
  & \stackrel{1}{\longrightarrow} \texttt{2 * (1 * (0 * fact(0-1)))}\\
  & \longrightarrow \texttt{2 * (1 * (0 * fact(-1)))}\\
  & \stackrel{1}{\longrightarrow} \texttt{2 * (1 * (0 * (-1 *
  fact(-1-1))))}\\
  & \longrightarrow \ldots
\end{align*}
This results in an infinite loop because the head of the second clause
can never match ---~and thus terminate the computation.

\end{frame}

% ------------------------------------------------------------------------
%
\begin{frame}[containsverbatim]
\frametitle{Introduction/Classic mistakes}

Let us try to call an undefined function:
{\small
\begin{verbatim}
1> math:double(5).
** exited: {undef,[{math,double,[5]},
                   {erl_eval,do_apply,5},
                   {shell,exprs,6},
                   {shell,eval_loop,3}]} **

=ERROR REPORT==== 12-Jan-2007::01:46:49 ===
Error in process <0.29.0> with exit value:
{undef,[{math,double,[5]},
{erl_eval,do_apply,5},{shell,exprs,6},{shell,eval_loop,3}]}
\end{verbatim}
}

Similar problem with the call \texttt{math2:triple(5)} and
\texttt{math2:double()} and \texttt{math2:mult(5,5)}.

\end{frame}

% ------------------------------------------------------------------------
%
\begin{frame}[containsverbatim]
\frametitle{Introduction/Function composition and stop}

As in mathematics, function calls can be composed (i.e., nested), as in
\begin{verbatim}
1> math2:double(math2:double(5)).
20
2> _
\end{verbatim}
To stop the \Erlang shell, type
\begin{verbatim}
2> init:stop().
ok
3> $ _
\end{verbatim}

\end{frame}
