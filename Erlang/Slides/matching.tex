%% -*- latex -*-

% ------------------------------------------------------------------------
%
\begin{frame}
\frametitle{Comparing expressions}

\Erlang offers two different mechanisms to compare expressions:
\textbf{matching} and numeric comparisons. The former is found in
functional and logic programming, but not in mainstream programming
languages. The latter is ubiquitous.

\bigskip

Numeric comparisons involve a comparison operator and two
operands. For instance, consider the following equality in both \Java
and \Erlang:
\begin{center}
\tt
fact(5) == 119 + 1
\end{center}
Note that
\begin{enumerate}

  \item The two sides are expressions

  \item which are separately evaluated, 

  \item then the values are compared and the result is either true or
    false.

\end{enumerate}

\end{frame}

% ------------------------------------------------------------------------
%
\begin{frame}[containsverbatim]
\frametitle{Java variable definitions}

Consider a \Java variable definition like
\begin{verbatim}
int N = 3;
\end{verbatim}
This actually means three different things:
\begin{enumerate}

  \item define the variable \texttt{N}

  \item whose type is \texttt{int}

  \item and initial value is \texttt{3}.

\end{enumerate}
Initialisation is a separate concept since the above is equivalent to
\begin{verbatim}
int N;
N = 3;
\end{verbatim}

\end{frame}

% ------------------------------------------------------------------------
%
\begin{frame}[containsverbatim]
\frametitle{Matching in Erlang}

In \Erlang, the programmer does not specify the types, as the compiler
does not check most of them, so the above is simply written in \Erlang
\begin{verbatim}
N = 3
\end{verbatim}
\Erlang provides an extension of the \Java-like variable definition:
\emph{matching}.
\begin{enumerate}

  \item The left-hand side must be a \textbf{term} (not an
    expression),

  \item the right-hand side is an expression

  \item which is evaluated before the matching;

  \item the matching is itself an expression whose
    value (in case of success) is the value of the right-hand side (as
    in \cpp{} ``\texttt{a = (b = 1);}'').

\end{enumerate}

\end{frame}

% ------------------------------------------------------------------------
%
\begin{frame}
\frametitle{Matching (cont)}

What if \texttt{N} does not occur for the first time in the matching?
Consider
\ErlangIn{match}
In this case, \texttt{N} in the matching is replaced first by its
value, which is the argument of the function \texttt{f} when it is
called. 

\bigskip

So \texttt{match:f(3)} succeeds but, for example, \texttt{match:f(2)}
fails.

\end{frame}

% ------------------------------------------------------------------------
%
\begin{frame}
\frametitle{Matching (cont)}

Therefore, the rule for evaluating the matching 
\[
t \, \texttt{=} \, e
\]
is actually better expressed as follows:
\begin{enumerate}

  \item replace all occurrences of variables in the term \(t\) which
    are already bound by their value: the new term is \(t'\);

  \item evaluate expression \(e\) into \(v\);

  \item match \(v\) against \(t'\);

  \item in case of success, the value of the matching is \(v\).

\end{enumerate}
The left-hand side of a matching (here noted \(t\)) is called a
\textbf{pattern}.

\end{frame}

% ------------------------------------------------------------------------
%
\begin{frame}[containsverbatim]
\frametitle{Matching (cont)}

Assume that \texttt{N} appears for the first time in the matching
\begin{verbatim}
N = 3
\end{verbatim}
Then, it is a simple case of matching since
\begin{enumerate}

  \item the left-hand side, \texttt{N}, being a variable, is a term;

  \item the right-hand side, \texttt{3}, is an expression (actually a
    value);

  \item the evaluation of the right-hand side is itself, \texttt{3},
    since it is a value already;

  \item the matching of numbers is the same as the numeric equality;

  \item the value of the matching is \texttt{3}.

\end{enumerate}
Importantly, the matching leads to bind variable \texttt{N} to
\texttt{3}, noted \texttt{N} \(\mapsto\) \texttt{3}.

\end{frame}

% ------------------------------------------------------------------------
%
\begin{frame}[containsverbatim]
\frametitle{Matching (cont)}

What is the difference, in \Erlang, between
\verb|N = 3| and \verb|N == 3|?

\bigskip

The first case is a matching and the second is a numeric
comparison. 

\bigskip

In the latter, \texttt{N} must already be defined (bound) so it
can be evaluated before the comparison takes place. The result if
either the atom \texttt{true} or \texttt{false}.

\bigskip

In the former, if \texttt{N} is already bound, the matching either
fails or succeeds and its value is \texttt{3}. 

\bigskip

If \texttt{N} is not already bound, then the matching succeeds with
the value \texttt{3} and the binding of \texttt{N} to \texttt{3},
noted \texttt{N} \(\mapsto\) \texttt{3}.

\end{frame}

% ------------------------------------------------------------------------
%
\begin{frame}
\frametitle{Matching (cont)}

You may wonder now why bother? What is the need of matching when we
have equality and other comparisons?

\bigskip

Actually the fist difference is the possible
\textbf{failure}. Consider
\begin{columns}
  \column{0.5\textwidth}
    \ErlangIn{match}
  \column{0.5\textwidth}
    \ErlangIn{eq}
\end{columns}

\bigskip

The calls to \texttt{match:f} either fail, therefore stop the whole
program, or return the integer \texttt{3}. The calls to
\texttt{eq:f} return either \texttt{true} or \texttt{false}.

\end{frame}

% ------------------------------------------------------------------------
%
\begin{frame}[containsverbatim]
\frametitle{Matching (cont)}

The main difference between matching and equality appears when
matching complex terms, involving lists and tuples. We will explain
this from page~\pageref{tuple_matching} on.

\bigskip

Remember also that if \texttt{N} is not bound, then 
\begin{verbatim}
N = 3
\end{verbatim}
acts as a variable definition.

\bigskip

This way, we can think of the matching as a way to combine both
definition and comparison.

\end{frame}

% ------------------------------------------------------------------------
%
\begin{frame}[containsverbatim]
\frametitle{Matching/Basic examples}

Here is another example:
\begin{verbatim}
N = 3 + 1
\end{verbatim}
Here, the right-hand side is not a term, it is evaluated into
\texttt{4} before the matching takes place.

\bigskip

Another one:
\begin{verbatim}
N = 2 * f(3)
\end{verbatim}
Here, the function \texttt{f} must be in the scope, i.e. must be
bound at this point.

\end{frame}

% ------------------------------------------------------------------------
%
\begin{frame}[containsverbatim]
\frametitle{Scope and multiple clauses}

\label{scope}

How are variables defined (i.e. bound) by matching used latter? In
other words: what is the scope? In \cpp, the scope is the definition
point until the end of the current bloc. For example 
\begin{verbatim}
...
{ 
  ...
  int n = 3;
  ...  // `n' is bound here.
}
... // `n' is unbound here.
\end{verbatim}
In \Erlang, function bodies can be made of series of expressions
separated by commas. For example \verb|f(X) -> X=3, X+1.|

\end{frame}

% ------------------------------------------------------------------------
%
\begin{frame}[containsverbatim]
\frametitle{Scope and multiple clauses (cont)}

The expressions are evaluated from left to right and, as exemplified
in
\begin{verbatim}
f(X) -> X=3, X+1.
\end{verbatim}
the bindings created by matches (a match is a successful matching) are
accessible for the following expressions. This is why \texttt{X} is
bound to the value \texttt{3} in the expression \texttt{X + 1}.

\bigskip

Note that only the value of the last expression becomes the return
value of the function. Thus, for example, the value of \texttt{X=3},
which is \texttt{3}, is discarded, \textbf{but not the binding of
  \texttt{X}!} Remember that evaluating a match yields to
\begin{enumerate}

  \item a value

  \item \emph{and}, if variables are present on the left-hand side,
    bindings for these variables.

\end{enumerate}

\end{frame}

% ------------------------------------------------------------------------
%
\begin{frame}
\frametitle{Matching/Constants}

Two numbers match if they represent the same mathematical number, e.g.
\begin{center}
\tt
37 = 37
\end{center}
Two atoms match if they are made of the same characters, e.g.
\begin{center}
\tt
abc = abc\\
'hello world' = 'hello world'
\end{center}
A variable matches any term that can be evaluated, e.g.
\begin{center}
\tt
X = 37\\
X = Y
\end{center}

\end{frame}

% ------------------------------------------------------------------------
%
\begin{frame}
\frametitle{Matching/Tuples}

\label{tuple_matching}

A tuple \(t\) matches an expression \(e\), that is
\[
t \, \texttt{=} \, e
\]
if and only if
\begin{enumerate}

  \item the replacement of the variables of \(t\) that are bound
    yields the (term) tuple \(t'\);

  \item the evaluation of \(e\) yields the tuple \(v\);

  \item both \(t'\) and \(v\) have the same arity;

  \item the \(n\)-th component of \(t\) matches the \(n\)-th component
    of \(v\).
  
\end{enumerate}
In case of success, the resulting bindings of all sub-matches are
joined.

\end{frame}

% ------------------------------------------------------------------------
%
\begin{frame}[containsverbatim]
\frametitle{Matching/Tuples (cont)}

The matching of tuples is very useful, because it allows to
destructure the return values of functions in a single expression.

\bigskip

For example
\begin{verbatim}
go(Job) ->
  {Status,NewJob,Priority} = launch(Job),
  forward(Message),
  schedule(NewJob,Priority).
\end{verbatim}
The tuples can be arbitrarily nested:
\begin{verbatim}
go(Job) ->
  {{status,{Answer,Log}},Version} = 
    {run(Job),version(Job)}.
\end{verbatim}

\end{frame}

% ------------------------------------------------------------------------
%
\begin{frame}
\frametitle{Matching/Lists}

A list \(l\) matches an expression \(e\), that is
\[
l \, \texttt{=} \, e
\]
if and only if
\begin{enumerate}

  \item the replacement of the variables in \(l\) which are bound
    yields the (term) list \(l'\);

  \item the evaluation of \(e\) yields the list \(v\);

  \item both \(e\) and \(v\) are the empty list \texttt{[]} or

  \item the head of \(l'\) matches the head of \(v\) and

  \item the tail of \(l'\) matches the tail of \(v\).

\end{enumerate}

\end{frame}

% ------------------------------------------------------------------------
%
\begin{frame}[containsverbatim]
\frametitle{Matching/Lists/Examples}

Consider for instance
\begin{verbatim}
[Head | Tail] = [1,a,2,b,c]
\end{verbatim}
This match yields the value \texttt{[1,a,2,b,c]} because of the matches
\begin{verbatim}
Head = 1
Tail = [a,2,b,c]
\end{verbatim}
The bindings are thus \texttt{Head} \(\mapsto\) \texttt{1} and
\texttt{Tail} \(\mapsto\) \texttt{[a,2,b,c]}.

Consider also the two matchings
\begin{verbatim}
{X,Y} = {a,b}, [A,X|B] = [Y,a]
\end{verbatim}
which succeed, yielding the bindings \texttt{X} \(\mapsto\)
\texttt{a}, \texttt{Y} \(\mapsto\) \texttt{b}, \texttt{A}
\(\mapsto\) \texttt{b} and \texttt{B} \(\mapsto\) \texttt{[]}.

\end{frame}

% ------------------------------------------------------------------------
%
\begin{frame}[containsverbatim]
\frametitle{Matching/Mixed examples}

The matching
\begin{verbatim}
{C, [Head | Tail]} = {{222, man}, [a,b,c]}
\end{verbatim}
succeeds with the bindings \{\texttt{C} \(\mapsto\) \texttt{\{222,
  man\}}, \texttt{Head} \(\mapsto\) \texttt{a}, \texttt{Tail}
\(\mapsto\) \texttt{[b,c]}\}.

The matching
\begin{verbatim}
[{person,Name,Age,_}|T] = [{person,fred,22,male}]
\end{verbatim}
succeeds with bindings \{\texttt{Name} \(\mapsto\) \texttt{fred},
\texttt{Age} \(\mapsto\) \texttt{22}, \texttt{T} \(\mapsto\)
\texttt{[]}\}. Note the pattern ``\texttt{\huge \_}'' which stands for
an unknown but unique variable.

\end{frame}

% ------------------------------------------------------------------------
%
\begin{frame}[containsverbatim]
\frametitle{Matching/Non-linear patterns}

Remember that a matching consists in comparing structurally the value
of an expression to a term, called a pattern.

\bigskip

It is possible in \Erlang to repeat a variable in a pattern, in order
to save another match later. For example
\begin{verbatim}
{X,X} = {a,f()}
\end{verbatim}
is equivalent to
\begin{verbatim}
{X,Y} = {a,f()}, X = Y
\end{verbatim}
The matching
\begin{verbatim}
{A, foo, A} = {123, foo, abc}
\end{verbatim}
fails.

\end{frame}
 
% ------------------------------------------------------------------------
%
\begin{frame}
\frametitle{Matching/Calling functions}

The \Erlang functions are called using pattern matching too. At
page~\pageref{fact} we defined the factorial function as
\begin{columns}
  \column{0.5\textwidth} \ErlangIn{math1} There are two cases. When
  evaluating a call, the \Erlang virtual machine (called \BEAM) needs
  to determine which case has to be chosen.

  \column{0.5\textwidth} This is decided by matching the call against
  the heads of the cases in turn, here \texttt{fact(0)} first then
  \texttt{fact(N)}, considered as patterns --- assuming that a
  function name only matches the same name.

  \medskip

  The first match selects the function body to be executed.
\end{columns}

\end{frame}

% ------------------------------------------------------------------------
%
\begin{frame}
\frametitle{Matching/Calling functions (cont)}

The procedure can be summarised as follows:
\begin{enumerate}

  \item evaluate the arguments of the function call;

  \item find the cases defining the called function;

  \item try to match the head of the first case against the function
    call;

  \item in case of success, the body of the case is evaluated with the
    bindings of the match;

  \item otherwise the next case is tried;

  \item if no case matches, an error is issued.

\end{enumerate}

\end{frame}

% ------------------------------------------------------------------------
%
\begin{frame}[containsverbatim]
\frametitle{Matching/Calling functions (cont)}

Consider a more complex case:
\begin{verbatim}
X=2, fact(X)
\end{verbatim}
\begin{enumerate}

  \item the argument \texttt{X} of \texttt{fact(X)} is evaluated with
    the bindings \{\texttt{X} \(\mapsto\) \texttt{2}\}, so the call is
    now \texttt{fact(2)};

  \item the matching \texttt{fact(0) = fact(2)} fails, 

  \item so \texttt{fact(N) = fact(2)} is tried, and succeeds;

  \item the expression \texttt{N * fact(N-1)} is evaluated with the
    bindings \{\texttt{N} \(\mapsto\) \texttt{2}\}.

\end{enumerate}

\end{frame}

% ------------------------------------------------------------------------
%
\begin{frame}
\frametitle{Matching/Calling functions (cont)}

Consider a function definition relying on the matching of a list:
\ErlangIn{list1}
This function takes a list as argument and return the number of items
in it.

\bigskip

There are two cases: one for the empty list and another for the
non-empty list.

\end{frame}
