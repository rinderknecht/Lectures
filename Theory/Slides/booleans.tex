%%-*-latex-*-

% ------------------------------------------------------------------------
%
\begin{frame}[containsverbatim]
\frametitle{Order of evaluation in \C}

In case of \C, the order of evaluation arguments of arithmetic
operators and functions is not specified, it actually depends on the
compiler and its version. Therefore it would be dangerous to rely on
the order of evaluation in \C. 

\bigskip

Why is the order left unspecified? Simply because of possible
\textbf{side-effects}. Imagine the following situation:
{\small
\begin{verbatim}
int n = 0;
int f () { n++; return n; };
int g () { n = 2; return n; };
int main () { return f() + g() + n; }
\end{verbatim}
}
What is the result?

\end{frame}

% ------------------------------------------------------------------------
%
\begin{frame}
\frametitle{Boolean expressions}

Let \(\wedge\) represent the conjunction (``and''), \(\vee\) the
disjunction (``or''), \(\neg\) the negation (``not''), and let the two
values \true and \false.

\bigskip

The definition of the conjunction of values is very simple:
\begin{mathpar}
\inferrule
{}
{\true \wedge \true \rightarrow \true}
\and
\inferrule
{}
{\true \wedge \false \rightarrow \false}\\
\inferrule
{}
{\false \wedge \true \rightarrow \false}
\and
\inferrule
{}
{\false \wedge \false \rightarrow \false}
\end{mathpar}

\end{frame}


% ------------------------------------------------------------------------
%
\begin{frame}
\frametitle{Boolean expressions made simpler}

This system can even be made simpler, by means of meta-variables:
\begin{mathpar}
\inferrule
{}
{\true \wedge \true \rightarrow \true}
\quad\rname{\TirName{True}}
\\
\inferrule
{}
{\false \wedge v \rightarrow \false}
\quad\rname{\TirName{False}_1}
\\
\inferrule
{}
{v \wedge \false \rightarrow \false}
\quad\rname{\TirName{False}_2}
\end{mathpar}
where \(v\) denotes any value, i.e., either \true or \false.

\end{frame}

% ------------------------------------------------------------------------
%
\begin{frame}
\frametitle{Boolean expressions (cont)}

So, if we define the rewrite of boolean expressions (not just the
values \true and \false) as
\begin{mathpar}
\inferrule
{e_1 \rightarrow e'_1}
{e_1 \wedge\, e_2 \rightarrow e'_1 \wedge\, e_2}
\;\rname{\TirName{\(\wedge_1\)}}
\and
\inferrule
{e_2 \rightarrow e'_2}
{e_1 \wedge\, e_2 \rightarrow e_1 \wedge\, e'_2}
\;\rname{\TirName{\(\wedge_2\)}}
\end{mathpar}
then, for example,
\begin{mathpar}
\inferrule
{\false \wedge\; \true \rightarrow \false \quad \rname{\TirName{False}_1}}
{(\false \wedge\; \true) \wedge\, (\true \wedge\, \true) \rightarrow
  \false \wedge\, (\true \wedge\, \true)}
\;\sigma_1\rname{\wedge_1}
\end{mathpar}
where \(\sigma_1=\{e_1 \mapsto \false \wedge \true; e_2 \mapsto \true
\wedge \true; e'_1 \mapsto \false\}\).

\end{frame}

% ------------------------------------------------------------------------
%
\begin{frame}
\frametitle{Boolean expressions (cont)}

Then
\begin{mathpar}
\inferrule
{\true \wedge\, \true \rightarrow \true \quad \rname{\TirName{True}}}
{\false \wedge\, (\true \wedge\, \true) \rightarrow \false \wedge\,
  \true}
\;\sigma_2\rname{\wedge_2}
\end{mathpar}
where \(\sigma_2 = \{e_1 \mapsto \false; e_2 \mapsto \true \wedge
\true; e'_1 \mapsto \true\}\).

\bigskip

and, finally, we can instantiate \(\rname{\TirName{False}_1}\) with
\(\sigma_3 = \{v \mapsto \true\}\):
\begin{mathpar}
\inferrule
{}
{\false \wedge\, \true \rightarrow \false}
\quad\sigma_3\rname{\TirName{False}_1}
\end{mathpar}
In this example, we computed both arguments of \(\wedge\) to reach the
value.

\end{frame}

% ------------------------------------------------------------------------
%
\begin{frame}
\frametitle{Boolean expressions (cont)}

But the inspection of the basic rules shows that if one of the
arguments of \(\wedge\) reduces to \false, then there is no need to
reduce the other argument: the rewrite will always be
\false. Therefore, it is more efficient to extend the rules on values
to the expressions:
\begin{mathpar}
\inferrule
{}
{\false \wedge e \rightarrow \false}
\quad\rname{\TirName{False}_1}
\and
\inferrule
{}
{e \wedge \false \rightarrow \false}
\quad\rname{\TirName{False}_2}\\
\inferrule
{}
{\true \wedge e \rightarrow e}
\quad\rname{\TirName{True}_1}
\and
\inferrule
{}
{e \wedge \true \rightarrow e}
\quad\rname{\TirName{True}_2}
\end{mathpar}
where \(e\) represents any boolean expression. Then, now
\begin{mathpar}
\inferrule
{}
{\false \wedge\, (\true \wedge\, \true) \rightarrow \false}
\quad\sigma_3\rname{\TirName{False}_1}
\end{mathpar}
where \(\sigma_3 = \{e \mapsto \true \wedge \true\}\). 

\end{frame}

% ------------------------------------------------------------------------
%
\begin{frame}
\frametitle{Boolean expressions (cont)}

Try to reduce
\[(
\true \wedge (\false \wedge \true)) \wedge (\true \wedge (\true \wedge
\true))
\]
Note that this rewrite system is not deterministic. If we enforce that
the left argument of \(\wedge\) is completely reduced first, then we
can check if it is \true or \false: in the latter case, we do not need
to reduce the right argument.

\end{frame}

% ------------------------------------------------------------------------
%
\begin{frame}[containsverbatim]
\frametitle{Boolean expressions (cont)}

This is easy to specify:
\begin{mathpar}
\inferrule
{}
{\true \wedge e \rightarrow e}
\quad\rname{\wedge_{\TirName{True}}}
\and
\inferrule
{}
{\false \wedge e \rightarrow \false}
\quad\rname{\wedge_{\TirName{False}}}\\
\inferrule
{e_1 \rightarrow e'_1}
{e_1 \wedge\, e_2 \rightarrow e'_1 \wedge\, e_2}
\;\rname{\wedge}
\end{mathpar}
Many programming languages, as \C, use this strategy. It allows the
programmers to write handy code like
{\small 
\begin{verbatim}
while (index <= max && tab[index] > 0) ...
\end{verbatim}
}

\end{frame}

% ------------------------------------------------------------------------
%
\begin{frame}
\frametitle{Boolean expressions (cont)}

Note that this strategy (left argument before the right one) is not
optimal in general: the first argument may reduce to
\true. \emph{Finding an optimal strategy is not always possible.} For
example, we could add another rule that shortens sometimes the
reductions:
\begin{mathpar}
\inferrule
{}
{e \wedge e \rightarrow e}
\;\rname{\wedge_{\TirName{Refl}}}
\end{mathpar}
This rule explicitly specifies the \textbf{reflexivity} of the
conjunction. 

But the system becomes non-deterministic because there is now an
overlap between rule \(\rname{\wedge_{\TirName{Refl}}}\) and both
\(\rname{\wedge_{\proc{True}}}\) and \(\rname{\wedge_{\proc{False}}}\)
(consider for example \(\true \wedge \false\)).

\end{frame}

% ------------------------------------------------------------------------
%
\begin{frame}
\frametitle{Determinism versus non-determinism}

Let us summarise the concepts of determinism and non-determinism.
\begin{itemize}

  \item The determinism means that, for any expression, there is only
    one rule to rewrite it. This implies that, every time a program
    is run with the same input, the output will the same. This is a
    useful property, called soundness, in particular for debugging.

  \item Non-determinism means that, for any expression, there may be
    several rules to rewrite it. Moreover,
    \begin{itemize}

      \item if the rewrite system is sound, the output will always be
        the same for any run on the same input, even if the trace may
        be different each time, which makes the debugging more
        difficult.
 
      \item the rewrite system may be unsound.

   \end{itemize}
\end{itemize}
\end{frame}

% ------------------------------------------------------------------------
%
\begin{frame}[containsverbatim]
\frametitle{Determinism versus non-determinism (cont)}

\noindent Some programming languages have features that may be both
non-deterministic and unsound.

\bigskip

Perhaps the most infamous case is in \C, where variables do not need
to be initialised at their declaration: {\small
\begin{verbatim}
int a;
\end{verbatim}
}
If the programmer forgets to give a value to the variable, any access
to it is unspecified. 

\bigskip

Practically, this means that whatever is located in the memory at the
location of the variable will be used at each read access. Of course,
this value may change from one run of the program to another, even
on the same input. 

\bigskip

Maybe not.

\end{frame}

% ------------------------------------------------------------------------
%
\begin{frame}
\frametitle{Commutativity and its consequences}

We may try another rewrite system for \(\wedge\):
\begin{mathpar}
\inferrule
{\true \wedge e \rightarrow e}
{}
\quad\rname{\TirName{True}}
\and
\inferrule
{}
{\false \wedge e \rightarrow e}
\quad\rname{\TirName{False}}\\
\inferrule
{e_1 \rightarrow e'_1}
{e_1 \wedge e_2 \rightarrow e'_1 \wedge e_2}
\;\rname{\wedge}
\and
\inferrule
{}
{e_1 \wedge e_2 \rightarrow e_2 \wedge e_1}
\;\rname{\TirName{Comm}}
\end{mathpar}
While this system is in theory fine, there are two practical problems:
\begin{enumerate}

  \item it is \textbf{not deterministic} due to 
    \(\rname{\TirName{Comm}}\) that overlaps with all the other rules;

  \item it may \textbf{not terminate} due to rule
    \(\rname{\TirName{Comm}}\):
\begin{align*}
\true \wedge \false &\xrightarrow{\rname{\TirName{Comm}}} \false \wedge
\true \xrightarrow{\rname{\TirName{Comm}}} \true \wedge \false\\
&\xrightarrow{\rname{\TirName{Comm}}} \false \wedge \true \rightarrow
\cdots
\end{align*}

\end{enumerate}

\end{frame}

% ------------------------------------------------------------------------
%
\begin{frame}[containsverbatim]
\frametitle{Non-termination}

Now some reductions may not be finite, which models
\textbf{non-termination}. 

\bigskip

This particular kind of non-termination, due to commutativity, is the
model in programming languages of so-called ``\textbf{infinite
  loops}'' that do not allocate memory, i.e., that do not require
additional memory at each loop step. Equivalently, it corresponds to a
\emph{recursive call} like {\small
\begin{verbatim}
bool and (bool e1, bool e2) { ... return and (e2, e1); ... }
\end{verbatim}
}
Anyway, in presence of infinite reductions, the soundness property
must be weakened as follows:
\begin{quote}
\textbf{Weak soundness.} \emph{All finite reductions of an expression
  end with the same value, if any.}
\end{quote}

\end{frame}

% ------------------------------------------------------------------------
%
\begin{frame}
\frametitle{Non-termination (cont)}

Let us define the negation:
\begin{mathpar}
\inferrule
{}
{\neg \true \rightarrow \false}
\and
\inferrule
{}
{\neg \false \rightarrow \true}
\end{mathpar}
It is also well known that the negation satisfies \(e = \neg\neg
e\). So we could imagine another rewrite rule
\[
\neg\neg e \rightarrow e
\]
This is fine, this rule simplifies double negations. But what if we
had the converse rule?
\[
e \rightarrow \neg\neg e \qquad \text{where} \; e \; \text{is not a
  value.}
\]
This allows infinite reductions:
\(
e \rightarrow \neg\neg e \rightarrow e \rightarrow \neg\neg e
\rightarrow \dots
\)
% excluded middle: e or ~e
% non-contradiction: ~(e and ~e)
% de Morgan applied to non-contradiction: ~(e and ~e) = ~e or ~~e

\end{frame}

% ------------------------------------------------------------------------
%
\begin{frame}
\frametitle{Summary}

There are three kinds of reductions:
\begin{enumerate}

  \item Finite and ending in a value
  \[e_1 \rightarrow e_2 \rightarrow \dots \rightarrow v\]

  \item Finite but ending in a stuck expression
  \[e_1 \rightarrow e_2 \rightarrow \dots \rightarrow e_n
    \nrightarrow\]

  \item Infinite
  \[e_1 \rightarrow e_2 \rightarrow \cdots\]

\end{enumerate}

\end{frame}

% ------------------------------------------------------------------------
%
\begin{frame}
\frametitle{Conditioned expressions}

At this point, boolean expressions can be
\begin{itemize}

  \item \Xtrue (a value)

  \item \Xfalse (a value)

  \item \(e_1 \wedge e_2\)

  \item \(e_1 \vee e_2\)

  \item \(\neg e_1\)

\end{itemize}
where \(e_1\) and \(e_2\) are boolean expressions. Let us
extend our boolean expressions with a \textbf{conditioned expression}:
\begin{itemize}
  \item \Xif \(e_1\) \Xthen \(e_2\) \Xelse \(e_3\)
\end{itemize}

\end{frame}

% ------------------------------------------------------------------------
%
\begin{frame}
\frametitle{Conditioned expressions (cont)}

It is easy to extend the rewrite system to cope with conditioned:
\begin{mathpar}
\inferrule
{}
{\Xif \; \Xtrue \; \Xthen \; e_2 \; \Xelse \; e_3 \rightarrow e_2}
\quad\rname{\TirName{If-True}}
\and
\inferrule
{}
{\Xif \; \Xfalse \; \Xthen \; e_2 \; \Xelse \; e_3 \rightarrow e_3}
\quad\rname{\TirName{If-False}}
\and
\inferrule
{e_1 \rightarrow e'_1}
{\Xif \; e_1 \; \Xthen \; e_2 \; \Xelse \; e_3 \rightarrow \Xif \;
  e'_1 \; \Xthen \; e_2 \; \Xelse \; e_3}
\;\rname{\TirName{If}}
\end{mathpar}
This means that the \textbf{conditional expression} \(e_1\) must be
completely reduced \emph{before} the others. This is the usual way in
programming languages, the rationale being to avoid unnecessary
computations (evaluating \(e_2\) and \(e_3\) is not necessary).

\end{frame}
