\paragraph{R�ponses.}

\begin{enumerate}

  \item Un arbre binaire est un ensemble fini, �ventuellement vide, de
    n{\oe}uds li�s par une relation de parent� orient�e:
    \[
    \left\{
      \begin{array}{l}
        x \, {\cal F}_d \, y \Longleftrightarrow x \, \textnormal{est
          le fils droit de} \, y \Longrightarrow y \, \textnormal{est le
          p�re de} \, x \\
        x' \, {\cal F}_g \, y \Longleftrightarrow x' \,
        \textnormal{est le fils gauche de} \, y \Longrightarrow y \,
        \textnormal{est le p�re de} \, x'
      \end{array}
      \right.
    \]
    avec les propri�t�s:
    \begin{itemize}

     \item si l'arbre est non vide alors il existe un unique �l�ment
       n'ayant pas de p�re, appel� \emph{racine} de l'arbre;

     \item tout �l�ment � part la racine a un et un seul p�re;

     \item tout �l�ment a au plus un fils droit et au plus un fils
       gauche;
 
     \item tout �l�ment est un descendant de la racine.

   \end{itemize}

  \item 
    \begin{itemize}

      \item Le parcours en profondeur d'abord: partant de la racine,
        on explore d'abord compl�tement les sous-arbres;

      \item le parcours en largeur d'abord: partant de la racine, on
        explore tous les fils avant de passer au niveau suivant.

    \end{itemize}

  \item Un parcours pr�fixe passe par le p�re puis les sous-arbres des
    fils. Un parcours postfixe explore les sous-arbres des fils, puis
    le p�re. Un parcours infixe passe par un sous-arbre, puis le p�re,
    puis l'autre sous-arbre (l'arbre doit �tre binaire).

  \item 
    \begin{enumerate}

      \item Un arbre binaire r�duit � la racine poss�de deux
        sous-arbres vides. Supposons la propri�t� vraie au rang
        \(n\). L'ajout d'un n{\oe}ud revient � remplacer un sous-arbre
        vide par un sous-arbre non vide, et donc d'ajouter deux
        sous-arbres vides. Au total on a donc ajout� un sous-arbre
        vide en ajoutant un n{\oe}ud.

      \item Soit \(p\) le nombre de n{\oe}uds ayant un seul fils et
        \(q\) le nombre de n{\oe}uds sans fils. On a: \(p + 2q = n -
        1\) (le nombre de sous-arbres vides), donc \(q = (n-1-p)/2\).

      \item La hauteur maximale d'un arbre de taille \(n\) est au plus
        \(n-1\) (cas de l'arbre d�g�n�r� en liste\footnote{On parle
          alors de \emph{peigne}}). Un arbre binaire non vide de
        hauteur \(h\) a au plus \(1 + \ldots + 2^{h} = 2^{h+1} - 1\)
        n{\oe}uds. Soit \(a\) un arbre binaire de taille \(n\) et
        \(h=\lfloor\log_2{n}\rfloor\): tout arbre binaire non vide de
        hauteur inf�rieure ou �gale � \(h-1\) a donc au plus \(2^{h} -
        1 < n\) n{\oe}uds. La hauteur de \(a\) est au moins �gale �
        \(h\).

    \end{enumerate}

    \item Un arbre binaire est dit \emph{�quilibr� en hauteur} s'il
      est vide ou si pour tout n{\oe}ud, les sous-arbres gauches et
      droits ont m�me hauteur � une unit� pr�s.


\end{enumerate}
