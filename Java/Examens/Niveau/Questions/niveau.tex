%%-*-latex-*-

\documentclass[a4paper]{article}

\usepackage[francais]{babel}
\usepackage[T1]{fontenc}
\usepackage[latin1]{inputenc}
\usepackage{amssymb}

/home/rinderkn/LaTeX/TeX/trace.tex

\newcommand{\T}{\textrm{T}}
\newtheorem{Def}{Definition}[section]

\title{Test de niveau en algorithmique}
\author{Christian Rinderknecht}
\date{Mardi 4 mars 2003}

\begin{document}

\maketitle

\section{Bin�me de Pascal}

\noindent Soit la fonction \verb+binome+ sp�cifi�e ainsi:

\begin{verbatim}
binome (n,p) {
  if p = 0 or p = n then return 1
  else return binome (n-1,p) + binome (n-1,p-1);
}
\end{verbatim}
\paragraph{Question.} Design a protocol to be used between an automatic teller
machine (ATM) and a bank's centralised computer. Your protocol should
allow a user's card and password to be verified, the account balance
(which is maintained at the bank) to be queried, and an account
withdrawal to be made (money is given to the
customer). \textbf{1.~Specify your protocol} by listing the messages
exchanged and the action taken by the ATM or the bank's centralised
computer on transmission or receipt of each message:
\begin{center}
\begin{tabular}{@{}p{170pt}p{170pt}@{}}
\toprule
\multicolumn{2}{c}{From ATM to Bank}\\
\cmidrule(r{0pt}){1-2}
Message name & Meaning/Action\\
and arguments &\\
\midrule
 & \\
 & \\
 & \\
 & \\
 & \\
 & \\
 & \\
 & \\
\bottomrule
\end{tabular}

\begin{tabular}{@{}p{170pt}p{170pt}@{}}
\toprule
\multicolumn{2}{c}{From Bank to ATM}\\
\cmidrule(r{0pt}){1-2}
Message name & Meaning/Action\\
and arguments &\\
\midrule
 & \\
 & \\
 & \\
 & \\
 & \\
 & \\
 & \\
 & \\
\bottomrule
\end{tabular}
\end{center}
\noindent\textbf{2.~Sketch the operation of your protocol}, using a
diagram, for the cases of a simple withdrawal \textbf{(a)} with no
errors, \textbf{(b)} with one error.


\section{Complexit� et notations de Landau}

Give the Morris-Pratt algorithm, assuming you know the supply function.


\section{Tris}

\paragraph{Question.} Define the meaning of the pointers
\(\upharpoonleft\), \(\upharpoonright\) and \(\Uparrow\) presented in
class and show how the input is analysed using the transition diagrams
of the previous questions.



\section{Arbres}

  \subsection{Arbres binaires}

  \paragraph{Question.} Decipher the secret message by converting
  \textbf{efficiently} from octal to hexadecimal the following
  numbers:
\begin{center}
5413 \quad 3726355 \quad 12 \quad 157255 \quad 150320
\end{center}

        
  \subsection{Arbres binaires de recherche}

  \paragraph{Question 4.} Simplify, if possible, the following regular
expressions.
\begin{gather*}
  \lparen \epsilon \, \disjM{} \, a\kleeneM{} \, \disjM{} \,
  b\kleeneM{} \, \disjM{} \, a \, \disjM{} \, b \rparen\kleeneM\\
  a \lparen a \, \disjM{} \, b \rparen\kleeneM{} b 
  \, \disjM{} \, \lparen a b \rparen\kleeneM{} 
  \, \disjM{} \, \lparen b a \rparen\kleeneM
\end{gather*}


\section{Tables de hachage}

  \begin{enumerate}
 
    \item Qu'est-ce qu'une table de hachage? Dans quelles
      circonstances sont-elles une structure de donn�e utile?

    \item Pourquoi recommande-t-on souvent de choisir un nombre
      premier pour la taille d'une table de hachage?

    \item Soient un ensemble \(E\) de \(n\) �l�ments et une fonction
      \(h: E \rightarrow [1..m]\) uniforme (c-�-d. \(\forall e \in
      E.\forall i \in [1..m].{\cal P} \{h(e)=i\}=1/m\)). Montrez que
      la probabilit� \(P\) que \(h\) soit injective vaut \(m!/(m-n)!
      m^{n}\). En particulier, si \(m=356\) et \(n=23\), alors \(P <
      1/2\). Parieriez-vous que deux �l�ves dans cette classe soient
      n�es le m�me jour du m�me mois? Qu'en conclure � propos des
      tables de hachage?
 
    \item Expliquez les m�thodes de r�solution des collisions par
      cha�nage interne et externe (\emph{hachage indirect}), et par
      calcul (\emph{hachage direct}).

  \end{enumerate}


\section{Indexation}

On d�sire �crire un programme qui, � la lecture d'un programme, rep�re
pour chaque identificateur les lignes o� il appara�t. Un
identificateur commence par une lettre et n'est compos� que de lettres
et de chiffres. Certains mots, dits r�serv�s, ne peuvent pas �tre
utilis�s comme identificateurs.

Apr�s la lecture, on demande d'imprimer le programme, puis la liste
(tri�e par ordre alphab�tique) des identificateurs suivis chacun de la
liste des num�ros de ligne o� il appara�t, ces num�ros de ligne �tant
rang�s en ordre croissant.

  \begin{enumerate}

    \item Faites une analyse compar�e des diff�rentes structures de
          donn�es possibles;

    \item programmez une m�thode utilisant des arbres binaires de
          recherche, et une m�thode utilisant un tableau de hachage;
          analysez les r�sultats.

  \end{enumerate}


\section{Graphes}

  \begin{enumerate}

    \item Qu'est-ce qu'un graphe non-orient�? Orient�?

    \item Qu'est-ce qu'un graphe connexe? Fortement connexe?

    \item Qu'est-ce qu'un parcours en profondeur d'abord? En largeur
          d'abord?

    \item Donnez un algorithme de calcul des composantes fortement
          connexes, et sa complexit� temporelle et spatiale dans le
          pire des cas.

    \item Citez des cas o� les graphes sont la structure de donn�e
          idoine.

  \end{enumerate}


\section{Automates et langages r�guliers}

  \begin{enumerate}

    \item Qu'est-ce qu'un automate fini?

    \item Qu'est-ce qu'un automate fini d�terministe et
          non-d�terministe? Quels sont les liens entre ces deux types
          d'automates?

    \item Qu'est-ce qu'un automate minimal? Y a-t-il unicit�? Donnez
          un algorithme de minimisation et discutez sa complexit�.

    \item Qu'est-ce qu'un langage r�gulier? Une expression r�guli�re?
 
    \item Quels liens y a-t-il entre les langages r�guliers et les
          automates finis?

    \item Citez des applications int�ressantes des automates finis et
          des expressions r�guli�res.

  \end{enumerate}



\end{document}
