%%-*-latex-*-

\documentclass[11pt,a4paper]{article}

\usepackage[francais]{babel}
\usepackage[T1]{fontenc}
\usepackage{ae,aecompl}
\usepackage[latin1]{inputenc}
\usepackage{xspace}

\usepackage{amsmath}
\usepackage{amssymb}

\newcommand{\boxitem}{\item[$\Box$]}
\newcommand{\java}{\textsf{Java}\xspace}

/home/rinderkn/LaTeX/TeX/trace.tex

\title{Examen 1 de Programmation en Java}
\author{Christian Rinderknecht}
%\date{Mercredi 19 novembre 2003}

\begin{document}

\maketitle

\noindent
\textbf{Dur�e: deux heures. Les documents et les calculatrices ne sont
 pas autoris�s.}

\smallskip

\section{Questionnaire � choix multiples}

\emph{Attention: il peut y avoir plusieurs bonnes r�ponses par question.}

\begin{enumerate}

  \item Quelles commandes lancent la compilation d'un programme \java?
  \begin{itemize}
    \boxitem \texttt{java prog.java}
    \boxitem \texttt{javac prog}
    \boxitem \texttt{javac prog.java}
    \boxitem \texttt{java prog.class}
  \end{itemize}

  \item Quelles commandes lancent l'ex�cution d'un programme \java?
  \begin{itemize}
    \boxitem \texttt{javac prog}
    \boxitem \texttt{java prog.class}
    \boxitem \texttt{java prog}
    \boxitem \texttt{javac prog.java}
  \end{itemize}

  \item Qu'affiche l'extrait de programme suivant?
  \begin{verbatim}
int value = 0;
int count = 1;
value = count++;
System.out.println("value:"+value+" count:"+count);
  \end{verbatim}
  \vspace*{-10pt}
  \begin{itemize}
    \boxitem \verb+value:0 count:0+
    \boxitem \verb+value:0 count:1+
    \boxitem \verb+value:1 count:1+
    \boxitem \verb+value:1 count:2+
  \end{itemize}
 
  \item Quelle est la valeur de \verb+values[2][1]+ dans le tableau
  suivant?
  \begin{verbatim}
double[][] values =
  {{1.2,9.0,3.2},{9.2,0.5,1.5,-1.2},{7.3,7.9,4.8}}
  \end{verbatim}
  \begin{itemize}
    \boxitem \verb+7.3+
    \boxitem \verb+7.9+
    \boxitem \verb+9.2+
    \boxitem Il n'y a pas d'�l�ment correspondant � ces indices.
  \end{itemize}

  \item Quelle est la valeur retourn�e par \verb+things.length+?
  \begin{verbatim}
double[][] things =
  {{1.2,9.0},{9.2,0.5,0.0},{7.3,7.9,1.2,3.9}}
  \end{verbatim}
  \vspace*{-10pt}
  \begin{itemize}
    \boxitem \verb+2+
    \boxitem \verb+3+
    \boxitem \verb+4+
    \boxitem \verb+9+
  \end{itemize}

  \item �tant donn� la d�claration de variable suivante:
  \verb+long[][] stuff;+
  Quelles instructions ci-dessous construisent un tableau de 5~lignes et
  7~colonnes et l'affecte � \verb+stuff+?
  \begin{itemize}
    \boxitem \verb+stuff = new stuff[5][7]+
    \boxitem \verb+stuff = new long[5][7]+
    \boxitem \verb+stuff = long[5][7]+
    \boxitem \verb+stuff = long[7][5]+
  \end{itemize}

  \item �tant donn�e la d�claration 
  \begin{verbatim}
int[][] items = {{0,1,3,4},{4,4,99,0,7},{3,2}};
  \end{verbatim}
  \vspace*{-10pt}
  Quelles boucles affichent tous les �l�ments de \verb+items+?
  \begin{itemize}
    \boxitem 
      \begin{verbatim}
for (int row=0; row < items.length; row++) {
  System.out.println();
  for (int col=0; col < items.length; col++)
    System.out.print (items[row][col] + " ");
}
      \end{verbatim}
      \vspace*{-10pt}
    \boxitem
      \begin{verbatim}
for (int row=0; row < items.length; row++) {
  System.out.println();
  for (int col=0;col < items[col].length; col++)
    System.out.print (items[row][col] + " ");
}
      \end{verbatim}
      \vspace*{-10pt}
    \boxitem
      \begin{verbatim}
for (int row=0; row < items.length; row++) {
  System.out.println();
  for (int col=0; col < items[row].length; col++)
    System.out.print (items[row][col] + " ");
}
      \end{verbatim}
      \vspace*{-10pt}
    \boxitem
      \begin{verbatim}
for (int row=0; row < items.length; row++) {
  for (int row=0; row < items[row].length; row++)
    System.out.print (items[row][col] + " ");
  System.out.println();
}
      \end{verbatim}
  \end{itemize}

  \item Soit la d�claration\\
  \verb+int[][] items = {{0,1,3,4},{4,3,99,0,7},{3,2}}+\\
  Quelles instructions remplacent enti�rement les valeurs de la
  premi�re ligne de \texttt{items}?
  \begin{itemize}

      \boxitem 
      \begin{verbatim}
items[0][0] = 8;
items[0][1] = 12;
items[0][2] = 6;
      \end{verbatim}
      \vspace*{-10pt}
    
      \boxitem \verb+items[0] = new {8,12,6};+
 
      \boxitem
      \begin{verbatim}
int[] temp = {8,12,6};
items[0] = temp;
      \end{verbatim}
      \vspace*{-10pt}

      \boxitem \verb+items[0] = {8,12,6};+
  \end{itemize}
    
  \item Dans l'extrait de code suivant, qu'elle est la valeur de
    \texttt{discount}?
    \begin{verbatim}
double discount;
char code = 'C';

switch (code) {
  case 'A':
    discount = 0.0;
  case 'B':
    discount = 0.1;
  case 'C':
    discount = 0.2;
  default:
    discount = 0.3;
}
    \end{verbatim}

  \item Qu'affiche l'extrait de programme suivant?
  \begin{verbatim}
int[] tab = {1,4,3,6,8,2,5};
int what = tab[0];

for (int index=0; index < tab.length; index++) {
  if (tab[index] < what) what = tab[index];
}
System.out.println (what);
  \end{verbatim}
  \vspace*{-10pt}
  \begin{itemize}
    \boxitem \verb+1+
    \boxitem \verb+5+
    \boxitem \verb+1 4 3 6 8 2 5+
    \boxitem \verb+8+
  \end{itemize}

\end{enumerate}


\section{Exercices}

\begin{enumerate}

  \item Que fait et affiche la suite d'instructions suivante?
  \begin{verbatim}
int[] z = new int[10];
for (int i=0; i < z.length; i++) {
  z[i] = i+1;
}
for (int i = z.length - 1; i >= 0; i--) {
  System.out.println (z[i]);
}
  \end{verbatim}

  \vspace*{4cm}
  
  \item �crivez, une fois avec une boucle \texttt{for}, une fois
  avec une boucle \texttt{while}, le code qui calcule et affiche la
  valeur de $4^5$.

\pagebreak

  \item �crire un programme \texttt{ChangeSousChaine} qui permet de
  substituer une sous-cha�ne par une autre dans une cha�ne saisie par
  l'utilisateur sur la ligne de commande.

  \noindent Exemple:

  \noindent \texttt{java ChangeSousChaine "Bonjour Mister Java" Mister
  Madame}
  
  \noindent devra exactement afficher:

  \noindent
  \texttt{Le remplacement de "Mister" par "Madame" dans la\\
  phra\-se "Bonjour Mister Java" donne:}\\
  \texttt{Bonjour Madame Java}

  \noindent De m�me

  \noindent \texttt{\small java ChangeSousChaine "Bonjour Mister Java"
  Bonjour Bonsoir}

  \noindent devra exactement afficher

  \noindent \texttt{Le remplacement de "Bonjour" par "Bonsoir" dans la
  phrase\\
  "Bonjour Mister Java" donne:}\\
  \texttt{Bonsoir Mister Java}

  \noindent \emph{R�ponse:}


\end{enumerate}


\end{document}
