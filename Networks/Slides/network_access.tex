%%-*-latex-*-

% ------------------------------------------------------------------------
%
\begin{frame}
\frametitle{Plan}

\begin{enumerate}

  \item Computer Networks and the Internet

    \begin{itemize}

      \item What is the Internet?

      \item The network edge

      \item The network core

      \item \textbf{Network access and physical media}

      \item ISPs and Internet backbones

      \item Delay and loss in packet-switched networks
 
      \item Protocol layers and their service models

    \end{itemize}

\end{enumerate}

\end{frame}

% ------------------------------------------------------------------------
%
\begin{frame}
\frametitle{Network access and physical media}

The network access is the physical media that connect an end system to
its \textbf{edge router}, which is the first router on the way to
another end system.

\bigskip

The access links in the facing picture are in thick blue.

\end{frame}

% ------------------------------------------------------------------------
%
\begin{frame}
\frametitle{Network access and physical media (cont)}

\begin{center}
  \includegraphics[scale=0.3]{01-14.eps}
\end{center}

\end{frame}

% ------------------------------------------------------------------------
%
\begin{frame}
\frametitle{Network access and physical media}

Network access can be loosely classified into three categories:
\begin{itemize}

  \item \textbf{Residential access}, connecting home end systems into
  the network;

  \item \textbf{Company access}, connecting end systems in a business
  or educational institution in the network;

  \item \textbf{Mobile access}, connecting mobile end systems into
  the network.

\end{itemize}
Note that these categories are not tight, for instance a company may
use access technology we ascribe to residential access.

\end{frame}

% ------------------------------------------------------------------------
%
\begin{frame}
\frametitle{Network access/Residential access}

Residential access refers to connecting a home end system, say a
computer, to an edge router.

\bigskip

In Europe, a common way is the \textbf{dial-up modem} over a wired
telephone line. The modem converts the digital output of the computer
into the analog format of the phone line, which is a
\textbf{twisted-pair copper wire} (same as a phone cable). On the
other side of the phone line, at the ISP, there is another modem which
converts back the analog signal into digital signal which is directed
to the edge router.

\bigskip

The data rate can be up to 56 Kbps, with hardware compression and if
the telephone line is of good quality.

\bigskip

Disadvantages are a slow data rate for nowadays applications and the
fact that the phone line is entirely devoted to the modem.

\end{frame}

% ------------------------------------------------------------------------
%
\begin{frame}
\frametitle{Network access/Residential access (cont)}

Other access technologies are \textbf{digital subscriber line}
(\textbf{DSL}) and \textbf{hybrid fiber coaxial cable} (\textbf{HFC}).

\bigskip

DSL is a new modem technology running over existing twisted-pair
telephone line. But by shortening the distance from home to the ISP,
the data rate is much higher. The rate is not symmetric: the rate from
the ISP to home (called \textbf{downstream}) is higher than from home
to the ISP (called \textbf{upstream}). Indeed, the standard assumes
that the user wants more probably to get information than produce
data.

\bigskip

In theory, DSL can provide rates of more than 1 Mbps from ISP to home
and more than 1 Mbps from home to ISP, but in practice the real rates
are much lower.

\end{frame}

% ------------------------------------------------------------------------
%
\begin{frame}
\frametitle{Network access/Residential access (cont)}

DSL uses frequency division multiplexing (FDM), see
page~\pageref{fdm_tdm}. In particular, DSL divides the communication
link into three non-overlapping frequency bands:
\begin{itemize}

  \item a high-speed downstream channel, in the 50kHz-1MHz band;

  \item a low-speed upstream channel, in the 4kHz-50kHz band;

  \item an ordinary two-way telephone channel, in the 0-4kHz band.

\end{itemize}
This allows to keep the same link available for conference calls while
being connected to the network.

\end{frame}

% ------------------------------------------------------------------------
%
\begin{frame}
\frametitle{Network access/Residential access/HFC}

HFC are extensions of the existing cable network for television
broadcasting. 
\begin{center}
\includegraphics[scale=0.35]{01-15.eps}
\end{center}

\end{frame}

% ------------------------------------------------------------------------
%
\begin{frame}
\frametitle{Network access/Residential access/HFC (cont)}

As with DSL, HFC requires special modems, called \textbf{cable
modems}. This is an external device that is connected to the home
computer through a \textbf{10-BaseT Ethernet} port (discussed later). 

\bigskip

Cable modems divide the HFC network into downstream and upstream
channels. As with DSL, the former is faster than the latter.

\bigskip

There is an important difference with DSL: HFC is a \textbf{shared
broadcast medium}.

\end{frame}

% ------------------------------------------------------------------------
%
\begin{frame}
\frametitle{Network access/Residential access/HFC (cont)}

Every packet sent by the head end is sent (on the downstream channel)
to \emph{all} the connected homes (this is broadcasting, a useful
feature for television). 

\bigskip

So if users download at the same time, the actual rate is slowed
down. But if they are surfing the web, the pages will be likely to
arrive at full speed, since it is improbable they all click at the
same time.

\bigskip

Since the upstream channel is also shared, two packets sent from
different homes at the same time will collide (but there is no
broadcasting on the upstream channel).

\end{frame}

% ------------------------------------------------------------------------
%
\begin{frame}
\frametitle{Network access/Company access}

On corporate and university campuses, a \textbf{local area network}
(\textbf{LAN}) is used to connect an end system to the edge
router. This means that the end system has to be connected first to
the LAN.

\bigskip

The most access technology to do so is the \textbf{Ethernet}. It uses
either twisted-pair copper wire or coaxial cable with each other and
with an edge router. The rates vary depending on the version: 10 Mbps,
100 Mbps, 1 Gbps or 10 Gbps.

\bigskip

Like HFC, Ethernet uses a shared medium, so end users share the
transmission rate of the LAN.

\bigskip

There is now \textbf{switched Ethernet}, which uses multiple
twisted-pair Ethernet segments connected at a special switch.

\end{frame}

% ------------------------------------------------------------------------
%
\begin{frame}
\frametitle{Network access/Mobile access}

The most popular wireless access to the Internet is through a
\textbf{wireless LAN} (or \textbf{Wi-Fi}). Mobile users are connected
to a radio \textbf{base station}, also called \textbf{wireless access
point}, which is itself connected to the Internet. The IEEE 802.11b
provides a bandwidth of 11 Mbps.

\begin{center}
\includegraphics[scale=0.3]{01-16.eps}
\end{center}

\end{frame}
