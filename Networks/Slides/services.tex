%%-*-latex-*-

% ------------------------------------------------------------------------
%
\begin{frame}
\frametitle{Plan}

\begin{enumerate}

  \item Computer Networks and the Internet

    \begin{itemize}

      \item What is the Internet?

      \item The network edge

      \item The network core

      \item Network access and physical media

      \item ISPs and Internet backbones

      \item Delay and loss in packet-switched networks
 
      \item \textbf{Protocol layers and their service models}

    \end{itemize}

\end{enumerate}

\end{frame}

% ------------------------------------------------------------------------
%
\begin{frame}
\frametitle{Layered architecture}

Until now we have been presenting the different components (switches,
hosts, links), their relationships, different switching policies and
different protocols, or services.

\bigskip

We need now a model to describe the \textbf{architecture} of the
services provided by the Internet.

\bigskip

A useful analogy is an airline system.

\bigskip

One way to guess a structure of it, is to describe the actions taken
to make a trip: first we purchase a flight ticket, then we go to
airport, we give our luggage to be checked, then we board the plane as
the baggage is loaded, then we take off, the plane travels to
destination, lands, we get off the plane, our baggage is unloaded, we
claim it and we complain to the ticketing service if the trip was bad.

\end{frame}

% ------------------------------------------------------------------------
%
\begin{frame}
\frametitle{Layered architecture (cont)}

The figure in the next slide summarises the series of actions
undertaken to make an airplane travel. 

\bigskip

It appears that each kind of activity, based on a functionality, in
the departing airport has a corresponding counterpart in the arrival
airport.

\bigskip

This suggests that we can look at the services in a \emph{horizontal
manner}, i.e., as belonging to same levels.

\end{frame}

% ------------------------------------------------------------------------
%
\begin{frame}
\frametitle{Layered architecture (cont)}

\begin{center}
  \includegraphics[scale=0.4]{01-20.eps}
\end{center}

\end{frame}

% ------------------------------------------------------------------------
%
\begin{frame}
\frametitle{Layered architecture (cont)}

The following picture changes the focus according to the horizontal
point of view on the service usages we described so far. The structure
is divided into \textbf{service layers}.

\begin{center}
\includegraphics[scale=0.45]{01-21.eps}
\end{center}

\end{frame}

% ------------------------------------------------------------------------
%
\begin{frame}
\frametitle{Layered architecture (cont)}

This layered architecture is of great value because it allows to
reason on a specific part of the system, just by considering the upper
layer (to which the current layer is a provider) and the lower layer
(to which the current layer is a customer).

\bigskip

For instance, this model allows to change the service of one layer in
a way that is not noticed by the surrounding layers. 

\bigskip

The way a service is achieved internally can be changed without
changing its observable behaviour (like reorganising the workflow of
the staff at the airport gates: this does not change anything outside
the service).

\bigskip

It is even possible sometimes to change some properties of the service
that are not noticed by the surrounding layers, like imposing that
boarding should be done depending on the height of the passengers:
this is not noticed by the baggage check and the runway takeoff
services.

\end{frame}

% ------------------------------------------------------------------------
%
\begin{frame}
\frametitle{Layered architecture (cont)}

Coming back to networks services, these are also structured into
layers. The input and output behaviour of one layer (service) is the
\textbf{protocol}. The way this behaviour is achieved internally to
the layer is the \textbf{protocol implementation}. 

\bigskip

As we just saw, these are different notions: you can change the
protocol (as imposing more constraints on the output, like sorting by
height the passengers), or independently change its implementation (as
reorganising the staff at the gates).

\bigskip

Notice also that a protocol in a given layer is implemented by remote
entities of the network, just as the services in the airplane system
were \emph{distributed} between the departing and arriving airports.

\end{frame}

% ------------------------------------------------------------------------
%
\begin{frame}
\frametitle{Layered architecture (cont)}

In a network, the protocols can either be implemented by hardware or
software, or even by a mixture of hardware and software.

\bigskip

Each implementation of layer \(n\) communicates with its peer at the
same level \(n\) by exchanging messages whose structure is specific to
this level, called \textbf{\(n\)-layer protocol data units} or, in
short, \textbf{\(n\)-PDUs}.

\bigskip

On one entity, all the protocols taken as a whole is called the
\textbf{protocol stack}.

\end{frame}


% ------------------------------------------------------------------------
%
\begin{frame}
\frametitle{Layered architecture (cont)}

When the layer \(n\) of host A sends an \(n\)-PDU to layer \(n\) of
host B, it passes the PDU to the \(n-1\) layer of host A and lets it
deliver the \(n\)-PDU to the layer \(n\) of host B. Thus layer \(n\)
is said to \emph{rely} on layer \(n-1\) for achieving the delivering,
and, in turn, layer \(n-1\) \emph{offers services} to layer \(n\).

\bigskip

Layer \(n\) is an \textbf{upper layer} with respect to layer \(n-1\),
which is then a \textbf{lower layer}.

\end{frame}

% ------------------------------------------------------------------------
%
\begin{frame}
\frametitle{Layered architecture (cont)}

Let us give more insight about protocol layering. Consider a network
with four layers in the following figure.
\begin{center}
\includegraphics[scale=0.37]{01-22.eps}
\end{center}

\end{frame}

% ------------------------------------------------------------------------
%
\begin{frame}
\frametitle{Layered architecture (cont)}

The application message, \(M\), is cut into two packets, \(M_1\) and
\(M_2\). They are both sent to the layer below, which is layer~3. This
layer does not know how to interpret the contents of \(M_1\) and
\(M_2\), and does not care about it. His role is to add a
\textbf{header} \(H_3\) to each packet, whose structure is specific to
layer~3, and then passes them to the layer below.

\bigskip

Layer~2, similarly, does not know the structure of the incoming 3-PDU,
and does not care. It just adds a header \(H_2\) specific to layer~2.
The next layer behaves similarly.

\bigskip

The remote protocol stack works symmetrically, as the message goes
up. Each level reads the outermost header (supposed to correspond, by
construction, to the current level) and removes it.

\bigskip

Finally, the application message is reassembled back into \(M\).

\end{frame}

% ------------------------------------------------------------------------
%
\begin{frame}
\frametitle{Layered architecture (cont)}

In order for one layer to \textbf{interoperate} with the layer below
it, the \textbf{interfaces} between the two layers must be precisely
defined. 

\bigskip

An interface contains the format of the PDU as well as the kind of
functionality a layer provides (you can think to the types of
functions in programming languages).

\bigskip

This allows to improve the implementation of a service to be unnoticed
from adjacent layers, as long as the interface remains the same (this
is similar to changing the code of a function, while keeping its
type).

\end{frame}

% ------------------------------------------------------------------------
%
\begin{frame}
\frametitle{Layered architecture (cont)}

In a computer networks, each layer may perform one or more of the
following general tasks:
\begin{itemize}

  \item \textbf{Error control}, which makes the logical channel
  (i.e., from one peer to another, at the same stack level, while
  ignoring the lower levels) more reliable.

  \item \textbf{Flow control}, which avoids overwhelming a slower
  peer with PDUs.

  \item \textbf{Segmentation and reassembly}, which divides data at
  the source and reassembles the pieces at the destination.

  \item \textbf{Multiplexing}, which allows several high-level
  sessions (i.e., full runs of a given service instance) to share a
  single lower-level channel.

  \item \textbf{Connection setup}, which provides handshaking with a
  peer.

\end{itemize}

\end{frame}

% ------------------------------------------------------------------------
%
\begin{frame}
\frametitle{Layered architecture (cont)}

The \textbf{Internet protocol stack} is made of five layers: the
physical layer, the data link layer, the network layer, the transport
layer and the application layer (bottom-up).

\bigskip

It is common usage to give a name to the four upper layers, instead of
\(n\)-PDU. They are, from data link layer to application layer:
\textbf{frame}, \textbf{datagram}, \textbf{segment},
\textbf{message}.

\end{frame}

% ------------------------------------------------------------------------
%
\begin{frame}
\frametitle{Layered architecture (cont)}

\begin{center}
  \includegraphics[scale=0.40]{01-23.eps}
\end{center}

\end{frame}

% ------------------------------------------------------------------------
%
\begin{frame}
\frametitle{Layered architecture (cont)}

The \textbf{application layer} supports network-oriented programs
and may include many different protocols, like \textsc{http} to
support the Web, \textsc{smtp} to support e-mail and \textsc{ftp} to
support file transfer.

\bigskip

The \textbf{transport layer} provides the service of transporting
application-layer messages between the client and the server. In the
Internet, there are two protocols at this level: \textsc{tcp} and
\textsc{udp}.

\bigskip

The \textbf{network layer} routes datagrams from one host to the
other. It has two main components. One defines the fields of the
datagrams as well as how the hosts and the routers acts on these
fields: this is the famous \textsc{IP} protocol. Another one contains
routing protocols, which determines the paths between sources and
destinations. 

\end{frame}

% ------------------------------------------------------------------------
%
\begin{frame}
\frametitle{Layered architecture (cont)}

The \textbf{link layer} is in charge of moving a packet from one
router to another. By contrast, the network layer moves a packet
along the whole path, not just two switches. 

\bigskip

At each node, the network layer passes the datagram to the link
layer, which delivers the corresponding frame to the next node. At
this node, the frame is passed up to the network layer again.

\bigskip

This is because the services provided by the link layer depends on the
specific link protocols, which may be different for the incoming link
and the out-going link (hence the frame must go up to the network
layer at each node). For instance, one link can use \textsc{ppp} and
the next one Ethernet.

\end{frame}

% ------------------------------------------------------------------------
%
\begin{frame}
\frametitle{Layered architecture (cont)}

The \textbf{physical layer} is in charge of moving bits along a link,
by propagating electromagnetic signals along a link. As with the link
layer, the protocols at this layer are link-dependent and also depend
on the actual transmission rate of the link.

\bigskip

For instance, Ethernet has many physical layer protocols: one for
twisted-pair copper wire, another for coaxial cable, another for fiber
optics etc. This is because in each case the way to move a bit across
the link is different.

\end{frame}

% ------------------------------------------------------------------------
%
\begin{frame}
\frametitle{Layered architecture (cont)}

\begin{center}
\includegraphics[scale=0.40]{01-24.eps}
\end{center}
\textbf{Bridges} are a special kind of switches (they do not support IP).

\end{frame}
