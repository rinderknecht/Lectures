%%-*-latex-*-

% ------------------------------------------------------------------------
%
\begin{frame}
\frametitle{What is the Internet?}

Millions of connected computing devices, called \textbf{end systems}
(PCs, workstations, servers, PDAs, phones, game consoles etc.),
running network applications, i.e., software for accessing and using
the network.

\bigskip

End systems are also called \textbf{hosts}. Hosts are sometimes
divided into \textbf{clients} and \textbf{servers}.

\end{frame}

% ------------------------------------------------------------------------
%
\begin{frame}
\frametitle{What is the Internet? (cont)}

\begin{center}
  \includegraphics[scale=0.28]{01-01.eps}
\end{center}

\end{frame}

% ------------------------------------------------------------------------
%
\begin{frame}
\frametitle{What is the Internet? (cont)}

\textbf{Communication links}, e.g., fiber optics, copper wire, radio
wave, satellite, connect hosts to each other; the transmission rate is
called \textbf{bandwidth} (measured in bits per seconds, or
``bit/sec'').

\bigskip
   
The \textbf{routers} are located between hosts; they receive
\textbf{packets} (chunks of data) from originating hosts and forward
them to their destination hosts.

\end{frame}

% ------------------------------------------------------------------------
%
\begin{frame}
\frametitle{What is the Internet? (cont)}

\begin{center}
  \includegraphics[scale=0.28]{01-01.eps}
\end{center}

\end{frame}

% ------------------------------------------------------------------------
%
\begin{frame}
\frametitle{What is the Internet? (cont)}

Hosts access the Internet through \textbf{Internet Service Providers}
(ISP), which are networks proposing many different kinds of
connections to the users: dial-up modem, cable modem (DSL), high-speed
Local Area Network (LAN) access, wireless access.

\bigskip

There are ISPs in universities, companies and for residents.

\end{frame}

% ------------------------------------------------------------------------
%
\begin{frame}
\frametitle{What is the Internet? (cont)}

\begin{center}
  \includegraphics[scale=0.28]{01-01.eps}
\end{center}

\end{frame}

% ------------------------------------------------------------------------
%
\begin{frame}
\frametitle{What is the Internet? (cont)}

ISPs are connected to others ISPs in a hierarchy: at the bottom lie
the content providers ISPs and, at the top, the international ISPs
(such as UUNet and Sprint) with high-speed routers interconnected with
high-speed fiber-optics links.

\bigskip

So, what is the Internet? 

\begin{itemize}

  \item As the name tells it (\emph{Inter\/}connected \emph{net\/}works),
  it is a worldwide system of computer networks: \textbf{a network of
  networks} with a loose hierarchy.

  \item These networks share the same \textbf{protocols}, i.e., ways
  of representing information (packets) and rules for accepting,
  refusing or sending messages.

\end{itemize}

\end{frame}

% ------------------------------------------------------------------------
%
\begin{frame}
\frametitle{What is the Internet? (cont)}

\begin{itemize}

  \item This network of networks is called \textbf{public Internet}. 

  \item Many companies have developed private networks based on the
  same hosts, links, routers and protocols as the public Internet:
  they are called \textbf{intranets}. 

  \item These intranets are connected to the (public) Internet
  through \textbf{firewalls}, which filter and restrict the
  information flows in and out.

\end{itemize}

\end{frame}

% ------------------------------------------------------------------------
%
\begin{frame}
\frametitle{What is the Internet? (cont)}

\begin{itemize}

  \item Some Internet protocol names are \textsc{http} (for the Web),
  \textsc{tcp}, \textsc{ip}, \textsc{ftp} (file transfer protocol),
  \textsc{ppp} (for modem connection), \textsc{smtp} (e-mails) etc.

  \item Internet standards are developed by the \textbf{Internet
  Engineering Task Force} (IETF).

  \item The IETF standards documents are called \textbf{Requests For
  Comments} (RFC).

\end{itemize}

\end{frame}

% ------------------------------------------------------------------------
%
\begin{frame}
\frametitle{What is the Internet?/A service view}

\begin{itemize}

  \item The Internet allows applications that inter-operate to run on
  the hosts. These are called \textbf{distributed applications} and
  include remote login, e-mail, web surfing, instant messaging, audio
  and video streaming, Internet telephony, distributed games,
  peer-to-peer (P2P) file sharing, voting, databases etc.

  \item The network provides services to the distributed
  applications, i.e., networking\hyp{}oriented features that
  programmers can use when they write such an application. Typically
  there are \textbf{connection-oriented services} and
  \textbf{connectionless services}.

  \item A connection-oriented service is \textbf{reliable}: it
  guarantees that the data is delivered in order and entirely. A
  connectionless service is \textbf{unreliable}: it guarantees
  nothing about delivery.

\end{itemize}

\end{frame}

% ------------------------------------------------------------------------
%
\begin{frame}
\frametitle{What is the Internet?/Protocols}

What is a protocol?

\begin{itemize}

  \item Protocols are part of human relationships. When we meet a
  friend we start greeting him (`Hi!') and wait for the similar
  greeting. If it does not come or if some unpleasant words come back,
  we understand that the communication will not be possible or
  not good. Otherwise, we go on talking.

  \item The machine counterpart of this introduction is a
  connection-oriented service: first the sending application informs
  the remote application that it wants to communicate (i.e., exchange
  data). The remote application must acknowledge that before data is
  sent.

\end{itemize}

\end{frame}

% ------------------------------------------------------------------------
%
\begin{frame}
\frametitle{What is the Internet?/Protocols (cont)}

\begin{itemize}
 
  \item The conversation between friends is similar to the data
  transmission, from one application to another.

  \item So, in human protocols, some specific messages are exchanged
  (e.g., greetings, goodbyes) and some specific actions are taken when
  messages are received or other events happen.

  \item Similarly network protocols define format and order of
    messages exchanged between hosts as well as actions to be taken
    upon message receipt or transmission.

\end{itemize}

\end{frame}

% ------------------------------------------------------------------------
%
\begin{frame}
\frametitle{What is the Internet?/Protocols (cont)}
\label{protocols}

\begin{center}
\includegraphics[scale=0.3]{01-02.eps}
\end{center}  
 
\end{frame}

% ------------------------------------------------------------------------
%
\begin{frame}
\frametitle{A closer look at the network structure}

\begin{itemize}

  \item The \textbf{network edge} is made of the hosts and their
  applications. 

  \item The \textbf{network core} is made of the routers and the
  protocols that enable the network of networks.

  \item The \textbf{access network} is the communication links.

\end{itemize}

\end{frame}
