%%-*-latex-*-

\documentclass[11pt,a4paper]{article}

/home/rinderkn/LaTeX/TeX/trace.tex

\usepackage{graphicx}

\title{Quiz \#1 of introduction to networking} 
\author{Christian Rinderknecht} 
\date{\today}

\begin{document}

\maketitle


\section{Review questions}

\begin{enumerate}

  \item What is the difference between a host and an end system? List
  different types of end systems. Is a mail server an end system?

  \item What is a client program? What is a server program? 

  \item What are the two kinds of services that the Internet provides
  to the applications? Describe some typical features of each.

  \item Are the objectives of flow control and congestion control the
  same?

  \item Give a very short description of how the connection-oriented
  service of the Internet provides reliable transport. What is the
  name of this service, by the way?

  \item Suppose there is exactly one packet switch between a sending
  host and a receiving host. The transmission rates between the
  sending host and the switch and between the switch and the receiving
  host are \(R_1\) and \(R_2\), respectively. Assuming that the switch
  uses store-and-forward packet switching, what is the total
  end-to-end delay to send a packet of length \(L\)? (Ignore queuing,
  propagation delay and processing delay.)

\end{enumerate}

\section{Problems}

\begin{enumerate}

  \item Design and describe an application-level protocol to be used
  between an automatic teller machine (ATM) and a bank's centralised
  computer. Your protocol should allow a user's card and password to
  be verified, the account balance (which is maintained at the bank's
  centralised computer) to be queried, and an account withdrawal to be
  made (money is given to the customer).

  Specify your protocol by listing the messages exchanged and the
  action taken by the ATM or the bank's centralised computer on
  transmission or receipt of each message.

  Sketch the operation of your protocol for the case of a simple
  withdrawal with no errors, using a diagram similar to
  Figure~\ref{protocols}. Explicitly state the assumptions made
  by your protocol about the underlying end-to-end transport service.

  \begin{figure}
  \begin{center}
  \includegraphics*[scale=0.35]{01-02.eps}
  \caption{Sketch of protocols}\label{protocols}
  \end{center}
  \end{figure}
 
  \item Consider an application that transmits data at a steady rate:
  it generates an \(N\)-bit unit of data every \(k\) time units, where
  \(k\) is small and fixed. Also, when such an application starts, it
  will continue running for a long period of time. Answer the
  following questions, briefly justifying your answer.

  \begin{enumerate}
 
    \item Would a packet-switched network or a circuit-switched
    network be more appropriate for this application? Why?

    \item Suppose that a packet-switched network is used and the only
    traffic in this network comes from such applications as described
    above. Furthermore, assume that the sum of the application data
    rates is less than the capacities of each and every link. Is some
    form of congestion control needed? Why?

  \end{enumerate}

\end{enumerate}

\end{document}
