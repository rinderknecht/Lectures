\section{Review questions}

\begin{enumerate}

  \item \emph{What advantage does a circuit-switched network have over
    a packet-switched network? What advantage does TDM have over FDM
    in a circuit-switched network?}

  A circuit-switched network can guarantee a certain amount of
  end-to-end bandwidth for the duration of the call. Most
  packet-switched networks today (including the internet) cannot make
  any end-to-end guarantees for bandwidth. In a circuit-switched
  network using TDM, an application can use the full bandwidth at
  periodical moments.

  \item \emph{Why is it that packet switching is said to employ
    statistical multiplexing? Contrast statistical multiplexing with
    the multiplexing that takes place in TDM.}

  In a packet-switched network, the packets from different sources
  flowing into a link do not follow any fixed pattern, or route. This
  is why packet switching is said to employ statistical
  multiplexing. In case of TDM circuit switching, each host gets the
  same slot in a revolving TDM frame: this is completely predictable.

  \item \emph{What is meant by connection state information in a
    virtual circuit network?}

  In a virtual circuit network, each packet switch keeps in memory
  some information (like a table translating interface numbers to
  virtual circuit numbers) about the virtual circuits passing through
  them.

  \item \emph{Suppose you are developing a standard for a new type of
    network. You need to decide whether your network will use VCs or
    datagram routing. What are the pros and cons for using VCs?}

   The cons of VC's include
   \begin{itemize}

     \item the need to have a signaling protocol to set up and
       teardown the VCs;
  
     \item the need to maintain connection state in the packet
       switches.

   \end{itemize}
   The main advantage of VC networks is that they allow to guarantee
   an end-to-end delay.

  \item \emph{What are the advantages of message segmentation in
    packet-switched networks? What are the disadvantages?}

  One advantage of message segmentation is that it allows for
  pipelined transmission over a series of links. Another advantage is
  that, without it, small messages would be stuck behind much bigger
  ones in routers.

  \item \emph{What is the key distinguishing difference between a
    tier-1 ISP (backbone) and a tier-2 ISP?}

  A tier-1 ISP connects to all other tier-1 ISPs; a tier-2 ISP
  connects to only a few tier-1 ISPs. Also, a tier-2 ISP is a customer
  of one or more tier-1 ISPs.

  \item \emph{Is HFC bandwidth dedicated or shared among users? Are
    collisions possible in a downstream HFC channel?}

  HFC bandwidth is shared among the users. On the downstream channel
  all the packets emanate from a single source, called the head end,
  so there are no collisions on this channel.

  \item \emph{Consider sending a series of packets from a sending host
    to a receiving host over a fixed route. List the delay components
    in the end-to-end delay for a single packet. Which of these delays
    are constant and which are variable?}

  The delay components are nodal processing delays, transmission
  delays, propagation delays and queuing delays. Over a fixed route,
  all these delays are fixed, except the queuing delay, which is
  unpredictable.

  \item \emph{List five tasks that a protocol layer can perform. Is it
    possible that one (or more) of these tasks could be performed by
    two (or more) layers?}
   
  Five generic tasks are error control, flow control, segmentation and
  reassembly, multiplexing and connections set-up. These tasks can be
  duplicated at different levels. For example, error control is often
  provided at more than one layer.
  
  \item \emph{What are the five layers in the Internet protocol stack?
    What are the principal responsibilities of each of these layers?}

  The five layers of the Internet are, from top to bottom -- the
  application layer, the transport layer, the network layer, the link
  layer and the physical layer.

  \item \emph{What information is used by a process running on one
    host to identify a process running on another host?}

  The IP address of the destination host and the port number of the
  destination socket.

  \item \emph{What is the difference between persistent \textsc{http}
    with pipelining and persistent \textsc{http} without pipelining?}

  In persistent \textsc{http} without pipelining, the browser first
  waits to receive a \textsc{http} response from the server before
  issuing a new request. In persistent \textsc{http} with pipelining,
  the browser issues requests as soon as it has the need to do so,
  without waiting for response messages from the server.

\end{enumerate}

