%%-*-latex-*-

\documentclass[11pt,a4paper]{article}

/home/rinderkn/LaTeX/TeX/trace.tex

\title{Mid-term examination on\\ Introduction to the Internet}
\author{Christian Rinderknecht}
\date{Spring 2006}

\begin{document}

\maketitle
\thispagestyle{empty}

\begin{enumerate}

  \item What advantage does a circuit-switched network have over a
    packet-switched network? What advantage does TDM have over FDM in
    a circuit-switched network?

  \item Why is it that packet switching is said to employ statistical
    multiplexing? Contrast statistical multiplexing with the
    multiplexing that takes place in TDM.

  \item What is meant by connection state information in a virtual
    circuit network? 

  \item Suppose you are developing a standard for a new type of
    network. You need to decide whether your network will use virtual
    circuits (VC) or datagram routing. What are the pros and cons for
    using VCs?

  \item What are the advantages of message segmentation in
    packet-switched networks? What are the disadvantages?

  \item What is the key distinguishing difference between a tier-1 ISP
    (backbone) and a tier-2 ISP?

  \item Is HFC bandwidth dedicated or shared among users? Are
    collisions possible in a downstream HFC channel?

  \item Consider sending a series of packets from a sending host to a
    receiving host over a fixed route. List the delay components in
    the end-to-end delay for a single packet. Which of these delays
    are constant and which are variable?

  \item List five tasks that a protocol layer can perform. Is it
    possible that one (or more) of these tasks could be performed by
    two (or more) layers?
 
\end{enumerate}


\end{document}
