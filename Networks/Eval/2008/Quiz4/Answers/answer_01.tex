%%-*-latex-*-

\paragraph{Answers.}

\begin{enumerate}

  \item A circuit-switched network can guarantee a certain amount of
    end-to-end bandwidth for the duration of the call. Most
    packet-switched networks today (including the internet) cannot
    make any end-to-end guarantees for bandwidth. In a
    circuit-switched network using TDM, an application can use the
    full bandwidth at periodical moments.

  \item In a packet-switched network, the packets from different
    sources flowing into a link do not follow any fixed pattern, or
    route. This is why packet switching is said to employ statistical
    multiplexing. In case of TDM circuit switching, each host gets the
    same slot in a revolving TDM frame: this is completely
    predictable.

  \item In a virtual circuit network, each packet switch keeps in
    memory some information (like a table translating interface
    numbers to virtual circuit numbers) about the virtual circuits
    passing through them.

  \item The cons of VC's include
   \begin{itemize}

     \item the need to have a signaling protocol to set up and
       teardown the VCs;
  
     \item the need to maintain connection state in the packet
       switches.

   \end{itemize}
   The main advantage of VC networks is that they allow to guarantee
   an end-to-end delay.

  \item One advantage of message segmentation is that it allows for
    pipelined transmission over a series of links. Another advantage
    is that, without it, small messages would be stuck behind much
    bigger ones in routers.

  \item HFC bandwidth is shared among the users. On the downstream
    channel all the packets emanate from a single source, called the
    head end, so there are no collisions on this channel.

  \item The delay components are nodal processing delays, transmission
    delays, propagation delays and queuing delays. Over a fixed route,
    all these delays are fixed, except the queuing delay, which is
    unpredictable.

  \item 
  \begin{enumerate}
 
    \item A circuit-switched network would be well suited to this
      application, because it involves long sessions with predictable
      bandwidth requirements. So bandwidth can be reserved for each
      session. The delay of setting up a circuit is low compared to
      the time the application is running.

    \item Given such generous link capacities, the network need no
    congestion control. In the worst case, all the applications are
    emitting on the same link, but the link offers enough bandwidth.

  \end{enumerate}

\end{enumerate}
