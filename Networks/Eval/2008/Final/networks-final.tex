%%-*-latex-*-

\documentclass[11pt,a4paper]{article}

/home/rinderkn/LaTeX/TeX/trace.tex

\title{Final examination on Introduction to the Internet}
\author{Christian Rinderknecht}
\date{20 June 2008}

\usepackage[english]{babel}

\begin{document}

\maketitle

\thispagestyle{empty}

\section{Review questions}

\begin{enumerate}

  \item What are the five layers in the Internet protocol stack? What
    are the principal responsibilities of each of these layers?

  \item What information is used by a process running on one host to
    identify a process running on another host?

  \item What is the difference between persistent \textsc{http}
    with pipelining and persistent \textsc{http} without pipelining?

  \item Describe briefly cookies in web browsers. Why are they useful?

  \item What is the conditional \texttt{GET} \textsc{http} request
    useful for?

\end{enumerate}

\section{True or false?}

  \begin{enumerate}

    \item Suppose a user requests a Web page that consists of some text
      and two images. For this page the client will send one request
      and receive three response messages.

    \item Two distinct Web pages can be sent over the same persistent
      connection.

    \item With non-persistent connections between browser and origin
      ser\-ver, it is possible for a single TCP segment to carry two
      distinct \textsc{http} request messages.

    \item The \verb+Date:+ header in the \textsc{http} response
      message indicates when the object in the response was last
      modified.

  \end{enumerate}


\end{document}
