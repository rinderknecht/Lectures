%%-*-latex-*-

\documentclass[10pt,a4paper]{article}

\usepackage[francais]{babel}
\usepackage[T1]{fontenc}
\usepackage[latin1]{inputenc}
\usepackage{xspace}

\usepackage{amsmath}
\usepackage{amssymb}
\usepackage{mathpartir}
\usepackage{ae,aecompl}

/home/rinderkn/LaTeX/TeX/commands.tex
/home/rinderkn/LaTeX/TeX/trace.tex

\title{Contr�le continu de compilation}
\author{Christian Rinderknecht}
\date{Mercredi 8 octobre 2003}

\begin{document}

\maketitle

\noindent
\textbf{Dur�e: deux heures. Longueur: deux pages. Les documents et les
calculatrices ne sont pas autoris�s.}

\smallskip


\bigskip

\begin{enumerate}

  \item Quelle est la structure simplifi�e d'un compilateur? Que sont
  les phases d'analyse et de synth�se?

  \item Qu'est-ce que le typage? Pourquoi est-ce utile que le typage
  soit statique?

  \item Qu'est-ce qu'un interpr�te? Comparez un interpr�te � un
  compilateur.

  \item Qu'est-ce que le lexique d'un langage? Qu'est-ce que la
  syntaxe concr�te d'un langage? Qu'est-ce que la syntaxe abstraite
  d'un langage?

  \item Consid�rez la session suivante:
  {\small
  \begin{verbatim}
$ javac Toto.java
Toto.java:3: integer number too large: 300000000000
        int x = 300000000000;
                ^
1 error
  \end{verbatim}
  }

  En supposant que le compilateur \textsf{javac} est bien con�u, �
  quel moment a-t-il �mis ce message?

  \item � quoi sert la s�mantique formelle d'un langage de
  programmation?

  \item Qu'est-ce qu'une s�mantique op�rationnelle d�terministe? En
  quoi ce d�terminisme est-il pertinent?

  \item En suivant la syntaxe concr�te de O'Caml, donnez une syntaxe
  abstraite et une s�mantique op�rationnelle pour une calculette
  comprenant les cons\-tan\-tes enti�res, les bool�ens, les variables,
  les quatres op�rations arithm�tiques, la conditionnelle et la
  liaison locale --- sans tenir compte des erreurs. Commentez vos
  choix.

  \item �tablissez formellement la valeur de \texttt{let x = 1 in
  ((let x = x in x) + x)}.

  \item Consid�rez la s�mantique op�rationnelle de la division qui ne
  sp�cifie pas l'ordre d'�valuation des op�randes. R��crivez la r�gle
  pour qu'elle force l'�valuation de son second op�rande avant le
  premier. Pour cela, inspirez-vous de la r�gle de la liaison locale
  et faites attention aux captures de variables temporaires.

  L'implantation fid�le de cette nouvelle s�mantique est
  maladroite. Proposez une variation simple du code O'Caml pour la
  division qui r�pond au probl�me.

  \item Ajoutez � votre s�mantique la prise en compte de toutes les
  erreurs possibles tout en minimisant le nombre de calculs. Commentez
  vos choix.

  \item R�pondez formellement aux questions: 

  \begin{enumerate}

    \item Qu'est-ce qu'une variable libre dans une expression? 

    \item Qu'est-ce qu'une expression close?

    \item Quels sont les variables libres de \texttt{let x = 1 in ((let
    x = 2 in x) + x)}?

  \end{enumerate}

  \item Est-ce utile de d�terminer les variables libres d'une
  expression avant son �valuation?

\end{enumerate}


\end{document}
