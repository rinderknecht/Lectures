%% -*-latex-*-

\documentclass[a4paper,11pt]{article}

\usepackage[francais]{babel}
\usepackage[T1]{fontenc}
\usepackage[latin1]{inputenc}
\usepackage{ae,aecompl}

/home/rinderkn/LaTeX/TeX/commands.tex
/home/rinderkn/LaTeX/TeX/ocaml_syntax.tex
/home/rinderkn/LaTeX/TeX/trace.tex

\title{TP 1 d'Interpr�tation et compilation}
\author{Christian Rinderknecht}
\date{23 et 24 septembre 2003}

\begin{document}

\maketitle

Le but de ce TP est de r�aliser la calculette fonctionnelle pr�sent�e
lors du premier cours et de l'�tendre. Un certain nombre d'�l�ments
vous sont fournis sous la forme de fichiers (analyseur lexical et
syntaxique, un d�but de programme principal et un
\texttt{Makefile}). � vous de construire l'�valuateur.

\bigskip

Ouvrez l'archive \texttt{tp1.tgz}, et construisez la calculette
initiale avec \texttt{make calc}. L'ex�cutable produit s'appelle
\texttt{calc}, et les fichiers � modifier et compl�ter sont
\texttt{eval.ml} et \texttt{main.ml}. Les autres fichiers fournis sont
les suivants (vous n'avez pas besoin de les modifier):

\begin{itemize}

  \item \texttt{ast.ml}: le type des arbres de syntaxe abstraite;

  \item \texttt{lexer.mll}: la sp�cification de l'analyseur
  lexical. Sa compilation par \textsf{ocamllex} produit
  \texttt{lexer.ml} (l'analyseur lexical).

  \item \texttt{parser.mly}: le sp�cification de la syntaxe concr�te
  (la grammaire). Sa compilation par \textsf{ocamlyacc} produit
  \texttt{parser.mli} (interface sp�cifiant le type des lex�mes et le
  type de l'analyseur syntaxique) et \texttt{parser.ml}, contenant
  l'analyseur syntaxique proprement dit.

\end{itemize}

\medskip

\noindent Le fonctionnement global de la calculette est le suivant:

\begin{enumerate}

  \item L'analyseur lexical \textsf{Lexer} exporte une fonction
  \textsf{token} qui renvoie le premier lex�me reconnaissable �
  partir de l'entr�e standard (par d�faut il s'agit du clavier);

  \item l'analyseur syntaxique \textsf{Parser} exporte une fonction
  \textsf{expression} qui prend en argument \textsf{Lexer.token} et
  l'entr�e standard, et renvoie l'arbre de syntaxe abstraite qui doit
  �tre �valu�;

  \item l'�valuateur \textsf{Eval} exporte une fonction \textsf{eval}
  qui prend en argument un environnement et l'arbre de syntaxe
  abstraite correspondant � l'expression, et renvoie la valeur
  associ�e ou d�clenche une exception en cas d'erreur;

  \item le pilote \textsf{Main} invite l'usager � saisir une
  expression, lance l'analyse lexico-syntaxique puis l'�valuation
  (dans un environnement vide); il affiche ensuite le r�sultat. En cas
  d'erreur, il affiche l'erreur dont le message est associ� aux
  exceptions provenant de l'�valuateur.

\end{enumerate}

\noindent Pour l'instant, l'�valuateur ne fait que renvoyer la cha�ne
\texttt{"ok"} quelque soit l'expression (attention, le r�le de
l'�valuateur n'est pas de faire des affichages: c'est un des r�les du
pilote), et le pilote passe effectivement l'AST � l'�valuateur, mais
il ne g�re aucune erreur provenant de celui-ci.

\noindent Le travail � faire est le suivant:

%%-*-latex-*-

\begin{enumerate}

  \item \label{q\theenumi} Implantez la calculette pr�sent�e dans le
  cours sans tenir compte des cas d'erreurs, puis traitez ceux-ci �
  l'aide des exceptions O'Caml.

  \item \label{q\theenumi} Ajoutez maintenant la liaison locale et les
  variables. Identifiez les erreurs possibles et ajoutez leur
  gestion. Testez des cas d'�valuation correcte et des cas provoquant
  ces erreurs. (\emph{Gardez une trace de vos cas de test.}) Par
  exemple:

  \begin{center}
  \texttt{1+2*(ifz 0 then let x = 0 in x+3 else 4)}
  \end{center}

  \item \label{q\theenumi} On souhaite se doter d'une conditionnelle
  \emph{sans ajout des bool�ens}, dans le style du langage C:

  \begin{itemize}

    \item[$\bullet$] \textbf{Syntaxe concr�te}
  {\small 
   \begin{verbatim}
Expression ::=  ...
  | "ifz" Expression "then" Expression "else" Expression
   \end{verbatim}
  }

  \vspace*{-6pt}

  \item[$\bullet$] \textbf{Exemple} \qquad \texttt{1+2*(ifz 0 then 3
  else 4)}

  \item[$\bullet$] \textbf{Syntaxe abstraite}

  \Xtype{} \type{expr} \equal{} \texttt{...} \vbar{} \cst{Ifz}
  \Xof{} \type{expr} $\times$ \type{expr} $\times$ \type{expr}\semi

  \end{itemize}

  Donnez une s�mantique op�rationnelle pour cette conditionnelle et
  compl�tez votre �valuateur par son implantation.

  \item \label{q\theenumi} Enrichissez la calculette par les bool�ens,
  les op�rateurs logiques de conjonction et disjonction, ainsi que
  d'une conditionnelle dont la condition est bool�enne. Un lexique et
  une grammaire sont respectivement propos�s dans les fichiers
  \textsf{lexer.mll} et \textsf{parser.mly}. Proposez une s�mantique
  op�rationnelle et une implantation en O'Caml.

  \item \label{q\theenumi} Le codage fonctionnel pour les
  environnements est tr�s inefficace car, dans le pire des cas, pour
  obtenir la valeur d'une variable il faut �valuer autant d'appels de
  fonction qu'il y a de variables dans l'environnement, et un appel de
  fonction est tr�s co�teux, en g�n�ral. Il est plus judicieux, dans
  un premier temps, de remplacer les appels de fonction par des acc�s
  dans une liste. Ainsi un environnement est une liste qui associe les
  variables aux valeurs. La recherche d'une valeur se fait alors �
  l'aide de la fonction \textsf{List.assoc}~:~$\forall \alpha,
  \beta.\alpha \rightarrow (\alpha \times \beta) \, \textsf{list}
  \rightarrow \beta$. Conservez un exemplaire de \texttt{eval.ml} et
  r��crivez-le en implantant ce sch�ma et testez �
  nouveau. (\emph{Gardez une trace de vos cas de test.})

\end{enumerate}


\end{document}
