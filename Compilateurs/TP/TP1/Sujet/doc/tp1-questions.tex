%%-*-latex-*-

\begin{enumerate}

  \item \label{q\theenumi} Implantez la calculette pr�sent�e dans le
  cours sans tenir compte des cas d'erreurs, puis traitez ceux-ci �
  l'aide des exceptions O'Caml.

  \item \label{q\theenumi} Ajoutez maintenant la liaison locale et les
  variables. Identifiez les erreurs possibles et ajoutez leur
  gestion. Testez des cas d'�valuation correcte et des cas provoquant
  ces erreurs. (\emph{Gardez une trace de vos cas de test.}) Par
  exemple:

  \begin{center}
  \texttt{1+2*(ifz 0 then let x = 0 in x+3 else 4)}
  \end{center}

  \item \label{q\theenumi} On souhaite se doter d'une conditionnelle
  \emph{sans ajout des bool�ens}, dans le style du langage C:

  \begin{itemize}

    \item[$\bullet$] \textbf{Syntaxe concr�te}
  {\small 
   \begin{verbatim}
Expression ::=  ...
  | "ifz" Expression "then" Expression "else" Expression
   \end{verbatim}
  }

  \vspace*{-6pt}

  \item[$\bullet$] \textbf{Exemple} \qquad \texttt{1+2*(ifz 0 then 3
  else 4)}

  \item[$\bullet$] \textbf{Syntaxe abstraite}

  \Xtype{} \type{expr} \equal{} \texttt{...} \vbar{} \cst{Ifz}
  \Xof{} \type{expr} $\times$ \type{expr} $\times$ \type{expr}\semi

  \end{itemize}

  Donnez une s�mantique op�rationnelle pour cette conditionnelle et
  compl�tez votre �valuateur par son implantation.

  \item \label{q\theenumi} Enrichissez la calculette par les bool�ens,
  les op�rateurs logiques de conjonction et disjonction, ainsi que
  d'une conditionnelle dont la condition est bool�enne. Un lexique et
  une grammaire sont respectivement propos�s dans les fichiers
  \textsf{lexer.mll} et \textsf{parser.mly}. Proposez une s�mantique
  op�rationnelle et une implantation en O'Caml.

  \item \label{q\theenumi} Le codage fonctionnel pour les
  environnements est tr�s inefficace car, dans le pire des cas, pour
  obtenir la valeur d'une variable il faut �valuer autant d'appels de
  fonction qu'il y a de variables dans l'environnement, et un appel de
  fonction est tr�s co�teux, en g�n�ral. Il est plus judicieux, dans
  un premier temps, de remplacer les appels de fonction par des acc�s
  dans une liste. Ainsi un environnement est une liste qui associe les
  variables aux valeurs. La recherche d'une valeur se fait alors �
  l'aide de la fonction \textsf{List.assoc}~:~$\forall \alpha,
  \beta.\alpha \rightarrow (\alpha \times \beta) \, \textsf{list}
  \rightarrow \beta$. Conservez un exemplaire de \texttt{eval.ml} et
  r��crivez-le en implantant ce sch�ma et testez �
  nouveau. (\emph{Gardez une trace de vos cas de test.})

\end{enumerate}
