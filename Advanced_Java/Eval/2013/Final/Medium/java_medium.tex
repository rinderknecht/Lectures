%%-*-latex-*-

\documentclass[11pt,a4paper]{article}

\usepackage[british]{babel}
\usepackage[T1]{fontenc}
\usepackage[utf8]{inputenc}

/home/rinderkn/LaTeX/TeX/trace.tex

\title{Final examination\\ of Introduction to Networking}

\author{Christian Rinderknecht}
\date{24 October 2013}

\begin{document}

\maketitle

\noindent Consider the following classes implementing ephemeral
stacks:
\begin{verbatim}
public class Stack<Key extends Comparable<Key>> {
    protected Cell<Key> top;
    public Stack () { top = null; }
}

public class Cell<Key extends Comparable<Key>> {
  protected Key value;
  protected Cell<Key> next;
  public Cell (final Key key, final Cell<Key> cell) {
    value = key; next = cell; }
}
\end{verbatim}
Extend the class \texttt{Stack<Key>} with two methods defined as follows.
\begin{enumerate}

  \item The method \texttt{rm\_fst} (\emph{remove first}) whose
  signature is
\begin{verbatim}
public void rm_fst (final Key k);
\end{verbatim}
  performs a linear search in \texttt{this}: if the key~\texttt{k} is
  absent, nothing is done; otherwise, the (first) cell
  containing~\texttt{k} is taken off \texttt{this}.

\item The method \texttt{rm\_lst} (\emph{remove last}) whose signature
  is
\begin{verbatim}
public void rm_lst (final Key k);
\end{verbatim}
removes the last cell containing~\texttt{k}, and, otherwise, do nothing.
\end{enumerate}

\newpage

\noindent The main class for testing is
\begin{verbatim}
public class EphObj {
  public static void main (String[] args) {
    Stack<Integer> s = new Stack<Integer>();
    s.push(5); s.push(1); s.push(3); s.push(1);
    s.push(0); s.push(3);
    s.print(); // 3 0 1 3 1 5
    s.rm_fst(3);
    s.print(); // 0 1 3 1 5
    s.rm_fst(1);
    s.print(); // 0 3 1 5
    s.rm_fst(5);
    s.print(); // 0 3 1
    s.rm_fst(6);
    s.print(); // 0 3 1

    s.push(1); s.push(5); s.push(3);
    s.print(); // 3 5 1 0 3 1
    s.rm_lst(3);
    s.print(); // 3 5 1 0 1
    s.rm_lst(1);
    s.print(); // 3 5 1 0
    s.rm_lst(3);
    s.print(); // 5 1 0
    s.rm_lst(4);
    s.print(); // 5 1 0
  }
}
\end{verbatim}

\end{document}
