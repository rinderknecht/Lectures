\paragraph{Answer.}
\begin{align*}
\proc{Bottom}(S) &\rightarrow \proc{Bottom}'(\proc{Rev}(S))\\
\proc{Bottom}'(\proc{Push}(e,T)) &\rightarrow (e, \proc{Rev}(T))
\end{align*}

\paragraph{Note.} The two questions are related. Indeed,
\proc{Bottom} provides a means to use a stack as a queue:
\proc{Enqueue} is simply \proc{Push} and \proc{Dequeue} is
\proc{Bottom}! But \proc{Bottom} is expensive: in order to get the
element at the bottom, i.e. dequeue, we have to reverse the whole
stack two times (see \(\proc{S}\) and \(\proc{T}\)), except one
element. But in the first question, \proc{Dequeue} is cheap: if the
second stack is not empty, it is simply equivalent to a \proc{Pop}
(one rewrite). If it is empty, then the cost is to reverse the first
stack, i.e. to traverse \emph{only once} all the elements --- instead
of two times, every time, with \proc{Bottom}.
