%%-*-latex-*-

% ------------------------------------------------------------------------
% Knuth, Morris and Pratt
%
\begin{frame}
\frametitle{Knuth-Morris-Pratt algorithm}

The algorithm we present now is an improvement due to Knuth of the
Morris-Pratt algorithm, based on avoiding situations which lead to
certain letter comparison failures.
\[\small
\begin{array}{rcc@{\,}c@{}ccr@{}r@{\,}c@{}ccc@{}ccc}
  & & & \scriptstyle j-i & & & 
  & & \multicolumn{2}{@{}l}{\scriptstyle j-i+p}
  & &
  & \multicolumn{1}{r@{}}{\scriptstyle j-1}
  & \multicolumn{1}{c}{\scriptstyle j}\\
\cline{2-15}
    \multicolumn{1}{r|}{t:}
  & \multicolumn{2}{c}{\phantom{l}\ldots\phantom{l}}
  & \multicolumn{3}{|c|}{\Border{}{x[1 \dots i-1]}}
  & \multicolumn{2}{c|}{\ldots}
  & \multicolumn{5}{c}{\Border{}{x[1 \dots i-1]}}
  & \multicolumn{1}{|>{\columncolor{mygrey}}c|}{\textcolor{white}{\texttt{b}}}
  & \phantom{ll}\\
\cline{2-15}
  &
  & 
  & \multicolumn{1}{:c}{\scriptstyle 1}
  &
  & \multicolumn{1}{c:}{}
  & 
  & \multicolumn{1}{r@{\,}:}{\scriptstyle p}
  & 
  &
  &
  &
  & \multicolumn{1}{r@{}:}{\scriptstyle i-1}
  & \scriptstyle i\\ 
\cline{4-15}
    \multicolumn{1}{r}{x:}
  &
  &
  & \multicolumn{3}{|c|}{\Border{}{x[1 \dots i-1]}}
  & \multicolumn{2}{c}{\ldots}
  & \multicolumn{5}{|c}{\Border{}{x[1 \dots i-1]}}
  & \multicolumn{1}{|>{\columncolor{mygrey}}c|}{\textcolor{white}{\texttt{a}}}
  & \\
\cline{4-15}
  &&&&&&
  & 
  &\multicolumn{1}{:c}{\scriptstyle 1}
  &&&& \multicolumn{1}{r@{}:}{\scriptstyle \MPsupply{x}{}{i-1}}
  & \multicolumn{1}{@{}c@{}}{\scriptstyle \sX{x}{}{i}}\\
\cline{9-15}
  &&&&&
  & \text{slided}
  & \multicolumn{1}{c}{x:}
  & \multicolumn{5}{|c}{\Border{}{x[1 \dots i-1]}}
  & \multicolumn{1}{|c|}{a}\\
\cline{9-15}
\end{array}
\]
The observation is that if \(a = \texttt{a}\) then the sliding would
lead to a comparison failure. So let us enforce that a sliding to
compare \(x[\sX{x}{}{i}]\) to \(t[j]\) is done if and only if
  \(x[\sX{x}{}{i}] \neq x[i]\).

\end{frame}

% ------------------------------------------------------------------------
%
\begin{frame}
\frametitle{Knuth-Morris-Pratt/Maximum disjoint borders}

If \(x[\sX{x}{}{i}] = x[i]\), what should we do?

\bigskip

Let us note \(y=x[1...i-1]\). We should consider successively
\(\Border{2}{y} \cdot \texttt{a}\), \(\Border{3}{y} \cdot \texttt{a}\)
etc., until we find a \(k\) such that \(\Border{k}{y} \cdot \texttt{a}
\npreccurlyeq x\) or \(\Border{k}{y} = \varepsilon\). Such a border
\(\Border{k}{y}\) is called a \textbf{maximum disjoint border} of
\(y\) in \(x\).

\bigskip

If it is non-empty, we slide the word so that we compare
\(x[\sX{x}{k}{i}]\) and \(t[j]\).

\bigskip

This disjointedness constraint can be precomputed on the pattern \(x\)
\emph{alone} before comparing it to the text \(t\): all we have to do
is to change the failure function \(s_{x}\) of the algorithm of Morris
and Pratt and provide a new one. \emph{The search algorithm itself
  does not change.}

\end{frame}

% ------------------------------------------------------------------------
%
\begin{frame}
\frametitle{Knuth-Morris-Pratt/Maximum disjoint borders (cont)}

The disjointedness implicitly entails that it is a relative concept:
we consider the disjoint border of a \emph{proper prefix} of another
word. Thus, the letter with follows the prefix is used as a
constraint, i.e., it must not follow the disjoint border.

\bigskip

Let \(ya \preccurlyeq x\). Let us note \(\DBorder{x}{}{y}\) the
maximum disjoint border of \(y\) \emph{in \(x\)}. In this case, the
letter \(a\) is used to constrain the definition of the disjoint
border, as we must have \(\DBorder{x}{}{y} \cdot a \npreccurlyeq y\).

\bigskip

What if \(y = x\), then? In this case, there is no right-context, like
the letter \(a\) above, to constrain the maximum disjoint border. In
this case, we still can take the maximum border, i.e., \(\DBorder{y}{}{y} =
\Border{}{y}\).

\end{frame}

% ------------------------------------------------------------------------
%
\begin{frame}
\frametitle{Knuth-Morris-Pratt/Maximum disjoint borders (cont)}

More precisely, if the maximum border of \(y\) is not followed by
\(a\), then it is also the maximum disjoint border we are looking
for. In other words:
\[
\DBorder{x}{}{y} = \Border{}{y} \qquad \text{if} \; ya \preccurlyeq x
\; \text{and} \; \Border{}{y} \cdot a \npreccurlyeq y
\]
If the maximum border of \(y\) is followed by \(a\) we must take the
maximum disjoint border of the maximum border:
\[
\DBorder{x}{}{y} = \DBorder{x}{}{\Border{}{y}} 
\qquad \text{if} \; ya \preccurlyeq x \; \text{and} \; \Border{}{y}
\cdot a \preccurlyeq y
\]
By extension, if \(x = y\), we take the maximum border:
\[
\DBorder{y}{}{y} = \Border{}{y}
\]

\end{frame}

% ------------------------------------------------------------------------
%
\begin{frame}
\frametitle{Knuth-Morris-Pratt/Maximum disjoint borders (cont)}

In summary, assuming \(ya \preccurlyeq x\) \textbf{and} \(y \neq
\varepsilon\)
\[
\begin{aligned}
   \DBorder{y}{}{y}
&= \Border{}{y}\\
   \DBorder{x}{}{y}
&= \left\lbrace
   \begin{aligned}
    &  \DBorder{x}{}{\Border{}{y}}
    && \text{if} \; \Border{}{y} \cdot a \preccurlyeq y\\
    &  \Border{}{y}
    && \text{otherwise}
   \end{aligned}
   \right.
\end{aligned}
\]

\end{frame}

% ------------------------------------------------------------------------
%
\begin{frame}
\frametitle{Knuth-Morris-Pratt/Maximum disjoint borders (cont)}

By taking the length of each side of the equations,
\[
  \len{\DBorder{x}{}{y}}
= \left\lbrace
  \begin{aligned}
    &  \len{\Border{}{y}}
    && \text{if} \; x = y \; \text{or} 
       \; \Border{}{y} \cdot a \npreccurlyeq y\\
    &  \len{\DBorder{x}{}{\Border{}{y}}}
    && \text{otherwise}
  \end{aligned}
  \right.
\]
Let \(\KMPsupply{x}{}{i}\) be the length of the disjoint maximum border of
\(x[1 \dots i]\):
\[
\KMPsupply{x}{}{i} = \len{\DBorder{x}{}{x[1...i]}} 
\qquad 1 \leqslant i \leqslant \len{x}
\]
or, equivalently,
\[
\KMPsupply{x}{}{\len{y}} = \len{\DBorder{x}{}{y}} 
\qquad y \preccurlyeq x
\]

\end{frame}

% ------------------------------------------------------------------------
%
\begin{frame}
\frametitle{Knuth-Morris-Pratt/Maximum disjoint borders (cont)}

Therefore
\[
  \KMPsupply{x}{}{\len{y}}
= \left\lbrace
  \begin{aligned}
    &  \MPsupply{x}{}{\len{y}}
    && \text{if} \; x = y \; \text{or} 
       \; \Border{}{y} \cdot a \npreccurlyeq y\\
    &  \KMPsupply{x}{}{\len{\Border{}{y}}}
    && \text{otherwise}
  \end{aligned}
  \right.
\]
that is to say
\[
  \KMPsupply{x}{}{\len{y}}
= \left\lbrace
  \begin{aligned}
    &  \MPsupply{x}{}{\len{y}}
    && \text{if} \; x = y \; \text{or} 
       \; \Border{}{y} \cdot a \npreccurlyeq y\\
    &  \KMPsupply{x}{}{\MPsupply{x}{}{\len{y}}}
    && \text{otherwise}
  \end{aligned}
  \right.
\]
If \(\len{y} = i\), we can write instead
\[
  \KMPsupply{x}{}{i}
= \left\lbrace
  \begin{aligned}
    &  \MPsupply{x}{}{i}
    && \text{if} \; x = y \; \text{or} 
       \; \Border{}{y} \cdot a \npreccurlyeq y\\
    &  \KMPsupply{x}{}{\MPsupply{x}{}{i}}
    && \text{otherwise}
  \end{aligned}
  \right.
\]

\end{frame}

% ------------------------------------------------------------------------
%
\begin{frame}
\frametitle{Knuth-Morris-Pratt/Maximum disjoint borders (cont)}

\label{gamma_ref_def}

The condition can also be rewritten in terms of \(\MPsupplyName_{x}\)
and \(i\), just as we did with the Morris-Pratt algorithm,
pages~\pageref{condition} and~\pageref{morris-pratt-conf}. Let
\(\len{y} = i\), then
\begin{align*}
  \Border{}{y} \cdot a \npreccurlyeq y
& \Longleftrightarrow x [\MPsupply{x}{}{i} + 1] \neq x [i + 1]\\
  x = y
& \Longleftrightarrow \len{x} = i
\end{align*}
So, finally, for \(1 \leqslant i \leqslant \len{x}\),
\[
  \KMPsupply{x}{}{i}
= \left\lbrace
  \begin{aligned}
    &  \MPsupply{x}{}{i}
    && \text{if} \; i = \len{x} \; \text{or} 
       \; x [\MPsupply{x}{}{i} + 1] \neq x [i + 1]\\
    &  \KMPsupply{x}{}{\MPsupply{x}{}{i}}
    && \text{otherwise}
  \end{aligned}
  \right.
\]
which allows us to naturally extends \(\KMPsupplyName_{x}\) on \(0\): 
\(\KMPsupply{x}{}{0} = \MPsupply{x}{}{0} = -1\).

\end{frame}

% ------------------------------------------------------------------------
%
\begin{frame}
\frametitle{Knuth-Morris-Pratt/Maximum disjoint borders (cont)}

Before going further, let us check by hand the following values of
\(\KMPsupply{x}{}{i}\) and compare them to \(\MPsupply{x}{}{i}\): we must
always have \(\KMPsupply{x}{}{i} \leqslant \MPsupply{x}{}{i}\), since we
plan an optimisation. 
\[
\begin{array}{c|ccccccccccc|}
x & & \texttt{a} & \texttt{b} & \texttt{c} & \texttt{a} 
  & \texttt{b} & \texttt{a} & \texttt{b} & \texttt{c}
  & \texttt{a} & \texttt{c}\\
\hline
                  i  &  0 & 1 & 2 &  3 & 4 & 5 & 6 & 7 &  8 & 9 & 10\\
   \MPsupply{x}{}{i} & -1 & 0 & 0 &  0 & 1 & 2 & 1 & 2 &  3 & 4 &  0\\
  \KMPsupply{x}{}{i} & -1 & 0 & 0 & -1 & 0 & 2 & 0 & 0 & -1 & 4 &  0\\
\hline
\end{array}
\]
One difference between \(\MPsupplyName_{x}\) and
\(\KMPsupplyName_{x}\) is that, in the worst case, there is always an
empty maximum border \(\varepsilon\), i.e., \(\MPsupply{x}{}{i} = 0\),
whereas there may be no maximum disjoint border at all,
i.e., \(\KMPsupply{x}{}{i} = -1\). For instance, the prefix
\texttt{abc} of \(x\) has an empty maximum border,
i.e., \(\MPsupply{x}{}{3}=0\), but has no maximum disjoint border,
i.e., \(\KMPsupply{x}{}{3} = -1\), since \(x[1] = x[4]\).

\end{frame}

% ------------------------------------------------------------------------
%
\begin{frame}
\frametitle{Knuth-Morris-Pratt/Maximum disjoint borders (cont)}

This definition of \(\KMPsupplyName_{x}\) relies on
\(\MPsupplyName_{x}\), more precisely, the computation of
\(\KMPsupply{x}{}{i}\) requires \(\MPsupply{x}{p}{i}\), i.e.,
values \(\MPsupply{x}{}{j}\) with \(j < i\), since
\[
\MPsupply{x}{}{0} = -1 
\qquad \MPsupply{x}{}{i} = 1 + \MPsupply{x}{k}{i-1}
\qquad 1 \leqslant i
\]
where \(k\) is the smallest non-zero integer such that
\begin{itemize}

  \item either \(1 + \MPsupply{x}{k}{i-1} = 0\) or
    \(x[1+\MPsupply{x}{k}{i-1}] = x[i]\)

\end{itemize}
or, equivalently,
\[
\Border{}{ya} = 
\left\{
  \begin{aligned}
   & \Border{k}{y} \cdot a 
   && \text{if} \; \Border{k}{y} \cdot a \preccurlyeq y\\
   & \varepsilon 
   && \text{if} \; \Border{k-1}{y} = \varepsilon
  \end{aligned}
\right. 
\]
where \(k\) is the smallest non-zero integer such that \(\Border{k}{y}
\cdot a \preccurlyeq y\) or \(\Border{k-1}{y} = \varepsilon\).

\end{frame}

% ------------------------------------------------------------------------
%
\begin{frame}
\frametitle{Knuth-Morris-Pratt/Maximum disjoint borders (cont)}

Therefore, let us find another definition of \(\Border{}{ya}\) which
relies on \(\Border{}{y}\) but \textbf{not} on \(\Border{q}{y}\) with
\(2 \leqslant q\). This can be achieved by considering again the
figure
\[
\begin{array}{r@{}>{\,}cccccc@{\,}c}
\cline{2-8}
    \multicolumn{1}{r|}{y \cdot a:}
  & \multicolumn{2}{c|}{\Border{}{y}}
  & \multicolumn{1}{c|}{b}
  & \phantom{HHHHHHHH}
  & \multicolumn{2}{|c}{\Border{}{y}}
  & \multicolumn{1}{|>{\columncolor{mygrey}}c|}{\textcolor{white}{a}}\\
\cline{2-8}
\end{array}
\]
Indeed, if \(a = b\) then \(\Border{}{ya} = \Border{}{y}\). Else, \(a
\neq b\), and thus the maximum border can be found among the
\emph{disjoint} borders of \(\Border{}{y}\), otherwise \(\Border{}{ya} =
\varepsilon\).
\[
\Border{}{ya} = \left\{
\begin{aligned}
 & \DBorder{x}{q}{\Border{}{y}} \cdot a
 \quad \text{if} \; \DBorder{x}{q}{\Border{}{y}} \cdot a \preccurlyeq y\\
 & \varepsilon
 \qquad\qquad\qquad\quad \text{if} \; y = \varepsilon \;\text{or}\;
    \DBorder{x}{q-1}{\Border{}{y}} = \varepsilon
\end{aligned}
\right.
\]
where \(ya \preccurlyeq x\) and \(q\) is the smallest integer such
that \(\DBorder{x}{q}{\Border{}{y}} \cdot a \preccurlyeq y\) or
\(\DBorder{x}{q-1}{\Border{}{y}} = \varepsilon\).

\end{frame}

% ------------------------------------------------------------------------
%
\begin{frame}
\frametitle{Knuth-Morris-Pratt/Maximum disjoint borders (cont)}

By taking the lengths and letting \(0 \leqslant i \leqslant \len{x} - 1\)
and \(\len{y} = i\),
\begin{align*}
\MPsupply{x}{}{0} &= -1\\
\MPsupply{x}{}{i + 1} &=
\left\lbrace
\begin{aligned}
  & \KMPsupply{x}{q}{\MPsupply{x}{}{i}} + 1
  && \text{if} \; x[\KMPsupply{x}{q}{\MPsupply{x}{}{i}} + 1]=x[i + 1]\\
  & 0
  && \text{if} \; i = 0 \; \text{or} \;
     \KMPsupply{x}{q-1}{\MPsupply{x}{}{i}} = 0
\end{aligned}
\right.
\end{align*}
where \(q\) is the smallest integer such that
\begin{itemize}

  \item either \(x[\KMPsupply{x}{q}{\MPsupply{x}{}{i}} + 1]=x[i+1]\) 

  \item or \(\KMPsupply{x}{q-1}{\MPsupply{x}{}{i}} = 0\)

\end{itemize}

\end{frame}

% ------------------------------------------------------------------------
%
\begin{frame}
\frametitle{Knuth-Morris-Pratt/Maximum disjoint borders (cont)}

Since \(\KMPsupply{x}{}{i} = -1\) if and only if \(i=0\), we have
\begin{align*}
\KMPsupply{x}{q-1}{\MPsupply{x}{}{i}} = 0
&\Longleftrightarrow
\KMPsupply{x}{}{\KMPsupply{x}{q-1}{\MPsupply{x}{}{i}}} =
\KMPsupply{x}{}{0}
\Longleftrightarrow
\KMPsupply{x}{q}{\MPsupply{x}{}{i}} = -1\\
&\Longleftrightarrow
\KMPsupply{x}{q}{\MPsupply{x}{}{i}} + 1 = 0
\end{align*}
Therefore we can simplify the new definition of \(\MPsupplyName_{x}\)
as
\begin{align*}
\MPsupply{x}{}{0} &= -1\\
\MPsupply{x}{}{1} &= 0\\
\MPsupply{x}{}{i + 1} &= \KMPsupply{x}{q}{\MPsupply{x}{}{i}} + 1
\qquad 1 \leqslant i \leqslant \len{x} - 1
\end{align*}
where \(q\) is the smallest integer such that
\begin{itemize}

  \item either \(x[\KMPsupply{x}{q}{\MPsupply{x}{}{i}} + 1]=x[i+1]\)
  
  \item or \(\KMPsupply{x}{q}{\MPsupply{x}{}{i}} + 1 = 0\)

\end{itemize}

\end{frame}

% ------------------------------------------------------------------------
%
\begin{frame}
\frametitle{Knuth-Morris-Pratt/Maximum disjoint borders/Imperative}

{\small
\begin{codebox}
\(\proc{Gamma} (x)\)\\
\li \(\KMPsupplyName_{x}[0] \gets -1\); \(\id{offset} \gets -1\)
    \RComment \(\id{offset} = \MPsupply{x}{}{0}\)
\li \For \(i \gets 1\) \To \(\len{x}-1\) 
\li \Do \Repeat
\li       \(\id{offset} \gets \KMPsupplyName_{x}[\id{offset}]\)
\li     \Until \(\id{offset} = -1\) \LogOr \(x[\id{offset} + 1] = x[i]\)
\li     \(\id{offset} \gets \id{offset} + 1\)
        \RComment \(\id{offset} = \KMPsupply{x}{q}{\MPsupply{x}{}{i-1}} +
        1 = \MPsupply{x}{}{i}\)
\li     \If \(i = \len{x}\) \LogOr \(x[\id{offset}+1] = x[i+1]\)
\li     \Then \(\KMPsupplyName_{x}[i] \gets \id{offset}\)
              \RComment \(\KMPsupply{x}{}{i} = \MPsupply{x}{}{i}\)
\li     \Else \(\KMPsupplyName_{x}[i] \gets \KMPsupplyName_{x}[\id{offset}]\)
              \RComment \(\KMPsupply{x}{}{i} =
              \KMPsupply{x}{}{\MPsupply{x}{}{i}}\)
        \End
    \End
\li \(\proc{Gamma} \gets \KMPsupplyName_{x}\)
\end{codebox}
}

\end{frame}

% ------------------------------------------------------------------------
%
\begin{frame}
\frametitle{Knuth-Morris-Pratt/Better failure function}

With the algorithm of Morris and Pratt, it was convenient to use a
failure function \(s(i) = 1 + \MPsupplyName(i-1)\). Similarly, we need
here another failure function, noted \(r\), such that \(r(i) = 1 +
\KMPsupplyName(i-1)\), for \(1 \leqslant i\).

\bigskip

In order to finish, we need to give the algorithm for the failure
function \(r\).

\bigskip

We could then keep \proc{Beta} and create a separate function
\proc{KMP-Fail} simply by sticking to the above definition:
\begin{codebox}
\(\proc{KMP-Fail} (\KMPsupplyName_{x}, i)\)\\
\li \(\proc{KMP-Fail} \gets 1 + \KMPsupplyName_{x}[i-1]\)\RComment \(r_{x}[i] \gets 1 + \KMPsupplyName_{x}[i-1]\)
\end{codebox}
But this is not efficient: it would be better to precompute
and store all the values of \proc{KMP-Fail} in an array.

\end{frame}

% ------------------------------------------------------------------------
%
\begin{frame}
\frametitle{Knuth-Morris-Pratt/Better failure function (cont)}

Let us recall the definition of \(\KMPsupplyName_{x}\). Let \(1
\leqslant i \leqslant \len{x}\) and
\begin{align*}
\KMPsupply{x}{}{0} &= -1\\
  \KMPsupply{x}{}{i}
&= \left\lbrace
  \begin{aligned}
    &  \MPsupply{x}{}{i}
    && \text{if} \; i = \len{x} \; \text{or} 
       \; x [\MPsupply{x}{}{i} + 1] \neq x [i + 1]\\
    &  \KMPsupply{x}{}{\MPsupply{x}{}{i}}
    && \text{otherwise}
  \end{aligned}
  \right.
\end{align*}
Then, for \(1 \leqslant i \leqslant \len{x}-1\)
\begin{align*}
   \rX{x}{}{1} 
&= 1 + \KMPsupply{x}{}{0} = 0\\
   \rX{x}{}{1+i}
&= 1 + \KMPsupply{x}{}{i}
 = \left\lbrace
   \begin{aligned}
   & 1 + \MPsupply{x}{}{i}
   && \text{if} \; x [\MPsupply{x}{}{i} + 1] \neq x [i + 1]\\
   & 1 + \KMPsupply{x}{}{\MPsupply{x}{}{i}}
   && \text{otherwise}
   \end{aligned}
   \right.
\end{align*}

\end{frame}

% ------------------------------------------------------------------------
%
\begin{frame}
\frametitle{Knuth-Morris-Pratt/Better failure function (cont)}

This can be slightly simplified into
\begin{align*}
   \rX{x}{}{1} 
&= 0\\
   \rX{x}{}{1+i}
&= \left\lbrace
   \begin{aligned}
   & \rX{x}{}{1 + \MPsupply{x}{}{i}}
   && \text{if} \; x [1 + \MPsupply{x}{}{i}] = x [1 + i]\\
   & 1 + \MPsupply{x}{}{i}
   && \text{otherwise}
   \end{aligned}
   \right.
\end{align*}
where for \(1 \leqslant i \leqslant \len{x}-1\).

\bigskip

Now, we need to express \(\MPsupplyName_{x}\) in terms of \(r_{x}\)
instead of \(\KMPsupplyName_{x}\). 

\end{frame}

% ------------------------------------------------------------------------
%
\begin{frame}
\frametitle{Knuth-Morris-Pratt/Better failure function (cont)}

First, let us prove by induction on \(0 \leqslant p\) that
\[
\rX{x}{p}{1+i} = 1 + \KMPsupply{x}{p}{i} \qquad 0 \leqslant i
\leqslant \len{x} - 1
\]
Because 
\[
\rX{x}{0}{1+i} = 1 + \KMPsupply{x}{0}{i}
\Longleftrightarrow
1 + i = 1 + i
\]
and, assuming it is true up to \(p\), we have
\[
  \rX{x}{p+1}{1+i}
= \rX{x}{}{\rX{x}{p}{1+i}} = \rX{x}{}{1 + \KMPsupply{x}{p}{i}}
= 1 + \KMPsupply{x}{}{\KMPsupply{x}{p}{i}}
= 1 + \KMPsupply{x}{p+1}{i}
\]
which is the property at rank \(p+1\).

\end{frame}

% ------------------------------------------------------------------------
%
\begin{frame}
\frametitle{Knuth-Morris-Pratt/Better failure function (cont)}

So, the definition of \(\MPsupplyName_{x}\)
\[
\MPsupply{x}{}{0} = -1
\qquad
\MPsupply{x}{}{1} = 0
\qquad
\MPsupply{x}{}{1+i} = 1 + \KMPsupply{x}{q}{\MPsupply{x}{}{i}}
\qquad 1 \leqslant i \leqslant \len{x} - 1
\]
where \(q\) is the smallest integer such that either
\(x[\KMPsupply{x}{q}{1 + \MPsupply{x}{}{i}}]=x[1+i]\) or
\(1+\KMPsupply{x}{q}{\MPsupply{x}{}{i}} = 0\), is equivalent to
\[
\MPsupply{x}{}{0} = -1
\qquad
\MPsupply{x}{}{1} = 0
\qquad
\MPsupply{x}{}{1+i} = \rX{x}{q}{1 + \MPsupply{x}{}{i}}
\qquad 1 \leqslant i \leqslant \len{x} - 1
\]
where \(q\) is the smallest integer such that
\begin{itemize}

  \item either \(x[\rX{x}{q}{1 + \MPsupply{x}{}{i}}]=x[1+i]\)
  
  \item or \(\rX{x}{q}{1 + \MPsupply{x}{}{i}} = 0\)

\end{itemize}

\end{frame}

% ------------------------------------------------------------------------
%
\begin{frame}
\frametitle{Knuth-Morris-Pratt/Better failure function (cont)}

Or, by changing \(i+1\) into \(i\),
\[
\MPsupply{x}{}{0} = -1
\qquad
\MPsupply{x}{}{1} = 0
\qquad
\MPsupply{x}{}{i} = \rX{x}{q}{1 + \MPsupply{x}{}{i-1}}
\qquad 2 \leqslant i \leqslant \len{x}
\]
where \(q\) is the smallest integer such that
\begin{itemize}

  \item either \(x[\rX{x}{q}{1 + \MPsupply{x}{}{i-1}}]=x[i]\)
  
  \item or \(\rX{x}{q}{1 + \MPsupply{x}{}{i-1}} = 0\)

\end{itemize}

\end{frame}

% ------------------------------------------------------------------------
%
\begin{frame}
\frametitle{Knuth-Morris-Pratt/Better failure function/Imperative}

{\small
\begin{codebox}
\(\proc{KMP-Fail} (x)\)\\
\li \(r_{x}[1] \gets 0\); \(\id{offset} \gets 0\)
    \RComment \(\id{offset} = 1 + \MPsupply{x}{}{1-1}\)
\li \For \(i \gets 1\) \To \(\len{x}-1\) 
\li \Do \Repeat
\li       \(\id{offset} \gets r_{x}[\id{offset}]\)
\li     \Until \(\id{offset} = 0\) \LogOr \(x[\id{offset}] = x[i]\)
\li     \(\id{offset} \gets \id{offset} + 1\)
        \RComment \(\id{offset} = 1 + \MPsupply{x}{}{i}\)
\li     \RComment \(\id{offset} = 1 + \rX{x}{q}{1+\MPsupply{x}{}{i-1}}\)
\li     \If \(x[\id{offset}] = x[1+i]\)
\li     \Then \(r_{x}[i] \gets r_{x}[\id{offset}]\)
              \RComment \(\rX{x}{}{i} = \rX{x}{}{1+\MPsupply{x}{}{i}}\)
\li     \Else \(r_{x}[i] \gets \id{offset}\)
              \RComment \(\rX{x}{}{i} = 1 + \MPsupply{x}{}{i}\)
        \End
    \End
\li \(\proc{KMP-Fail} \gets r_{x}\)
\end{codebox}
}

\end{frame}

% ------------------------------------------------------------------------
%
\begin{frame}
\frametitle{Knuth-Morris-Pratt/The code}

The Knuth-Morris-Pratt algorithm is exactly the Morris-Pratt
algorithm, except that we use a better failure function \(r\) instead
of \(s\):
\begin{codebox}
\(\proc{KMP} (x, t)\)\\
\li \(r \gets \proc{KMP-Fail} (x)\)
\li \(i \gets 1\); \(j \gets 1\)
\li \While \(i \leqslant \len{x}\) \LogAnd \(j \leqslant \len{t}\)
\li \Do \If \(i = 0\) \LogOr \(x[i] = t[j]\)
\li	\Then \(i \gets i + 1\); \(j \gets j + 1\)
\li	\Else \(i \gets r[i]\)
        \End
    \End
\li \If \(\len{x} < i\) 
\li \Then \(\ldots\) \RComment Occurrence of \(x\) in \(t\) at position \(j - \len{x}\).
\li \Else \(\ldots\) \RComment No occurrences. 
    \End
\end{codebox}

\end{frame}

% ------------------------------------------------------------------------
%
\begin{frame}
\frametitle{Knuth-Morris-Pratt/The code (cont)}

If we allow arguments to be functions, we can elegantly factorise the
two text search algorithms into one:
\begin{codebox}
\(\proc{Search} (x,t,f)\)\\
\li \(i \gets 1\); \(j \gets 1\)
\li \While \(i \leqslant \len{x}\) \LogAnd \(j \leqslant \len{t}\)
\li \Do \If \(i = 0\) \LogOr \(x[i] = t[j]\)
\li	\Then \(i \gets i + 1\); \(j \gets j + 1\)
\li	\Else \(i \gets f[i]\)
        \End
    \End
\li \If \(\len{x} < i\) 
\li \Then \(\ldots\) \RComment Occurrence of \(x\) in \(t\) at position \(j - m\).
\li \Else \(\ldots\) \RComment No occurrences. 
    \End
\end{codebox}

\end{frame}

% ------------------------------------------------------------------------
%
\begin{frame}
\frametitle{Knuth-Morris-Pratt/The code (cont)}

Then 
\begin{align*}
   \proc{MP} (x, t) 
&= \proc{Search} (x, t, \proc{MP-Fail})\\
   \proc{KMP} (x, t)
&= \proc{Search} (x, t, \proc{KMP-Fail})
\end{align*}

\end{frame}

% ------------------------------------------------------------------------
%
\begin{frame}
\frametitle{Knuth-Morris-Pratt/Maximum disjoint borders (cont)}

\textbf{Example.} Consider the following table for the search of the
word \(x=\texttt{abacabac}\) in the text
\(t=\texttt{babacacabacaab}\).
\[
\begin{array}{c|cccccccccccccc|}
j & 1 & 2 & 3 & 4 & 5 & 6 & 7 & 8 & 9 & 10 & 11 & 12 & 13 & 14\\
  & \texttt{b} & \texttt{a} & \texttt{b} & \texttt{a} & \texttt{c} 
  & \texttt{a} & \texttt{c} & \texttt{a} & \texttt{b} & \texttt{a} 
  & \texttt{c} & \texttt{a} & \texttt{a} & \texttt{b}\\
\hline
i & 1 & 1 & 2 & 3 & 4 & 5 & 6 & 1 & 2 & 3 & 4 & 5 & 6 & 2 \\
  & 0 &   &   &   &   &   & 1 &   &   &   &   &   & 1 &   \\
  &   &   &   &   &   &   & 0 &   &   &   &   &   &   &
\end{array}
\]

\end{frame}

% ------------------------------------------------------------------------
%
\begin{frame}
\frametitle{Knuth-Morris-Pratt/Example}

\label{kmp_example}

\setlength\arraycolsep{3pt}

\[
\begin{array}{rc|c|c|c|c|c|c|c|c|c|c|c|c|c|c|c|c|c|c|c|}
  & \multicolumn{1}{c}{\scriptstyle 1}
  & \multicolumn{1}{c}{\scriptstyle 2}
  & \multicolumn{1}{c}{\scriptstyle 3}
  & \multicolumn{1}{c}{\scriptstyle 4}
  & \multicolumn{1}{c}{\scriptstyle 5}
  & \multicolumn{1}{c}{\scriptstyle 6}
  & \multicolumn{1}{c}{\scriptstyle 7}
  & \multicolumn{1}{c}{\scriptstyle 8}
  & \multicolumn{1}{c}{\scriptstyle 9}
  & \multicolumn{1}{c}{\scriptstyle 10}
  & \multicolumn{1}{c}{\scriptstyle 11}
  & \multicolumn{1}{c}{\scriptstyle 12}
  & \multicolumn{1}{c}{\scriptstyle 13}
  & \multicolumn{1}{c}{\scriptstyle 14}\\
\cline{2-15}
    \multicolumn{1}{r|}{t:}
  & \multicolumn{1}{>{\columncolor{mygrey}}c}{\textcolor{white}{\texttt{b}}}
  & \texttt{a} 
  & \texttt{b} 
  & \texttt{a} 
  & \texttt{c} 
  & \texttt{a} 
  & \multicolumn{1}{>{\columncolor{mygrey}}c}{\textcolor{white}{\texttt{c}}}
  & \texttt{a} 
  & \texttt{b} 
  & \texttt{a} 
  & \texttt{c} 
  & \texttt{a} 
  & \multicolumn{1}{>{\columncolor{mygrey}}c}{\textcolor{white}{\texttt{a}}}
  & \texttt{b}\\
\cline{2-15}
\\
\cline{2-9}
    \multicolumn{1}{r|}{x:}
  & \multicolumn{1}{>{\columncolor{mygrey}}c}{\textcolor{white}{\texttt{a}}}
  & \texttt{b} 
  & \texttt{a} 
  & \texttt{c} 
  & \texttt{a} 
  & \texttt{b} 
  & \texttt{a} 
  & \texttt{c}\\
\cline{2-10}
  &
  & \texttt{a} 
  & \texttt{b} 
  & \texttt{a} 
  & \texttt{c} 
  & \texttt{a} 
  & \multicolumn{1}{>{\columncolor{mygrey}}c}{\textcolor{white}{\texttt{b}}}
  & \texttt{a} 
  & \texttt{c}\\
\cline{3-15}
    \multicolumn{7}{c|}{}
  & \multicolumn{1}{>{\columncolor{mygrey}}c}{\textcolor{white}{\texttt{a}}}
  & \texttt{b} 
  & \texttt{a} 
  & \texttt{c} 
  & \texttt{a} 
  & \texttt{b} 
  & \texttt{a} 
  & \texttt{c}\\
\cline{8-16}
    \multicolumn{8}{c|}{}
  & \texttt{a} 
  & \texttt{b} 
  & \texttt{a} 
  & \texttt{c} 
  & \texttt{a} 
  & \multicolumn{1}{>{\columncolor{mygrey}}c}{\textcolor{white}{\texttt{b}}}
  & \texttt{a} 
  & \texttt{c}\\
\cline{9-21}
    \multicolumn{13}{c|}{}
  & \texttt{a} 
  & \texttt{b} 
  & \texttt{a} 
  & \texttt{c} 
  & \texttt{a} 
  & \texttt{b} 
  & \texttt{a} 
  & \texttt{c}\\
\cline{14-21}
\end{array}
\]

\end{frame}

% ------------------------------------------------------------------------
%
\begin{frame}
\frametitle{Comparison of Morris-Pratt and Knuth-Morris-Pratts}

For example, here is the table comparing the values of the two failure
functions for the same pattern:
\[
\begin{array}{c|cccccccc|}
 x & \texttt{a} & \texttt{b} & \texttt{a} & \texttt{c} 
 & \texttt{a} & \texttt{b} & \texttt{a} & \texttt{c}\\
\hline
  i
& 1 & 2 & 3 & 4 & 5 & 6 & 7 & 8\\
  s_{x}(i)
& 0 & 1 & 1 & 2 & 1 & 2 & 3 & 4\\
  r_{x}(i)
& 0 & 1 & 0 & 2 & 0 & 1 & 0 & 2\\
\hline
\end{array}
\]

\end{frame}
