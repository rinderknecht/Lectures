\paragraph{Answer 1.} 

  \begin{enumerate}

     \item \(\proc{Mem}: \type{t} \times \type{node} \rightarrow
     \type{bool}\)\\
     This function takes an argument of type \type{t}, i.e. a
       binary tree. Since binary trees can be constructed in two
       different ways, the first step is to write the left-hand sides
       with these constructors:
       \begin{align*}
         \proc{Mem} (\proc{Empty}, e) &= \textbf{?}\\
         \proc{Mem} (\proc{Make} (r, t_1, t_2), e) &= \textbf{?}
       \end{align*}
       The first equation is easy to complete: by definition,
         there is no node in the empty tree. Therefore
       \begin{align*}
         \proc{Mem} (\proc{Empty}, e) &= \proc{False}\\
         \proc{Mem} (\proc{Make} (r, t_1, t_2), e) &= \textbf{?}
       \end{align*}
       In the last equation, we notice that there are two nodes,
         \(r\) and \(e\), so we can compare them for equality and
         split the equation correspondingly:
       \begin{align*}
            \proc{Mem} (\proc{Empty}, e) 
         &= \proc{False}\\
            \proc{Mem} (\proc{Make} (e, t_1, t_2), e) 
         &= \textbf{?}\\
            \proc{Mem} (\proc{Make} (r, t_1, t_2), e) 
         &= \textbf{?} 
         & \text{if} \; r \neq e
       \end{align*}
       We can now complete the second equation because we know,
         by definition of the case, that \(e\) is in \(t\) (it is its
         root):
       \begin{align*}
            \proc{Mem} (\proc{Empty}, e) 
         &= \proc{False}\\
            \proc{Mem} (\proc{Make} (e, t_1, t_2), e) 
         &= \proc{True}\\
            \proc{Mem} (\proc{Make} (r, t_1, t_2), e) 
         &= \textbf{?} 
         & \text{if} \; r \neq e
       \end{align*}
       In the case of the last equation, we know that \(e\) is
         not the root of \(t\). Therefore, we must check whether \(e\)
         is not at some other place in \(t\) by means of recursive
         calls in the left and right subtrees of \(t\):
       \begin{align*}
            \proc{Mem} (\proc{Empty}, e) 
         &= \proc{False}\\
            \proc{Mem} (\proc{Make} (e, t_1, t_2), e) 
         &= \proc{True}\\
            \proc{Mem} (\proc{Make} (r, t_1, t_2), e) 
         &= \proc{Mem} (t_1, e) \, \| \, \proc{Mem} (t_2, e)
         & \text{if} \; r \neq e
       \end{align*}
       Actually, we can simplify the equations by merging back the
       third and second equations:
       \begin{align*}
            \proc{Mem} (\proc{Empty}, e) 
         &= \proc{False}\\
            \proc{Mem} (\proc{Make} (r, t_1, t_2), e) 
         &= (r = e) \, \| \, \proc{Mem} (t_1, e) \, \| \, \proc{Mem} (t_2, e)
       \end{align*}

     \item \(\proc{Min-Depth}: \type{t} \times \type{node} \rightarrow
       \type{int}\)\\
     The first step is to note that \proc{Min-Depth} takes an argument
     of type \type{t}, i.e. a binary tree, hence we can consider the
     two constructors for tree as this argument:
     \begin{align*}
       \proc{Min-Depth} (\proc{Empty}, e) &= \textbf{?}\\
       \proc{Min-Depth} (\proc{Make} (r, t_1, t_2), e) &= \textbf{?}
     \end{align*}
     What is the minimum depth of any node \(e\) in the empty
       tree?  Since, by definition, there is no node in the empty
       tree, there is no depth for any node: it is
       unspecified. Therefore, the first equation is meaningless, and
       we can remove it. Now remains
     \begin{align*}
       \proc{Min-Depth} (\proc{Make} (r, t_1, t_2), e) &= \textbf{?}
     \end{align*}
     There are two nodes, \(r\) and \(e\), in the left-hand side of
     this equation, so we can compare them for equality and split the
     equation accordingly:
     \begin{align*}
          \proc{Min-Depth} (\proc{Make} (e, t_1, t_2), e) 
       &= \textbf{?}\\
          \proc{Min-Depth} (\proc{Make} (r, t_1, t_2), e) 
       &= \textbf{?}
       & \text{if} \; r \neq e
     \end{align*}
     In the first equation, the node we are looking for is the root of
     \(t\). Since the text explaining the signature tells that depth
     of the root is always 0, we just found the right-hand side:
     \begin{align*}
          \proc{Min-Depth} (\proc{Make} (e, t_1, t_2), e) 
       &= 0\\
          \proc{Min-Depth} (\proc{Make} (r, t_1, t_2), e) 
       &= \textbf{?}
       & \text{if} \; r \neq e
     \end{align*}
     In the remaining equation, the node we search, \(e\), is not the
     root of \(t\). Thus we must search the left and right subtrees
     recursively and take the minimum of the two depth \textbf{plus
       one}:
     \begin{align*}
          \proc{Min-Depth} (\proc{Make} (e, t_1, t_2), e) 
       &= 0\\
          \proc{Min-Depth} (\proc{Make} (r, t_1, t_2), e) 
       &= 1 + \proc{Min} (&\proc{Min-Depth} (t_1, e),\\
       && \proc{Min-Depth} (t_2, e))\\
       && \text{if} \; r \neq e
     \end{align*}
     But something is wrong with the last equation... What if \(e\) is
     not in \(t_1\) or not in \(t_2\)? Then the values
     \(\proc{Min-Depth} (t_1, e)\) or \(\proc{Min-Depth} (t_2, e)\)
     would be undefined, but what is the minimum of two undefined
     values? It is preferable then to constrain \(e\) to be in \(t\)
     in order to get a specified value \(\proc{Min-Depth} (t, e)\):
     \begin{align*}
          \proc{Min-Depth} (\proc{Make} (e, t_1, t_2), e) 
       &= 0\\
          \proc{Min-Depth} (\proc{Make} (r, t_1, t_2), e) 
       &= 1 + \proc{Min} (&\proc{Min-Depth} (t_1, e),\\
       && \proc{Min-Depth} (t_2, e))
     \end{align*}
      \hfill{\(\qquad\qquad\qquad\text{if} \; r \neq e \; \text{and}
        \; \proc{Mem} (\proc{Make} (r, t_1, t_2), e)\)}
       
     \item \(\proc{Inv}: \type{t} \rightarrow \type{t}\)\\ The sole
       argument of \proc{Inv} is of type \type{t}, i.e. it is a binary
       tree, so we start by enumerating the tree constructors:
     \begin{align*}
       \proc{Inv} (\proc{Empty}) &= \textbf{?}\\
       \proc{Inv} (\proc{Make} (r, t_1, t_2)) &= \textbf{?}
     \end{align*}
     The first equation specifies the image of the empty tree in a
     mirror (inverse). Since, by definition, there is no node in the
     empty tree, its image can only be itself:
     \begin{align*}
       \proc{Inv} (\proc{Empty}) &= \proc{Empty}\\
       \proc{Inv} (\proc{Make} (r, t_1, t_2)) &= \textbf{?}
     \end{align*}
     Now what about the non-empty tree case? First, we should
     understand that the invariant part is the root: the image of the
     root is at the same place:
     \begin{align*}
       \proc{Inv} (\proc{Empty}) &= \proc{Empty}\\
       \proc{Inv} (\proc{Make} (r, t_1, t_2)) &= \proc{Make} (r,
       \textbf{?}, \textbf{?})
     \end{align*}
     The subtrees have to permute, i.e. we swap them in order to get
     the reverse order \textbf{and} we reverse them also with a
     recursive call:
     \begin{align*}
       \proc{Inv} (\proc{Empty}) &= \proc{Empty}\\
       \proc{Inv} (\proc{Make} (r, t_1, t_2)) &= \proc{Make} (r,
       \proc{Inv} (t_2), \proc{Inv} (t_1))
     \end{align*}
 
     \item \(\proc{Sum}: \proc{Bin-Tree}(\type{int}).\type{t}
       \rightarrow \type{int}\)\\ We note that \proc{Sum} as a unique
       argument of type \proc{Bin-Tree}(\type{int}).\type{t}, i.e. it
       is a binary tree over positive integers. So we start by
       enumerating the tree constructors for this argument:
     \begin{align*}
       \proc{Sum} (\proc{Empty}) &= \textbf{?}\\
       \proc{Sum} (\proc{Make} (n, t_1, t_2)) &= \textbf{?}
     \end{align*}
     The first equation refers to the sum of the nodes of th empty
     tree. Since, by definition, there is no node in the empty tree,
     the sum is therefore undefined. Hence we should remove this
     equation, but... what if the empty tree is part of a non-empty
     tree (like a tree just made of a root)? For instance,
     \(\proc{Sum} (\proc{Make} (5, \proc{Empty}, \proc{Empty})) =
     5\). In this case we expect the sum of the empty tree to be
     \(0\), because it should no modify the whole sum, so \(5\)
     actually comes from \(5 + 0 + 0\). Thus we have to stick with
     \begin{align*}
       \proc{Sum} (\proc{Empty}) &= 0\\
       \proc{Sum} (\proc{Make} (n, t_1, t_2)) &= \textbf{?}
     \end{align*}
     We already know that we have to sum \(n\) to the rest of
       the nodes:
     \begin{align*}
       \proc{Sum} (\proc{Empty}) &= 0\\
       \proc{Sum} (\proc{Make} (n, t_1, t_2)) &= n + \textbf{?}
     \end{align*}
     We have to add now the sum of the left subtree and the sum of the
     right subtree nodes, as recursive calls:
     \begin{align*}
       \proc{Sum} (\proc{Empty}) &= 0\\
       \proc{Sum} (\proc{Make} (n, t_1, t_2)) &= n + \proc{Sum} (t_1)
       + \proc{Sum} (t_2)
     \end{align*}

\end{enumerate}
