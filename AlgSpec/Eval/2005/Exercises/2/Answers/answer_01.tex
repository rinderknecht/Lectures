\paragraph{Answer.} 

\begin{itemize}

  \item \proc{Nth} can be expressed in terms of \proc{Get} as follows:
    \begin{align*}
         \proc{Nth} (a, n) 
      &= \proc{Get} (a,\proc{Lower}(a)+n-1),
      &\text{if} \,\; a \neq \proc{Empty}.
    \end{align*}

  \item \(\proc{Mem} : \type{t} \times \proc{Item}.\type{t} 
    \rightarrow \type{bool}\)\\
    The first argument of \proc{Mem} is an array, so we can start by
    envisaging the three constructors of arrays:
    \begin{align*}
        \proc{Mem} (\proc{Empty}, e)
     &= \textbf{?}\\
        \proc{Mem} (\proc{Create} (x, y), e)
     &= \textbf{?}\\
        \proc{Mem} (\proc{Set} (a, n, e_1), e_2)
     &= \textbf{?}
     & \text{if} \; \proc{Lower} (a) \leqslant n \leqslant
     \proc{Upper} (a)
    \end{align*}
    The first equation refers to an empty array into which we look for
    some element, so this element, as well as any element, is not
    present, for sure, so:
    \begin{align*}
        \proc{Mem} (\proc{Empty}, e)
     &= \proc{False}\\
        \proc{Mem} (\proc{Create} (x, y), e)
     &= \textbf{?}\\
        \proc{Mem} (\proc{Set} (a, n, e_1), e_2)
     &= \textbf{?}
     & \text{if} \; \proc{Lower} (a) \leqslant n \leqslant
     \proc{Upper} (a)
    \end{align*}
    As far as the second equation is concerned, we know that an
    unmodified array contains, by construction, only
    \proc{Item}.\proc{Default} elements. So we must consider two
    cases: \(e = \proc{Default}\) or not:
    \begin{align*}
        \proc{Mem} (\proc{Empty}, e)
     &= \proc{False}\\
        \proc{Mem} (\proc{Create} (x, y), e)
     &= \textbf{?}
     & \text{if} \; e = \proc{Default}\\
        \proc{Mem} (\proc{Create} (x, y), e)
     &= \textbf{?}
     & \text{if} \; e \neq \proc{Default}\\
        \proc{Mem} (\proc{Set} (a, n, e_1), e_2)
     &= \textbf{?}
     & \text{if} \; \proc{Lower} (a) \leqslant n \leqslant
     \proc{Upper} (a)
    \end{align*}
    Or, simply
    \begin{align*}
        \proc{Mem} (\proc{Empty}, e)
     &= \proc{False}\\
        \proc{Mem} (\proc{Create} (x, y), \proc{Default})
     &= \textbf{?}\\
        \proc{Mem} (\proc{Create} (x, y), e)
     &= \textbf{?}
     & \text{if} \; e \neq \proc{Default}\\
        \proc{Mem} (\proc{Set} (a, n, e_1), e_2)
     &= \textbf{?}
     & \text{if} \; \proc{Lower} (a) \leqslant n \leqslant
     \proc{Upper} (a)
    \end{align*}
    Now, the first case is simple:
    \begin{align*}
        \proc{Mem} (\proc{Empty}, e)
     &= \proc{False}\\
        \proc{Mem} (\proc{Create} (x, y), \proc{Default})
     &= \proc{True}\\
        \proc{Mem} (\proc{Create} (x, y), e)
     &= \textbf{?}
     & \text{if} \; e \neq \proc{Default}\\
        \proc{Mem} (\proc{Set} (a, n, e_1), e_2)
     &= \textbf{?}
     & \text{if} \; \proc{Lower} (a) \leqslant n \leqslant
     \proc{Upper} (a)
    \end{align*}
    The second one also:
    \begin{align*}
        \proc{Mem} (\proc{Empty}, e)
     &= \proc{False}\\
        \proc{Mem} (\proc{Create} (x, y), \proc{Default})
     &= \proc{True}\\
        \proc{Mem} (\proc{Create} (x, y), e)
     &= \proc{False}
     & \text{if} \; e \neq \proc{Default}\\
        \proc{Mem} (\proc{Set} (a, n, e_1), e_2)
     &= \textbf{?}
     & \text{if} \; \proc{Lower} (a) \leqslant n \leqslant
     \proc{Upper} (a)
    \end{align*}
    The last equation refers to the case of an array modified at index
    \(n\), in which we search for element \(e_2\). The first idea that
    should occur is to check whether the new element at index \(n\),
    namely \(e_1\), is \(e_2\) or not. Indeed, if they are equal, the
    result should be \proc{True}. So
    \begin{align*}
        \proc{Mem} (\proc{Empty}, e)
     &= \proc{False}\\
        \proc{Mem} (\proc{Create} (x, y), \proc{Default})
     &= \proc{True}\\
        \proc{Mem} (\proc{Create} (x, y), e)
     &= \proc{False}
     & \text{if} \; e \neq \proc{Default}\\
        \proc{Mem} (\proc{Set} (a, n, e_1), e_2)
     &= \textbf{?}
     & \text{if} \; \proc{Lower} (a) \leqslant n \leqslant
       \proc{Upper} (a)\\
     && \text{and} \; e_1 = e_2\\
        \proc{Mem} (\proc{Set} (a, n, e_1), e_2)
     &= \textbf{?}
     & \text{if} \; \proc{Lower} (a) \leqslant n \leqslant
     \proc{Upper} (a)\\
     && \text{and} \; e_1 \neq e_2
    \end{align*}
    Or, simply
    \begin{align*}
        \proc{Mem} (\proc{Empty}, e)
     &= \proc{False}\\
        \proc{Mem} (\proc{Create} (x, y), \proc{Default})
     &= \proc{True}\\
        \proc{Mem} (\proc{Create} (x, y), e)
     &= \proc{False}
     & \text{if} \; e \neq \proc{Default}\\
        \proc{Mem} (\proc{Set} (a, n, e), e)
     &= \textbf{?}
     & \text{if} \; \proc{Lower} (a) \leqslant n \leqslant
     \proc{Upper} (a)\\
        \proc{Mem} (\proc{Set} (a, n, e_1), e_2)
     &= \textbf{?}
     & \text{if} \; \proc{Lower} (a) \leqslant n \leqslant
     \proc{Upper} (a)\\
     && \text{if} \; e_1 \neq e_2
    \end{align*}
    Then
    \begin{align*}
        \proc{Mem} (\proc{Empty}, e)
     &= \proc{False}\\
        \proc{Mem} (\proc{Create} (x, y), \proc{Default})
     &= \proc{True}\\
        \proc{Mem} (\proc{Create} (x, y), e)
     &= \proc{False}
     & \text{if} \; e \neq \proc{Default}\\
        \proc{Mem} (\proc{Set} (a, n, e), e)
     &= \proc{True}
     & \text{if} \; \proc{Lower} (a) \leqslant n \leqslant
     \proc{Upper} (a)\\
        \proc{Mem} (\proc{Set} (a, n, e_1), e_2)
     &= \textbf{?}
     & \text{if} \; \proc{Lower} (a) \leqslant n \leqslant
     \proc{Upper} (a)\\
     && \text{and} \; e_1 \neq e_2\\
    \end{align*}
    The last case refers to a modified array whose new element is
    different from which the one we look for. What happen if we simply
    ignore  (i.e. undo) the last modification and search again? This
    means the right-hand side of the last equation would be
    \(\proc{Mem} (a, e_2)\).

    This would be a problem if \(a\) contains \(e_2\) at the same
    index \(n\), i.e. contains a subterm like \(\proc{Set} 
    (b, n, e_2)\). This is possible because our arrays remember all
    the modifications done since their creation. So, searching \(a\)
    instead of \(\proc{Set} (a, n, e_1)\) means searching in the
    previous modifications of the array \(\proc{Set} (a, n,
    e_1)\). But if \(e_2\) was present at index \(n\) in \(a\), the
    value of \(\proc{Mem} (a, e_2)\) will be \proc{True},
    \emph{despite we know that in the given array, there is no
      \(e_2\) at index \(n\)}. 

    So we must find a way to ignore index \(n\) in the subsequent
    searches. The simplest way is to use an auxiliary function
    \(\oMem: \type{t} \times \proc{Item}.\type{t} \times
    \proc{Stack}(\type{int}).\type{t} \rightarrow \type{bool}\). The
    difference with \proc{Mem} is that \(\oMem\) takes a supplementary
    argument which is a stack of indexes. These indexes are the
    already visited indexes, so they must be ignored in previous
    versions of the array, in order to avoid the problem we mentioned
    in the previous paragraph. 

    Therefore the definition of \proc{Mem} is now done in terms of
    \(\oMem\):
    \[
      \proc{Mem} (a, e) = \oMem (a, e, \proc{Stack}(\type{int}).\proc{Empty})
    \]
    In order to save some space, let us write
    \begin{itemize}
      \item \(n \in s\) instead of
      \(\proc{Stack}(\type{int}).\proc{Mem} (s, n)\) (we do not define
      this membership function on stack here, but is is easy to do);

      \item \(n :: s\) instead of
      \(\proc{Stack}(\type{int}).\proc{Push} (n, s)\)
    \end{itemize}
    Now we can rewrite our equations with \(\oMem\) instead of
    \proc{Mem}:
    \begin{align*}
        \oMem (\proc{Empty}, e, \mathbf{s})
     &= \proc{False}\\
        \oMem (\proc{Create} (x, y), \proc{Default}, \mathbf{s})
     &= \proc{True}\\
        \oMem (\proc{Create} (x, y), e, \mathbf{s})
     &= \proc{False}
        \qquad\qquad\quad \text{if} \; e \neq \proc{Default}\\
        \oMem (\proc{Set} (a, n, e), e, \mathbf{s})
     &= \proc{True}
       \qquad\qquad\qquad\qquad\quad \textbf{if} \; \mathbf{n \not\in s}\\
     & \qquad \text{and} \; \proc{Lower} (a)
        \leqslant n \leqslant \proc{Upper} (a)\\
        \oMem (\proc{Set} (a, n, e_1), e_2, \mathbf{s})
     &= \textbf{?}
      \qquad\qquad\qquad\qquad\qquad \text{if} \; e_1 \neq e_2\\
     & \qquad\quad \text{if} \; \proc{Lower}
     (a) \leqslant n \leqslant \proc{Upper} (a)
    \end{align*}
    We notice that we add a condition \(n \not\in s\) which means that
    the index \(n\) is visited here for the first time, so we can take
    the corresponding contents, \(e\), into account (here the result
    is \proc{True}).
   
    What if \(n \in s\)? Actually, this situation encompasses whether
    \(e_1 = e_2\) or not: in both cases, we must ignore the current
    index \(n\) and recursively explore the other indexes. This is
    achieved by the recursive call \(\oMem (a, e_2, s)\). Technically,
    we just removed the \proc{Set} call of the left-hand side \(\oMem
    (\underline{\proc{Set} (}a\underline{, n, e_1)}, e_2, s)\). We
    must add another equation:
    \begin{align*}
        \oMem (\proc{Empty}, e, s)
     &= \proc{False}\\
        \oMem (\proc{Create} (x, y), \proc{Default}, s)
     &= \proc{True}\\
        \oMem (\proc{Create} (x, y), e, s)
     &= \proc{False}
        \qquad\qquad\quad \text{if} \; e \neq \proc{Default}\\
        \oMem (\proc{Set} (a, n, e), e, s)
     &= \proc{True}
       \qquad\qquad\qquad\qquad\quad \textbf{if} \; \mathbf{n \not\in s}\\
     & \qquad \text{and} \; \proc{Lower} (a)
        \leqslant n \leqslant \proc{Upper} (a)\\
        \oMem (\proc{Set} (a, n, e_1), e_2, s)
     &= \oMem (a, e_2, s)
        \qquad\qquad\quad \;\; \textbf{if} \; \mathbf{n \in s}\\
        \oMem (\proc{Set} (a, n, e_1), e_2, s)
     &= \textbf{?}
      \qquad\qquad\quad \;\; \text{if} \; e_1 \neq e_2 \; \textbf{and}
      \; \mathbf{n \not\in s}\\
     & \qquad\quad \text{if} \; \proc{Lower}
     (a) \leqslant n \leqslant \proc{Upper} (a)
    \end{align*}
    Notice that the condition \(\text{if} \; \proc{Lower} (a)
    \leqslant n \leqslant \proc{Upper} (a)\) has been removed because
    it is a recursive call, so we just can let the non-recursive calls
    check the bounds (see the fourth equation). Note also that we had
    the condition \(n \not\in s\) to the last equation, because we
    just added the case \(n \in s\) in the fourth equation.

    The remaining case is also a recursive call because we know that
    \(e_1 \neq e_2\), i.e. the element at index \(n\) is not the one
    we look for, and it is the first time we visit this index \(n\),
    so we must record it in the stack \(s\) of indexes, so we will not
    consider it again (in the subsequent recursive calls):
    \begin{align*}
        \oMem (\proc{Empty}, e, s)
     &= \proc{False}\\
        \oMem (\proc{Create} (x, y), \proc{Default}, s)
     &= \proc{True}\\
        \oMem (\proc{Create} (x, y), e, s)
     &= \proc{False}
        \qquad\qquad\quad \text{if} \; e \neq \proc{Default}\\
        \oMem (\proc{Set} (a, n, e), e, s)
     &= \proc{True}
       \qquad\qquad\qquad\qquad\quad \text{if} \; n \not\in s\\
     & \qquad \text{and} \; \proc{Lower} (a)
        \leqslant n \leqslant \proc{Upper} (a)\\
        \oMem (\proc{Set} (a, n, e_1), e_2, s)
     &= \oMem (a, e_2, s)
        \qquad\qquad\quad \;\; \text{if} \; n \in s\\
        \oMem (\proc{Set} (a, n, e_1), e_2, s)
     &= \oMem (a, e_2, n :: s) \qquad\quad \text{if} \; e_1 \neq e_2\\
     &  \qquad\qquad\qquad\qquad\qquad\qquad  \text{and} \; n \not\in s
    \end{align*}
  
    \item \(\proc{Inv} : \type{t} \rightarrow \type{t}\)\\ 
    The sole argument of \proc{Inv} is an array, so we can simply
    start by enumerating all the array constructors:
    \begin{align*}
       \proc{Inv} (\proc{Empty}) 
    &= \textbf{?}\\
       \proc{Inv} (\proc{Create} (x, y))
    &= \textbf{?}\\
       \proc{Inv} (\proc{Set} (a, n, e))
    &= \textbf{?}
    & \text{if} \; \proc{Lower} (a) \leqslant n \leqslant
      \proc{Upper} (a)
    \end{align*}
    The first equation corresponds to the inversion of an empty
    array: it is pretty clear that it must be the empty array too:
    \begin{align*}
       \proc{Inv} (\proc{Empty}) 
    &= \proc{Empty}\\
       \proc{Inv} (\proc{Create} (x, y))
    &= \textbf{?}\\
       \proc{Inv} (\proc{Set} (a, n, e))
    &= \textbf{?}
    & \text{if} \; \proc{Lower} (a) \leqslant n \leqslant
      \proc{Upper} (a)
    \end{align*}
    The second equation refers to the case of an unmodified array to
    be inverted. Since, by construction, all elements of an unmodified
    array are \proc{Item}.\proc{Default}, the inverted array is the
    same as the original one, just as for the empty array. So
    \begin{align*}
       \proc{Inv} (\proc{Empty}) 
    &= \proc{Empty}\\
       \proc{Inv} (\proc{Create} (x, y))
    &= \proc{Create} (x, y)\\
       \proc{Inv} (\proc{Set} (a, n, e))
    &= \textbf{?}
    & \text{if} \; \proc{Lower} (a) \leqslant n \leqslant
      \proc{Upper} (a)
    \end{align*}
    The last equation refers to the case of an array modified at index
    \(n\). The inverted array has thus to be modified symmetrically,
    i.e. it must have the same modification but on the symmetric index
    (starting from the end of the array instead of the beginning). The
    symmetric index of \(n\) in array \(a\) is 
    \[\proc{Upper} (a) - n + \proc{Lower} (a)\]
    (Note that modifiying an element does not change the bounds.) So
    we have to do the modification
    \[\proc{Set} (\textbf{?}, \proc{Upper} (a) - n + \proc{Lower} (a), e)\]
    But what array do we must modify? It cannot be \(a\) because we
    have to invert the other elements in \(a\) before applying the
    last symmetric modification. So it must be \(\proc{Inv} (a)\). As
    a conclustion, we get
    \begin{align*}
       \proc{Inv} (\proc{Empty}) 
    &= \proc{Empty}\\
       \proc{Inv} (\proc{Create} (x, y))
    &= \proc{Create} (x, y)\\
       \proc{Inv} (\proc{Set} (a, n, e))
    &= \proc{Set} (\proc{Inv} (a), \proc{Upper} (a) - n + \proc{Lower}
       (a), e)\\
    & \qquad \qquad \qquad \qquad \text{if} \; \proc{Lower} (a)
      \leqslant n \leqslant \proc{Upper} (a) &
      \end{align*}
    Actually, we could simplify a little bit these equations by
    noting that if the array is not modified, then the inverted array
    is invariant. Formally:
    \begin{align*}
       \proc{Inv} (a) 
    &= a \qquad\qquad\qquad\qquad\qquad\qquad\quad\;\;
         \text{if} \; a \neq \proc{Set} (b, n, e)\\
       \proc{Inv} (\proc{Set} (a, n, e))
    &= \proc{Set} (\proc{Inv} (a), \proc{Upper} (a) - n + \proc{Lower}
       (a), e)\\
    & \qquad \qquad \qquad \qquad\qquad \text{if} \; \proc{Lower} (a)
      \leqslant n \leqslant \proc{Upper} (a) &
    \end{align*}

    \item \(\proc{Undo} : \type{t} \times \type{int} \rightarrow
    \type{t}\)\\
    Since the first argument of \proc{Undo} is an array and since
    there are three array constructors, we start by enumerating these
    cases:
    \begin{align*}
       \proc{Undo} (\proc{Empty}, n)
    &= \textbf{?}\\
       \proc{Undo} (\proc{Create} (x, y), n)
    &= \textbf{?}\\
       \proc{Undo} (\proc{Set} (a, p, e), n)
    &= \textbf{?}
    & \text{if} \; \proc{Lower} (a) \leqslant p \leqslant \proc{Upper} (a)
    \end{align*}
    The text says that if the array is empty or unmodified, the result
    of undoing the last modification is simply the original array:
    \begin{align*}
       \proc{Undo} (\proc{Empty}, n)
    &= \proc{Empty}\\
       \proc{Undo} (\proc{Create} (x, y), n)
    &= \proc{Create} (x, y)\\
       \proc{Undo} (\proc{Set} (a, p, e), n)
    &= \textbf{?}
    & \text{if} \; \proc{Lower} (a) \leqslant p \leqslant \proc{Upper} (a)
    \end{align*}
    The remaining equation refers to the case of an array whose last
    modification has been made at index \(p\). Since we want to undo
    the last modification at index \(n\), we must check whether \(p =
    n\) or not:
    \begin{align*}
       \proc{Undo} (\proc{Empty}, n)
    &= \proc{Empty}\\
       \proc{Undo} (\proc{Create} (x, y), n)
    &= \proc{Create} (x, y)\\
       \proc{Undo} (\proc{Set} (a, p, e), n)
    &= \textbf{?}
    & \text{if} \; p = n\\
    && \text{and} \; \proc{Lower} (a) \leqslant p
      \leqslant \proc{Upper} (a)\\
       \proc{Undo} (\proc{Set} (a, p, e), n)
    &= \textbf{?}
    & \text{if} \; p \neq n\\
    && \text{and} \; \proc{Lower} (a) \leqslant p
      \leqslant \proc{Upper} (a)
    \end{align*}
    Or, simply
    \begin{align*}
       \proc{Undo} (\proc{Empty}, n)
    &= \proc{Empty}\\
       \proc{Undo} (\proc{Create} (x, y), n)
    &= \proc{Create} (x, y)\\
       \proc{Undo} (\proc{Set} (a, n, e), n)
    &= \textbf{?}
    & \text{if} \; \proc{Lower} (a) \leqslant n
      \leqslant \proc{Upper} (a)\\
       \proc{Undo} (\proc{Set} (a, p, e), n)
    &= \textbf{?}
    & \text{if} \; p \neq n\\
    && \text{and} \; \proc{Lower} (a) \leqslant p
      \leqslant \proc{Upper} (a)
    \end{align*}
    If the last modification was done at the index we want to undo,
    the result is the array without the call to \proc{Set}:
    \begin{align*}
       \proc{Undo} (\proc{Empty}, n)
    &= \proc{Empty}\\
       \proc{Undo} (\proc{Create} (x, y), n)
    &= \proc{Create} (x, y)\\
       \proc{Undo} (\proc{Set} (a, n, e), n)
    &= a
    & \text{if} \; \proc{Lower} (a) \leqslant n
      \leqslant \proc{Upper} (a)\\
       \proc{Undo} (\proc{Set} (a, p, e), n)
    &= \textbf{?}
    & \text{if} \; p \neq n\\
    && \text{and} \; \proc{Lower} (a) \leqslant p
      \leqslant \proc{Upper} (a)
    \end{align*}
    Now, the last equation refers to the case where the last
    modification was done on a different index than the one we want to
    undo. Therefore, we must keep this modification and undo on the
    array \emph{without} this modification:
    \begin{align*}
       \proc{Undo} (\proc{Empty}, n)
    &= \proc{Empty}\\
       \proc{Undo} (\proc{Create} (x, y), n)
    &= \proc{Create} (x, y)\\
       \proc{Undo} (\proc{Set} (a, n, e), n)
    &= a
     \qquad\qquad\quad \text{if} \; \proc{Lower} (a) \leqslant n
      \leqslant \proc{Upper} (a)\\
       \proc{Undo} (\proc{Set} (a, p, e), n)
    &= \proc{Set} (\proc{Undo} (a, n), p, e)
     \qquad\qquad\qquad \text{if} \; p \neq n\\
    & \qquad \qquad \qquad \text{and} \; \proc{Lower} (a)
      \leqslant p \leqslant \proc{Upper} (a)
    \end{align*}
    As a final note, we can simplify a little bit more the equations
    by grouping the two first ones:
    \begin{align*}
       \proc{Undo} (a, n)
    &= a
    \qquad\qquad\qquad\qquad\qquad\qquad\quad \text{if} \; a
    \neq \proc{Set} (b, p, e)\\ 
       \proc{Undo} (\proc{Set} (a, n, e), n)
    &= a
      \qquad\qquad\qquad\quad \text{if} \; \proc{Lower} (a) \leqslant n
      \leqslant \proc{Upper} (a)\\
       \proc{Undo} (\proc{Set} (a, p, e), n)
    &= \proc{Set} (\proc{Undo} (a, n), p, e)
    \qquad \qquad \qquad \qquad \text{if} \; p \neq n\\
    & \qquad \qquad \qquad \qquad \text{and} \; \proc{Lower} (a)
      \leqslant p \leqslant \proc{Upper} (a)
    \end{align*}

\end{itemize}
