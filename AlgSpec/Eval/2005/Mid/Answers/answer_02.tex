\paragraph{Answer 2.} First let us consider the equations defining function
  \proc{Nth}. The signature tells us that this function takes two
  arguments: the first one is a stack and the second one is an integer
  corresponding to the position in the stack.

  We understand that the item at the top of the stack, if any, is the
  first one. This means that \(\proc{Nth} (s, 0)\) has no
  meaning, hence no equation about it (the equations define, that is
  they are the meaning of a function).

  The signature says also that the sought item must exist in the
  stack, so \(\proc{Nth} (\proc{Empty}, n)\) for any \(n\),
  where \proc{Empty} denotes the empty stack, has no meaning, thus no
  equation.

  What are the remaining cases? We still have to consider non empty
  stacks, i.e. of the kind \(\proc{Push} (e, s)\), and
  positions greater than 1, distinguishing the case 1:
  \begin{align}
     \proc{Nth} (\proc{Push} (e, s), 1)
  &= \textbf{?} \label{Nth_1}\\
     \proc{Nth} (\proc{Push} (e, s), n)
  &= \textbf{?} & \text{where} \;n > 1 \label{Nth_2}
  \end{align}
  Equation~\ref{Nth_1} is already answered: the top item on the stack
  is the fist one. So
  \begin{align*}
     \proc{Nth} (\proc{Push} (e, s), 1)
  &= e\\
     \proc{Nth} (\proc{Push} (e, s), n)
  &= \textbf{?} & \text{where} \;n > 1
  \end{align*}
  In order to complete equation~\ref{Nth_2}, a good picture is very
  useful. Since \(n > 1\), the item we look for is not \(e\). So
  we must look in \(s\). But the top item of \(s\) is the
  \emph{second} of \(\proc{Push} (e, s)\), not the first
  one. So the item we look for is at position \(n-1\) in \(s\),
  whilst being at position \(n\) in \(\proc{Push} (e,
  s)\). Therefore
  \begin{align*}
     \proc{Nth} (\proc{Push} (e, s), 1)
  &= e\\
     \proc{Nth} (\proc{Push} (e, s), n)
  &= \proc{Nth} (s, n-1) & \text{where} \;n > 1
  \end{align*}

  Let us consider the equations for equations defining function
  \proc{Flatten}. Here again, a good picture is very helpful. The
  signatures informs us that this function takes only one argument
  which is a stack of stack of some items. It is convenient to imagine
  a non empty stack as a stack of boxes. In this case, these boxes
  themselves contain stacks of smaller boxes. The effect of
  \proc{Flatten} is to take all these smaller stacks and to append all
  of them in a single stack.

  Since stacks can only be constructed using two constructors, let us
  consider two equations, one about empty stacks and the other about
  non-empty ones:
  \begin{align}
     \proc{Flatten} (\proc{Empty}) 
  &= \textbf{?} \label{Flatten_1}\\
     \proc{Flatten} (\proc{Push} (e, s))
  &= \textbf{?} \label{Flatten_2}
  \end{align}
  Equation~\ref{Flatten_1} is simple: if there no stack then there are
  no items at all.
  \begin{align*}
     \proc{Flatten} (\proc{Empty}) 
  &= \proc{Empty}\\
     \proc{Flatten} (\proc{Push} (e, s))
  &= \textbf{?}
  \end{align*}
  In equation~\ref{Flatten_2}, item \(e\) is a stack also (the
  smaller stacked boxes) and we have to put it on top of the
  flattening of \(s\). This is achieved by use of \proc{Append}:
  \begin{align*}
     \proc{Flatten} (\proc{Empty}) 
  &= \proc{Empty}\\
     \proc{Flatten} (\proc{Push} (e, s))
  &= \proc{Append} (e, \proc{Flatten} (s))
  \end{align*}
  
  Finally let us consider the equations defining function
  \proc{Exists}. The signature says that this function takes two
  arguments: the first one is a predicate over the type \type{item}
  (\(\type{item} \rightarrow \type{boolean}\))
  and the second one is a stack containing items of type \type{item}
  (\proc{Stack}(\type{item}).\type{t}).

  The purpose of the function call \(\proc{Exists} (f, s)\)
  is to tell if there exists an item in the stack \(s\) which
  satisfy a given property \(f\), or not.

  The first step is to deconstruct a general stack into two kinds:
  empty and non-empty:
  \begin{align}
      \proc{Exists} (f, \proc{Empty})
   &= \textbf{?} \label{Exists_1}\\
      \proc{Exists} (f, \proc{Push} (e, s))
   &= \textbf{?} \label{Exists_2}
  \end{align}
  Equation~\ref{Exists_1} is straightforward to complete: by
  construction, there is no item in the empty stack:
  \begin{align*}
      \proc{Exists} (f, \proc{Empty})
   &= \kw{false}\\
      \proc{Exists} (f, \proc{Push} (e, s))
   &= \textbf{?}
  \end{align*}
  Now, we have to consider two more cases: either \(f(e)\)
  (is \kw{true}) or \(\neg{f (e)}\) (is
  \kw{true})\footnote{Expression \(\neg{x}\) is the negation of
    \(x\).}:
  \begin{align*}
      \proc{Exists} (f, \proc{Empty})
   &= \kw{false}\\
      \proc{Exists} (f, \proc{Push} (e, s))
   &= \textbf{?} & \text{if} \;f(e)\\
      \proc{Exists} (f, \proc{Push} (e, s))
   &= \textbf{?} & \text{if} \;\neg{f(e)}
  \end{align*}
  If \(f(e)\) then we found the kind of item we were looking
  for:
  \begin{align*}
      \proc{Exists} (f, \proc{Empty})
   &= \kw{false}\\
      \proc{Exists} (f, \proc{Push} (e, s))
   &= \kw{true} & \text{if} \;f(e)\\
      \proc{Exists} (f, \proc{Push} (e, s))
   &= \textbf{?} & \text{if} \;\neg{f(e)}
  \end{align*}
  In the remaining equation, we know that \(e\) does not satisfy
  \(f\), hence we have to check if there is another candidate further
  down in \(s\):
  \begin{align*}
      \proc{Exists} (f, \proc{Empty})
   &= \kw{false}\\
      \proc{Exists} (f, \proc{Push} (e, s))
   &= \kw{true} & \text{if} \;f(e)\\
      \proc{Exists} (f, \proc{Push} (e, s))
   &= \proc{Exists} (f, s) & \text{if} \;\neg{f(e)}
  \end{align*}  

