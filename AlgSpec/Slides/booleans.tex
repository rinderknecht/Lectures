%%-*-latex-*-

% ------------------------------------------------------------------------
%
\begin{frame}
\frametitle{Booleans/Signature}

Let us start this section about search algorithms by presenting how to
specify data structures without relying on a specific programming
language. The method we introduce here is usually called
\textbf{algebraic specification}.

\bigskip

Consider one of the simplest data structure you can imagine: the
booleans. Let us call the following specification \proc{Bool}. Here is
its \textbf{signature}:

\begin{itemize}

  \item \textbf{Defined types}

  \begin{itemize}

    \item The type of the booleans is \type{t}.

  \end{itemize}

  \item \textbf{Constructors}

  \begin{itemize}

    \item \(\proc{True} : \type{t}\)

    \item \(\proc{False} : \type{t}\)

  \end{itemize}

\end{itemize}

\end{frame}

% ------------------------------------------------------------------------
%
\begin{frame}
\frametitle{Booleans/Signature (continued)}

\proc{True} and \proc{False} are called constructors because they
allow the construction of \textbf{values} of type \type{t},
i.e. booleans. This is why we write ``: \type{t}'' after these
constructors.

\bigskip

For more excitement, let us add some usual \textbf{functions} whose
arguments are booleans:
\begin{itemize}

  \item \(\proc{Not} : \type{t} \rightarrow \type{t}\)\\ Expression
    \(\proc{Not} (\id{b})\) is the negation of \id{b};

  \item \(\proc{And} : \type{t} \times \type{t} \rightarrow
    \type{t}\)\\ Expression \(\proc{And} (\id{b_1}, \id{b_2})\) is
    the conjunction of \id{b_1} and \id{b_2};

  \item \(\proc{Or} : \type{t} \times \type{t} \rightarrow
    \type{t}\)\\ Expression \(\proc{Or} (\id{b_1}, \id{b_2})\) is
    the disjunction of \id{b_1} and \id{b_2}.

\end{itemize}

\end{frame}

% ------------------------------------------------------------------------
%
\begin{frame}
\frametitle{Booleans/Equations}

We can give mode information about the previous functions by means of
\textbf{equations} (or \textbf{axioms}). A possible set of equations
matching the signature is
\begin{align*}
\proc{Not} (\proc{True}) &= \proc{False}\\
\proc{Not} (\proc{False}) &= \proc{True}\\
\proc{And} (\proc{True}, \proc{True}) &= \proc{True}\\
\proc{And} (\proc{True}, \proc{False}) &= \proc{False}\\
\proc{And} (\proc{False}, \proc{True}) &= \proc{False}\\
\proc{And} (\proc{False}, \proc{False}) &= \proc{False}\\
   \forall \id{b_1}, \id{b_2} \quad \proc{Or} (\id{b_1}, \id{b_2}) 
&= \proc{Not} (\proc{And} (\proc{Not} (\id{b_1}), \proc{Not} (\id{b_2})))
\end{align*}
The signature and the equations make a \textbf{specification}.

\end{frame}

% ------------------------------------------------------------------------
%
\begin{frame}
\frametitle{Booleans/Equations (cont)}

We note something interesting in the last equation: it contains
\textbf{variables}, here \id{b_1} and \id{b_2}. These variables must
be of type \type{t} because they are arguments of function \proc{Or},
whose type, as given by the signature, is \(\type{t} \times \type{t}
\rightarrow \type{t}\).

\bigskip

It is very important to notice also that these variables are bound to
a \textbf{universal quantifier}, \textbf{\(\forall\)}, at the
beginning of the equation. This means, in particular, that we can
rename these variables because they are just names which are local to
the equation. For instance
\[
  \forall \id{u}, \id{v} \quad \proc{Or} (\id{u}, \id{v}) 
= \proc{Not} (\proc{And} (\proc{Not} (\id{u}), \proc{Not} (\id{v})))
\]
would be equivalent.

\end{frame}

% ------------------------------------------------------------------------
%
\begin{frame}
\frametitle{Booleans/Simplifying the equations}

We can simplify these equations before going on. 

\bigskip

First we can omit the quantifiers (but remember they are implicitly
present).

\bigskip

Second we remark that it suffices that one argument of \proc{And} is
\proc{False} to make the call equal to \proc{False}. In other words
\begin{align*}
\proc{Not} (\proc{True}) &= \proc{False}\\
\proc{Not} (\proc{False}) &= \proc{True}\\
\proc{And} (\proc{True}, \proc{True}) &= \proc{True}\\
\proc{And} (\id{x}, \proc{False}) &= \proc{False}\\
\proc{And} (\proc{False}, \id{x}) &= \proc{False}\\
\proc{Or} (\id{b_1}, \id{b_2}) &= \proc{Not} (\proc{And} (\proc{Not}
(\id{b_1}), \proc{Not} (\id{b_2})))
\end{align*}

\end{frame}

% ------------------------------------------------------------------------
%
\begin{frame}
\frametitle{Booleans/Simplifying the equations (cont)}

In order to be 100\% confident, we must check whether the set of new
equations is \textbf{equivalent} to the first set.
\[\small
\left\{
\begin{aligned}
\proc{And} (\proc{True}, \proc{False}) &= \proc{False}\\
\proc{And} (\proc{False}, \proc{True}) &= \proc{False}\\
\proc{And} (\proc{False}, \proc{False}) &= \proc{False}
\end{aligned}
\right.
\stackrel{?}{\Longleftrightarrow}
\left\{
\begin{aligned}
\forall \id{x} \quad \proc{And} (\id{x}, \proc{False}) &= \proc{False}\\
\forall \id{x} \quad \proc{And} (\proc{False}, \id{x}) &= \proc{False}
\end{aligned}
\right.
\]
The new system is equivalent to
\(\small
\left\{
\begin{aligned}
\proc{And} (\proc{True}, \proc{False}) &= \proc{False}\\
\proc{And} (\proc{False}, \proc{False}) &= \proc{False}\\
\proc{And} (\proc{False}, \proc{True}) &= \proc{False}\\
\proc{And} (\proc{False}, \proc{False}) &= \proc{False}
\end{aligned}
\right.
\)

So the answer is yes.

\end{frame}
