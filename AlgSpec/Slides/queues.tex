%%-*-latex-*-

% -----------------------------------------------------------------------
%
\begin{frame}
\frametitle{Queues/Signature}

There is another common and useful linear data structure call
\textbf{queue}. 

\bigskip

As the stack, it is fairly intuitive, since we experience the concept
when we are waiting at some place to get some goods or service.

\bigskip

Let us call \(\proc{Queue} (\type{item})\) the specification of a
queue over elements of the \type{item} type.
\begin{itemize}

  \item \textbf{Parameter types}

  \begin{itemize}

    \item The type \type{item} of the elements in the queue.

  \end{itemize}

  \item \textbf{Defined types}

   \begin{itemize}

     \item The type of the queue is \type{t}.

   \end{itemize}

\end{itemize}

\end{frame}

% -----------------------------------------------------------------------
%
\begin{frame}
\frametitle{Queues/Constructors and other functions}

\begin{itemize}

  \item \textbf{Constructors}

  \begin{itemize}

    \item \(\proc{Empty} : \type{t}\)\\
    Expression \proc{Empty} represents the empty queue.

    \item \(\proc{Enqueue} : \type{item} \times \type{t}
    \rightarrow \type{t}\)\\
    Expression \(\proc{Enqueue} (\id{e}, \id{q})\) denotes the queue
    \id{q} with element \id{e} added at the end.

  \end{itemize}

  \item \textbf{Other functions}

  \begin{itemize}

    \item \(\proc{Dequeue} : \type{t} \rightarrow 
    \type{t} \times \type{item}\)\\
    Expression \(\proc{Dequeue} (\id{q})\) denotes the pair made of
    the \emph{first} element of \id{q} and the remaining queue. The
    queue \id{q} must not be empty.

  \end{itemize}

\end{itemize}

\end{frame}

%------------------------------------------------------------------------
%
\begin{frame}
\frametitle{Queues/Equations}

\begin{align*}
\proc{Dequeue} (\proc{Enqueue} (\id{e}, \proc{Empty}))
&= (\proc{Empty}, \id{e})\\
\proc{Dequeue} (\textbf{\proc{Enqueue} (\id{e}, \id{q})})
&= (\proc{Enqueue} (\id{e}, \id{q_1}), \id{e'})\\
\text{where} \quad (\id{q_1}, \id{e'}) &= \proc{Dequeue}
  (\textbf{\id{q}})\\
\text{and} \quad \id{q} & \neq \proc{Empty}
\end{align*}
They are easy to orient since \id{q} is a proper subterm of
\(\proc{Enqueue} (\id{e}, \id{q})\):
\begin{align*}
  \proc{Dequeue} (\proc{Enqueue} (\id{e}, \proc{Empty}))
&\rightarrow (\proc{Empty}, \id{e}) & \\
  \proc{Dequeue} (\proc{Enqueue} (\id{e}, \id{q}))
&\rightarrow (\proc{Enqueue} (\id{e}, \id{q_1}), \id{e'})
\end{align*}
where \(\id{q} \neq \proc{Empty}\) and where
\(\proc{Dequeue} (\id{q}) \rightarrow (\id{q_1}, \id{e'})\).

Note that we can remove the condition by replacing \id{q} by
\(\proc{Enqueue} (\dots)\).

\end{frame}
