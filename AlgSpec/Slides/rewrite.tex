%%-*-latex-*-

% ------------------------------------------------------------------------
%
\begin{frame}
\frametitle{From specifications to algorithms}

Now, how do we go from a specification to an \textbf{algorithm}? 

\bigskip

By algorithm, we mean an \textbf{operational description} of the
specification, i.e., a series of explicit computations that respect the
equations and yield the expected result.

\bigskip

A specification describes abstract properties (using notations like
\(\forall\), \((x,y)\), \(\varnothing\), \(\sum\) etc. and leaving out
explicit data structure definitions) and the algorithm is a calculus
schema (this introduces an abstract computer model, including time).

\end{frame}

% ------------------------------------------------------------------------
%
\begin{frame}
\frametitle{From specifications to algorithms (cont)}

One says that an algorithm is \textbf{correct with respect to its
specification} when all the results of the algorithm satisfy the
specification, i.e., there is no contradiction when the results are
substituted into the equations of the specification.

\bigskip

Correctness is always a relative concept (a property of the algorithm
in regard to its specification).

\end{frame}

% ------------------------------------------------------------------------
%
\begin{frame}
\frametitle{From specifications to algorithms (cont)}

So, an algorithm is more detailed than a specification. But how much
more?

\bigskip

Contrary to algorithms, \textbf{programs} depend on programming
languages, as we mentioned before. Thus algorithms can be considered
as a useful step from specification to programs, as a refinement
step.

\end{frame}

% ------------------------------------------------------------------------
%
\begin{frame}
\frametitle{From specifications to algorithms (cont)}

Algorithms can be implemented using different programming languages
(featuring objects, or pointers, or exceptions etc.). The more we want
to be free of using our favorite programming language, the more the
language for expressing algorithms should be abstract.

\bigskip

However, this language may assume an abstract model of the computer
which can be quite different from the one assumed by the chosen
programming language. On the contrary, sometimes they are the same
(this is the case of \Prolog).

\end{frame}

% ------------------------------------------------------------------------
%
\begin{frame}
\frametitle{From specifications to algorithms (cont)}

Coming back to our question: how do we go from an algebraic
specification (which, by definition, describes formally a concept) to
an algorithm (which describes formally a computation)?

\bigskip

One way is to rely on the two kinds of equations we have, recursive
and non-recursive. In mathematics, the integer sequence we give
page~\pageref{sequence} can be transformed into a functional
definition, i.e., \(U_{n}\) is expressed only in terms of \(n\), \(a\)
and \(b\), by forming \(U_{n+1} - U_{n} = b\) and summing both sides:
\(\sum_{n=0}^{p}{(U_{n+1} - U_{n})} = \sum_{n=0}^{p}{b}
\Leftrightarrow U_{p+1} - U_{0} = (p+1) b\)\\ \(\Leftrightarrow
U_{p+1} - a = (p+1) b \Leftrightarrow U_{p} = a + pb \Leftrightarrow
U_{n} = a + nb\).

\bigskip

But this is an \emph{ad hoc} technique (e.g., we rely on properties of
the integer numbers, as addition), which cannot be applied to our
algebraic specifications.

\end{frame}

% ------------------------------------------------------------------------
%
\begin{frame}
\frametitle{From specifications to algorithms (cont)}

The general approach consists in finding a \textbf{computation scheme}
instead of relying on a \textbf{reasoning}, as we did for the
sequence. 

\bigskip

This is achieved by looking at the equations in the specification and
\textbf{orienting them}. 

\bigskip

This means that an equation is then considered as a \textbf{rewriting
step}, from the left side to the right side, or the reverse way.

\end{frame}

% ------------------------------------------------------------------------
%
\begin{frame}
\frametitle{From specifications to algorithms (cont)}

Assume we have an equation \(A = B\), where \(A\) and \(B\) are
expressions. 

\bigskip

The way to pass from this equality to a computation is to orient the
sides as a rewriting step, i.e., wherever we find an occurrence of the
left side, we replace it by the right side.

\bigskip

For instance, if we set \(A \rightarrow B\), then it means that
wherever we find a \(A\), we can write instead a \(B\).

\bigskip

But, in theory, \(A = B\) is equivalent to \(A \rightarrow B\) and \(B
\rightarrow A\). But if we keep both (symmetric) rewriting steps, we
get a \textbf{non-terminating rewriting system}, as demonstrated by
the following infinite chain: \(A \rightarrow B \rightarrow A
\rightarrow B \rightarrow \dots\)

\end{frame}

% ------------------------------------------------------------------------
%
\begin{frame}
\frametitle{From specifications to algorithms (cont)}

\emph{A rewriting system terminates if it stops on a value.}

\bigskip

If we have a chain \(X_1 \rightarrow X_2 \rightarrow \dots \rightarrow
X_n\) and there is no way to rewrite \(X_n\), i.e., \(X_n\) does not
appear in any part of a left-hand side, then if \(X_n\) is a value,
the chain is terminating.

\bigskip

We want a rewriting system that terminates on a \textbf{value}.

\bigskip

A value is made of constructors only, whilst an \textbf{expression} is
made of constructors and other functions. For example,

\begin{itemize}
 
  \item \(\proc{True}\) is a value;

  \item \(\proc{Or} (\proc{True}, \proc{Not} (\proc{False}))\) is an
  expression.

\end{itemize}

\end{frame}

% ------------------------------------------------------------------------
%
\begin{frame}
\frametitle{From specifications to algorithms (cont)}

Now, how can we be sure we do not lose some property by just allowing
one orientation for each equality, as just keeping \(A \leftarrow
B\) when \(A = B\)?

\bigskip

This problem is a \textbf{completeness problem} and is very difficult
to tackle,\footnote{Refer to the Knuth-Bendix completion
  semi-algorithm.} therefore we will not discuss it here.

\bigskip

So, how should we orient our equations?

\end{frame}

% ------------------------------------------------------------------------
%
\begin{frame}
\frametitle{Booleans/Orienting the equations}

Since we want to use an equation to compute the function it
characterises, if one side of an equation contains an occurrence of
the function call and the other not, we orient from the former side to
the latter:
\begin{align*}
\proc{Not} (\proc{True}) &\rightarrow_1 \proc{False}\\
\proc{Not} (\proc{False}) &\rightarrow_2 \proc{True}\\
\proc{And} (\proc{True}, \proc{True}) &\rightarrow_3 \proc{True}\\
\proc{And} (\id{x}, \proc{False}) &\rightarrow_4 \proc{False}\\
\proc{And} (\proc{False}, \id{x}) &\rightarrow_5 \proc{False}\\
\proc{Or} (\id{b_1}, \id{b_2}) &\rightarrow_6 \proc{Not} (\proc{And}
(\proc{Not} (\id{b_1}), \proc{Not} (\id{b_2})))
\end{align*}

\end{frame}

% ------------------------------------------------------------------------
%
\begin{frame}
\frametitle{Booleans/Orienting the equations (cont)}

\noindent For example, here is a terminating chain for the expression
\(\proc{Or} (\proc{True}, \proc{True})\): {\footnotesize
\begin{align*}
\proc{Or} (\proc{True}, \underline{\proc{And} (\proc{True}, \proc{False})})
&\rightarrow_4 \proc{Or} (\proc{True}, \underline{\proc{False}})\\
  \underline{\proc{Or} (\proc{True}, \proc{False})}
&\rightarrow_6 \underline{\proc{Not} (\proc{And} (\proc{Not}
  (\proc{True}), \proc{Not} (\proc{False})))}\\
  \proc{Not} (\proc{And} (\underline{\proc{Not} (\proc{True})},
  \proc{Not} (\proc{False})))
&\rightarrow_1 \proc{Not} (\proc{And} (\underline{\proc{False}},
  \proc{Not} (\proc{False})))\\
  \proc{Not} (\proc{And} (\proc{False}, \underline{\proc{Not}
  (\proc{False})}))
&\rightarrow_2 \proc{Not} (\proc{And} (\proc{False},
  \underline{\proc{True}}))\\
  \proc{Not} (\underline{\proc{And} (\proc{False}, \proc{True})})
&\rightarrow_5 \proc{Not} (\underline{\proc{False}})\\
  \underline{\proc{Not} (\proc{False})}
&\rightarrow_2 \underline{\proc{True}}
\end{align*}
}
An important property of our system is that it allows different chains
starting from the same expression but they all end on the same value,
e.g., we could have applied \(\rightarrow_2\) before \(\rightarrow_1\).

\end{frame}

% ------------------------------------------------------------------------
%
\begin{frame}
\frametitle{Booleans/Orienting the equations (cont)}

Here, we will always use a \textbf{strategy} which rewrites the
arguments before the function calls.

\bigskip

For instance, given
\[
\proc{Or} (\proc{And} (\proc{True}, \proc{True}), \proc{False})
\]
we will rewrite first
\[
\proc{And} (\proc{True}, \proc{True}) \rightarrow_3 \proc{True}
\]
and then
\[
\proc{Or} (\proc{True}, \proc{False}) \rightarrow_6 \proc{True}
\]
This strategy is named \textbf{call-by-value} because we rewrite the
arguments into their values first before rewriting the function call.

\end{frame}
