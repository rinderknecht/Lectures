\paragraph{Question.} The \emph{negabinary} representation of a number
is similar to the binary representation, except that the base is not
\(2\) but \(-2\). Therefore, the general shape of the negabinary
representation with \(n\) bits \(b_{n-1}, b_{n-2}, \dots, b_0\) is
\[
b_{n-1}(-2)^{n-1} + b_{n-2}(-2)^{n-2} + \dots + b_1(-2)^1 + b_0
\]
Devise a \C function adding two numbers expressed in the negabinary
system: {\small
\begin{verbatim}
int* addNeg2 (int* a, int* b);
\end{verbatim}
}
\noindent where a number is represented as a pointer to an array of
integers. For example, the implementation of the negabinary
\(1101_{-2}\) is a pointer to the leftmost cell of the array
\begin{center}
\begin{tabular}{|c|c|c|c|c|}
\hline
\emph{\textbf{4}} & \(1\) & \(1\) & \(0\) & \(1\)\\
\hline
\end{tabular}
\end{center}

\paragraph{Hints.} Given two negabinary numbers \(b_{n-1}b_{n-2}\dots{b_0}\)
and \(b'_{n-1}b'_{n-2}\dots{b'_0}\), 
\begin{enumerate}

   \item find all the possible carries by considering \(b_0 + b'_0\),
     then \(b_i + b'_i + \text{carry}\);

  \item make a table associating each possible values of \(b_i + b'_i
    + \text{carry}\) to the next carry and the correct bit;

  \item in order to find the maximum size of the output, consider what
    happens when a nonzero carry comes out of the position \(n-1\);

  \item compute the size of the output with your algorithm with one
    carry (initially \(0\)) and one bit, incrementing a variable
    \texttt{size} each time a position is done, then check the last
    nonzero carry and fix \texttt{size} accordingly;

  \item allocate the result array with length \texttt{size + 1};

  \item rerun the algorithm (a loop) but, this time, fix the bits and
    save them in the resulting array.

\end{enumerate}

