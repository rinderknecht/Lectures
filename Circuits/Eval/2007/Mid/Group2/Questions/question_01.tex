\paragraph{Question.} Write a C function
  which converts an integer of type \texttt{int} into its balanced
  ternary representation as an array of \texttt{int}. The prototype
  is
\begin{verbatim}
int* from10to3b (int dec);
\end{verbatim}
The result of the function is an array of integers where the first
  element is the size of the balanced ternary representation and the
  remaining elements are the digits (also known as \emph{trits}), the
  least significant being the last. For
  example, \texttt{from10to3b(17)} returns an array
\begin{center}
\begin{tabular}{|c|c|c|c|c|}
\hline
\emph{\textbf{4}} & \(1\) & \(-1\) & \(0\) & \(-1\)\\
\hline
\end{tabular}
\end{center}
because \(17_{10} = 1\overline{1}0\overline{1}_{3b}\).

\paragraph{Hint.} First, compute the length of the balanced ternary
  representation by using the same algorithm used for the
  conversion. Then allocate an array and finally run the conversion
  algorithm as usual. Try some examples, like \(-17\). In C, the sign
  of \texttt{x \% y} is the sign of \texttt{x}. You need to include
  \texttt{stdio.h} and \texttt{stdlib.h} (for \texttt{malloc}).
