%%-*-latex-*-

\documentclass[11pt,a4paper]{article}

\usepackage{amsmath,amssymb}
\usepackage{verbatim}
\usepackage{graphicx}

/home/rinderkn/LaTeX/TeX/trace.tex
\input{prolog}

\title{Final examination\\ of Introduction to Networking}

\author{Christian Rinderknecht}
\date{12 October 2006}

\begin{document}

\maketitle
\thispagestyle{empty}

\noindent The following equations define the formal differentiation of
a function:
\begin{gather*}
\dfrac{dx}{dx}    = 1 \qquad
\dfrac{dy}{dx}    = 0 \quad \text{where} \;y \neq x\; \text{or}\;
                            y \in\mathbb{N}\\
\dfrac{d}{dx}(f + g) = \dfrac{df}{dx} + \dfrac{dg}{dx}
\qquad
\dfrac{d}{dx}(f - g) = \dfrac{df}{dx} - \dfrac{dg}{dx}\\
\dfrac{d(-f)}{dx} = -{\dfrac{df}{dx}}
\qquad
\dfrac{d}{dx}(f \times g) = f \times {\dfrac{dg}{dx}} +
g \times {\dfrac{df}{dx}} 
\end{gather*}

\noindent Let us call \texttt{plus}, \texttt{minus} and \texttt{times}
the relations for, respectively, addition (\(+\)), subtraction
(\(-\)) and multiplication (\(\times\)). Variables are defined by the
predicate \texttt{var} and numbers by \texttt{num}. For instance,
\texttt{minus(var(x),var(y))} denotes \(x-y\); \texttt{minus(var(x))}
denotes \(-x\) (note that subtraction can have one or two arguments);
\texttt{times(num(3),plus(var(x),var(x)))} means \(3(x + x)\), etc.

\paragraph{Question.} Design a protocol to be used between an automatic teller
machine (ATM) and a bank's centralised computer. Your protocol should
allow a user's card and password to be verified, the account balance
(which is maintained at the bank) to be queried, and an account
withdrawal to be made (money is given to the
customer). \textbf{1.~Specify your protocol} by listing the messages
exchanged and the action taken by the ATM or the bank's centralised
computer on transmission or receipt of each message:
\begin{center}
\begin{tabular}{@{}p{170pt}p{170pt}@{}}
\toprule
\multicolumn{2}{c}{From ATM to Bank}\\
\cmidrule(r{0pt}){1-2}
Message name & Meaning/Action\\
and arguments &\\
\midrule
 & \\
 & \\
 & \\
 & \\
 & \\
 & \\
 & \\
 & \\
\bottomrule
\end{tabular}

\begin{tabular}{@{}p{170pt}p{170pt}@{}}
\toprule
\multicolumn{2}{c}{From Bank to ATM}\\
\cmidrule(r{0pt}){1-2}
Message name & Meaning/Action\\
and arguments &\\
\midrule
 & \\
 & \\
 & \\
 & \\
 & \\
 & \\
 & \\
 & \\
\bottomrule
\end{tabular}
\end{center}
\noindent\textbf{2.~Sketch the operation of your protocol}, using a
diagram, for the cases of a simple withdrawal \textbf{(a)} with no
errors, \textbf{(b)} with one error.

Give the Morris-Pratt algorithm, assuming you know the supply function.


\end{document}
