%%-*-latex-*-

\documentclass[11pt,a4paper]{article}

/home/rinderkn/LaTeX/TeX/trace.tex
\input{prolog}

\title{Final examination\\ of Introduction to Networking}

\author{Christian Rinderknecht}
\date{14 December 2006}

\begin{document}

\maketitle
\thispagestyle{empty}

\begin{enumerate}

  \item Let \texttt{delete(X,S,T)} be a relation true when the list
    \texttt{T} contains the same items as the list \texttt{S}, in the
    same order, except the first \texttt{X} in \texttt{S} (starting
    from the top). One possible definition is
    \PrologIn{delete}
%    {\small\verbatiminput{delete.pl}}

    \noindent We want to modify this definition so that the relation
    is true when \texttt{S} does not contain any \texttt{X}.
 
    \paragraph{Question.} Design a protocol to be used between an automatic teller
machine (ATM) and a bank's centralised computer. Your protocol should
allow a user's card and password to be verified, the account balance
(which is maintained at the bank) to be queried, and an account
withdrawal to be made (money is given to the
customer). \textbf{1.~Specify your protocol} by listing the messages
exchanged and the action taken by the ATM or the bank's centralised
computer on transmission or receipt of each message:
\begin{center}
\begin{tabular}{@{}p{170pt}p{170pt}@{}}
\toprule
\multicolumn{2}{c}{From ATM to Bank}\\
\cmidrule(r{0pt}){1-2}
Message name & Meaning/Action\\
and arguments &\\
\midrule
 & \\
 & \\
 & \\
 & \\
 & \\
 & \\
 & \\
 & \\
\bottomrule
\end{tabular}

\begin{tabular}{@{}p{170pt}p{170pt}@{}}
\toprule
\multicolumn{2}{c}{From Bank to ATM}\\
\cmidrule(r{0pt}){1-2}
Message name & Meaning/Action\\
and arguments &\\
\midrule
 & \\
 & \\
 & \\
 & \\
 & \\
 & \\
 & \\
 & \\
\bottomrule
\end{tabular}
\end{center}
\noindent\textbf{2.~Sketch the operation of your protocol}, using a
diagram, for the cases of a simple withdrawal \textbf{(a)} with no
errors, \textbf{(b)} with one error.


  \item
    Give the Morris-Pratt algorithm, assuming you know the supply function.


  \item 
    \paragraph{Question.} Define the meaning of the pointers
\(\upharpoonleft\), \(\upharpoonright\) and \(\Uparrow\) presented in
class and show how the input is analysed using the transition diagrams
of the previous questions.



  \item 
    \paragraph{Question.} Decipher the secret message by converting
  \textbf{efficiently} from octal to hexadecimal the following
  numbers:
\begin{center}
5413 \quad 3726355 \quad 12 \quad 157255 \quad 150320
\end{center}


\end{enumerate}


\end{document}
