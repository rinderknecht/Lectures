%%-*-Latex-*-

% ------------------------------------------------------------------------
%
\begin{frame}
\frametitle{Matching}

Once we have objects, it is useful to compare them. The mechanism to
do so is called \textbf{matching} in \Prolog, and is different from
the equalities we find in other programming languages.
\begin{itemize}

  \item Two numbers match if they represent the same mathematical
    number.

  \item Two atoms match if they are made of the same characters.

  \item A variable matches any object.

  \item Two structures match if
    \begin{itemize}

    \item their functors match (as atoms),
      
    \item all their corresponding arguments match.
      
    \end{itemize}
\end{itemize}

\end{frame}

% ------------------------------------------------------------------------
%
\begin{frame}[containsverbatim]
\frametitle{Matching (cont)}

Matching and equality agree on \textbf{ground terms}, i.e. objects
containing no variables. In this case, they both either return true
(\texttt{Yes} in \Prolog) or false (\texttt{No}).

\bigskip

The difference between matching and equality concerns
\textbf{non-ground terms}, i.e. objects containing variables.

\bigskip

Consider the following fragment of a C program:
{\small
\begin{verbatim}
if (x == 5) x++;
\end{verbatim}
} 
Here, the run-time has to compare \emph{the value of} \texttt{x} with
\texttt{5}, i.e. it looks up the value to which \texttt{x} is bound
and then matches it against \texttt{5} (remember that equality and
matching agree on ground terms).

\end{frame}

% ------------------------------------------------------------------------
%
\begin{frame}[containsverbatim]
\frametitle{Matching (cont)}

In \Prolog, the closest clause, as far as comparison is concerned, is
the query
{\small
\begin{verbatim}
?- X = 5.
\end{verbatim}
}
But the scoping rules of \Prolog say that this occurrence of variable
\texttt{X} is visible only in this clause. Therefore it is unbound,
i.e., it is not associated to any ``previous'' value.

\bigskip

Instead, because \texttt{X} is a variable, it must match \texttt{5},
so the interpreter answers
{\small
\begin{verbatim}
X = 5
Yes
\end{verbatim}
} 
Here, the successful matching returns a \textbf{binding} for
\texttt{X}, before answering \texttt{Yes}.

\end{frame}

% ------------------------------------------------------------------------
%
\begin{frame}[containsverbatim]
\frametitle{Matching (cont)}

Imagine now a matching involving a structure, like
{\small
\begin{verbatim}
?- date(D,july,2006) = date(9,M,2006).
\end{verbatim}
}
The interpreter first checks whether the functors match (are equal),
which is true. Next, it matches the corresponding arguments against
each other: \texttt{D} against \texttt{9}, \texttt{july} against
\texttt{M} and \texttt{2006} against \texttt{2006}. 

\bigskip

The first matching involves a variable and a number, so the
interpreter chooses the binding \texttt{D = 9}.

\bigskip

The second matching involves an atom and a variable, so the
interpreter chooses the binding \texttt{M = july}.

\bigskip

The last matching is trivial, \texttt{2006 = 2006}, and does not
require any binding.

\end{frame}

% ------------------------------------------------------------------------
%
\begin{frame}[containsverbatim]
\frametitle{Matching (cont)}

So the answer is
{\small
\begin{verbatim}
D = 9
M = july
Yes
\end{verbatim}
}
In other words, a successful matching returns bindings for all
variables in the terms being matched, such as the corresponding
\textbf{instantiation} leads to equal ground terms:
{\small
\begin{verbatim}
date(9,july,2006) = date(9,july,2006)
\end{verbatim}
}
Instantiation means to replace all the variables by the object
to which they are bound.

\end{frame}

% ------------------------------------------------------------------------
%
\begin{frame}[containsverbatim]
\frametitle{Matching/Failure}

A failed matching consists of only \texttt{No}. For example
{\small
\begin{verbatim}
?- 1 = 2.
No
?- date(9,july,2006) = date(9,july,2007).
No
?- date(X,july,2006) = date(9,july,X).
No
\end{verbatim}
}
In the last case, the interpreter finds two bindings for \texttt{X}
which are different: \texttt{X=9} and \texttt{X=2006}, which leads to
failure.

\end{frame}

% ------------------------------------------------------------------------
%
\begin{frame}[containsverbatim]
\frametitle{Matching/Most general substitution}

In general, a successful matching returns several bindings. A set of
bindings is called a \textbf{substitution}. So, an instantiation
consists in applying a substitution to a clause.

\bigskip

Sometimes there can be several possible substitutions that make the
matching a success. For example the matching
{\small
\begin{verbatim}
?- X = Y.
\end{verbatim}
}
can be satisfied by
{\small \verb|X = -3, Y = -3|} or {\small \verb|X = 7, Y = 7|}
and so on.

\end{frame}

% ------------------------------------------------------------------------
%
\begin{frame}[containsverbatim]
\frametitle{Matching/Most general substitution (cont)}

In such cases, \Prolog ensures that the most general substitution will
be retained. In our example, \texttt{X = Y} is the most general,
because all the instances can be obtained from it by replacing
\texttt{X} and \texttt{Y} by the same arbitrary object.

\bigskip

In other words, when matching a variable \texttt{A} against another
variable \texttt{B}, the matching succeeds with the most general
substitution \texttt{A = B}, or
{\small
\begin{verbatim}
A = _G187
B = _G187
\end{verbatim}
}
where \texttt{\_G187} is a variable generated by the interpreter. In
these slides we prefer to write
{\small
\begin{semiverbatim}
A = \(\alpha\)
B = \(\alpha\)
\end{semiverbatim}
}

\end{frame}

% ------------------------------------------------------------------------
%
\begin{frame}[containsverbatim]
\frametitle{Matching/Most general substitution (cont)}

Consider the special case
{\small
\begin{semiverbatim}
?- X = X.
X = \(\alpha\)
Yes
\end{semiverbatim}
}
The danger here is that \Prolog uses the same syntactical convention,
the \texttt{=} character, to denote both variable bindings and
matchings: \texttt{A = 2} can be a matching or a binding, same for
\texttt{A = B}. 

\bigskip

But \texttt{2 = A} is a matching but \textbf{not} a variable
binding. Same for 
{\small
\begin{verbatim}
date(9,july,2006) = date(9,july,2006)
\end{verbatim}
}

\end{frame}

% ------------------------------------------------------------------------
%
\begin{frame}[containsverbatim]
\frametitle{Matching/Tree representation}

It is useful to use the tree representation of \Prolog terms
(page~\pageref{structures_as_trees}) to understand how a matching is
performed.

\bigskip

Consider the two terms 
{\small
\begin{verbatim}
triangle(point(1,1),A,point(2,3))
triangle(X,point(4,Y),point(2,Z))
\end{verbatim}
}
These terms are represented by the trees
\begin{center}
\includegraphics[bb=72 657 177 721]{triangle_tree1}
\hspace{1cm}
\includegraphics[bb=71 657 180 721]{triangle_tree2}
\end{center}

\end{frame}

% ------------------------------------------------------------------------
%
\begin{frame}
\frametitle{Matching/Tree representation (cont)}

The interpreter traverse the two trees from the root to the leaves,
following the same order when visiting the sub-trees. Let us assume
that order between siblings is from left to right.

\bigskip

It matches first the two roots: if one of them is a variable, it stops
and declares success, otherwise it matches the subtrees. Here,
\texttt{triangle = triangle}, so, next, it matches the first subtree
of the first tree with the first subtree of the second tree,
i.e. \texttt{?- point(1,1) = X.} This is a success with the
substitution \texttt{X = point(1,1)}.

\bigskip

Then, the second subtrees are matched, i.e.,
{\small
\begin{semiverbatim}
?- A = point(4,Y).
A = point(4,\(\alpha\))
Y = \(\alpha\)
\end{semiverbatim}
}

\end{frame}

% ------------------------------------------------------------------------
%
\begin{frame}[containsverbatim]
\frametitle{Matching/Tree representation (cont)}

Next, the last subtrees are matched, i.e. the interpreter tries to
answer the query 
{\small
\begin{verbatim}
?- point(2,3) = point(2,Z).
\end{verbatim}
}
The roots are the same: \texttt{point = point}. So, it then matches the
subtrees, i.e. answers now the queries
{\small
\begin{verbatim}
?- 2 = 2.    ?- 3 = Z.
\end{verbatim}
}
successfully with substitution \texttt{Z = 3}. Finally the answer is
the substitution which is the union of all the others:
{\small
\begin{semiverbatim}
X = point(1,1)
A = point(4,\(\alpha\))
Y = \(\alpha\)
Z = 3
\end{semiverbatim}
}

\end{frame}

% ------------------------------------------------------------------------
%
\begin{frame}[containsverbatim]
\frametitle{Matching/Tree representation (cont)}

The proof of the matching can be displayed by means of a proof tree:
\begin{mathpar}
\inferrule
{\texttt{\small point(1,1) = X} 
 \and
 \texttt{\small A = point(4,Y)}
  \and
  \inferrule*
  {\texttt{\small 2 = 2}
   \and
   \texttt{\small 3 = Z}
  }
  {\texttt{\small point(2,3) = point(2,Z)}}
}
{\small
 \texttt{triangle(point(1,1),A,point(2,3))}\\
\texttt{= triangle(X,point(4,Y),point(2,Z))}
}
\end{mathpar}

\end{frame}
