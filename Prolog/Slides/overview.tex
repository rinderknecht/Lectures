%%-*-latex-*-

% ------------------------------------------------------------------------
%
\begin{frame}
\frametitle{Overview of artificial intelligence}

Experts in AI would say that AI is a domain in itself, which is
neither reducible to computer science nor biology, for instance.

\bigskip

Some experts would probably say that AI pretends to \emph{solve
complex problems the way humans do}.

\bigskip

Some other AI experts would disagree and say that the goal of AI is to
solve so hard problems that, somehow, \emph{programs may seem
intelligent}.

\end{frame}

% ------------------------------------------------------------------------
%
\begin{frame}
\frametitle{Overview/A computer scientist's
  point of view}

My presentation of AI is not based on history or philosophy or
cognitive sciences or biology. I will only try to give some
\textbf{intuitions} and some \textbf{techniques}, as algorithms and
heuristics, that are useful in practice.

\bigskip

So what is AI, then? 

\bigskip

From our point of view, it is a collection of techniques for solving
or getting partial or suboptimal solutions to complex or intractable
problems.

\end{frame}

% ------------------------------------------------------------------------
%
\begin{frame}
\frametitle{Plan}

\begin{itemize}

  \item[\textbf{I.}] \textbf{Overview}

    \begin{itemize}

      \item A computer scientist's point of view

      \item \textbf{Applications}

      \item Symbolic approaches

      \item Connectionist approaches

    \end{itemize}

\end{itemize}

\end{frame}

% ------------------------------------------------------------------------
%
\begin{frame}
\frametitle{Overview/Applications}

One reason why some problems are hard to solve is that they may come
from the interaction between the physical world and the computer,
considered as a discrete model of the world. 

\bigskip

For instance, how can the computer cope with shape or \textbf{face
recognition} in signal processing (or \textbf{speech recognition}) and
improve its success rate given successful and failed samples? 

\bigskip

In this case, part of the difficulty lies in the passage from
continuity (physical phenomenon) to discontinuity (logical
abstraction).

\end{frame}

% ------------------------------------------------------------------------
%
\begin{frame}
\frametitle{Overview/Applications (cont)}

Some problems are difficult because their computational complexity is
too high (i.e. their solution requires too much computation to be
realistic) and/or because the data is incomplete, hence the need for
good approximations.

\bigskip

For instance, find the \textbf{shortest path} from one town to
another, given the whole map, or play a \textbf{maze game}, without
the whole map.

\end{frame}

% ------------------------------------------------------------------------
%
\begin{frame}
\frametitle{Overview/Applications (cont)}

Some problems are difficult because they involve \textbf{automatic
logical reasoning}. 

\bigskip

For instance, is it possible to assist a physician in establishing a
diagnosis or help to pinpoint a car failure?

\bigskip

Or, in security analysis: how can we prove this safety theorem about
this train without pilot?

\end{frame}

% ------------------------------------------------------------------------
%
\begin{frame}
\frametitle{Overview/Applications (cont)}

Another approach is more ideological, which consists in modeling some
biological process, without any specific problem in mind, and then
find the fruitful problems. 

\bigskip

For instance, some techniques come from a model inspired by the brain,
called \textbf{abstract neural networks}, or from genetics, called
\textbf{genetic algorithms}. 

\bigskip

These techniques are nevertheless applied, with various extents of
success, to unexpected different kinds of problems.

\end{frame}

% ------------------------------------------------------------------------
%
\begin{frame}
\frametitle{Plan}

\begin{itemize}

  \item[\textbf{I.}] \textbf{Overview}

    \begin{itemize}

      \item A computer scientist's point of view

      \item Applications

      \item \textbf{Symbolic approaches}

      \item Connectionist approaches

    \end{itemize}

\end{itemize}

\end{frame}

% ------------------------------------------------------------------------
%
\begin{frame}
\frametitle{Overview/Symbolic approaches}

\textbf{Symbolic approaches} are based on formal logic reasoning. 

The rationale is to model objects of the problem, as well as their
relationships (in time or space), by \textbf{axioms} and
\textbf{logical rules}. 

\begin{itemize}

  \item An axiom is a fact, something which is true by definition.

  \item A logical rule is an implication, like \(A \Rightarrow B\),
  allowing to go from axioms to \textbf{theorems}.

  \item A theorem is a formula that can be deduced by logical rules
  from the axioms.

\end{itemize}

\end{frame}

% ------------------------------------------------------------------------
%
\begin{frame}
\frametitle{Overview/Symbolic approaches
(cont)}

A software assisting the task of proving theorems is called a
\textbf{proof assistant} or a \textbf{knowledge-based system} (or
\textbf{expert system}), depending on the abstraction level. 

\bigskip

Proof assistants are used by computer theorists in the framework of
very expressive logics, in order to prove for instance security
property for embedded systems (i.e. autonomous vehicles like rockets,
space robots, subway or plane without pilot, car ignition systems
etc.).

\bigskip

Expert systems are more similar to a database of domain-specific
informations and logic rules (simpler than within proof assistants)
which allow queries to be formulated. Such systems are often used as
an interactive tool to help an expert diagnose a failure. The
\textbf{programming language \Prolog} is most famous for such
applications.

\end{frame}

% ------------------------------------------------------------------------
%
\begin{frame}
\frametitle{Plan}

\begin{itemize}

  \item[\textbf{I.}] \textbf{Overview}

    \begin{itemize}

      \item A computer scientist's point of view

      \item Applications

      \item Symbolic approaches

      \item \textbf{Connectionist approaches}

    \end{itemize}

\end{itemize}

\end{frame}

% ------------------------------------------------------------------------
%
\begin{frame}
\frametitle{Overview/Connectionist approaches}

\textbf{Connectionist approaches} are based on the modeling of the
brain as an \textbf{abstract neural network} whose properties are
inspired by the electrical behaviour of biological neurons. 

\bigskip

This approach tries to capture the transition between the continuity
of physical phenomenon (electric waves) and discrete logical units
(bits) by imitating brain-like structures. 

\bigskip

An interesting question (not answered here) is how information pass
from one model to another.

\end{frame}

% ------------------------------------------------------------------------
%
\begin{frame}
\frametitle{Overview/Connectionist approaches (cont)}

Following the brain metaphor, the abstract neural networks are usually
programmed through \textbf{learning} (or \textbf{supervised
training}), i.e. by submitting inputs that must be accepted and inputs
that must be rejected and, each time, adjusting the network parameters
(called \textbf{weights}) to fit \emph{all} the given examples (accept
some and reject others).

\bigskip

Next, the network is expected to extrapolate accordingly to its
learning phase to unknown examples. One application is \textbf{pattern
recognition} (like bit-map character classification in noisy
environments).

\end{frame}
