%%-*-latex-*-

% ------------------------------------------------------------------------
%
\begin{frame}[containsverbatim]
\frametitle{Recursive rules}

Let us add another relation to our family program, the
\texttt{ancestor} relation. 

\bigskip

If we consider again the family tree page~\pageref{family_tree}, we
model the \texttt{ancestor} binary relation by saying that \texttt{X}
is an ancestor of \texttt{Y} if there is a chain of parenthood
relationships from \texttt{X} to \texttt{Y}. For example, Pam is an
ancestor of Jim and Tom is an ancestor of Liz.

\bigskip

The latter example actually illustrates a special case of the
\texttt{ancestor} relation, when \texttt{X} is a parent of
\texttt{Y}. This case is very easy to specify in \Prolog:
{\small
\begin{verbatim}
ancestor(X,Y) :- parent(X,Y).
\end{verbatim}
}
But what if the chain of parents is longer? We could try to add
another rule
{\small
\begin{verbatim}
ancestor(X,Y) :- parent(X,Z), parent(Z,Y).
\end{verbatim}
}

\end{frame}

% ------------------------------------------------------------------------
%
\begin{frame}[containsverbatim]
\frametitle{Recursive rules (cont)}

But what if the chain of ancestors is strictly longer than two
parent relationships? We could try to add
{\small
\begin{verbatim}
ancestor(X,Y) :- parent(X,Z1), parent(Z1,Z2), parent(Z2,Y).
\end{verbatim}
}
but we can never attain the general case, when the chain is
arbitrarily long... The solution consists in designing a
\textbf{recursive rule}. For example:
{\small
\begin{verbatim}
ancestor(X,Y) :- parent(X,Z), ancestor(Z,Y).
\end{verbatim}
}
which translates ``\texttt{X} is an ancestor of \texttt{Y} if
\texttt{X} is the parent of some \texttt{Z} which is in turn the
ancestor of \texttt{Y}.''

\end{frame}

% ------------------------------------------------------------------------
%
\begin{frame}[containsverbatim]
\frametitle{Recursive rules (cont)}

\label{ancestor}

We also must keep the special case when ancestorship is parenthood, so
all together we have
{\small
\begin{verbatim}
ancestor(X,Y) :- parent(X,Y).
ancestor(X,Y) :- parent(X,Z), ancestor(Z,Y).
\end{verbatim}
} 
This pattern is typical of recursive rules: there is at least one rule
which is not recursive, to cover base cases, and some others which are
recursive. Let us query for example
{\small
\begin{verbatim}
?- ancestor(pam,X).
X = bob;
X = ann;
X = pat;
X = jim
\end{verbatim}
}

\end{frame}
