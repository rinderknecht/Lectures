%%-*-latex-*-

% ------------------------------------------------------------------------
%
\begin{frame}[containsverbatim]
\frametitle{Tuples}

\textbf{Tuples} and \textbf{lists} are a very useful data structure
that aggregates objects in a specific order.

\bigskip

For example
{\small
\begin{verbatim}
.(a, b)       % Pair made of `a' and `b'
.(a, b, c)    % Triple made of `a', `b' and `c'
.(a, .(b, c)) 
\end{verbatim}
}
Tuples are common in mathematics, e.g. ``Let \(p\) a point of
coordinates \((x, y)\) such as \dots'' We saw
page~\pageref{structures_as_trees} that \Prolog objects can be
represented as trees:
\begin{center}
\includegraphics[bb=72 695 104 719]{ab}
\hspace{1cm}
\includegraphics[bb=72 696 130 719]{abc}
\hspace{1cm}
\includegraphics[bb=72 675 115 719]{a_bc}
\end{center}

\end{frame}

% ------------------------------------------------------------------------
%
\begin{frame}[containsverbatim]
\frametitle{Lists}

Lists are similar to tuples
{\small
\begin{verbatim}
[a, b]        % List made of `a' and `b'
[a, b, c]     % List made of `a', `b' and `c'
[a, [b, c]]
\end{verbatim}
}
\bigskip

So, where is the difference? List have a special syntax that allows
to distinguish and extract the first elements, called \textbf{head},
while the remaining elements are called the \textbf{tail}:
{\small
\begin{verbatim}
[a, b, c]     % List made of `a', `b' and `c'.
[a | [b,c]]   % Idem. Head is `a' and tail is [b,c].
[a,b | [c]]   % Idem but head is [a, b] and tail is [c].
[a,b,c | []]  % Idem but head is [a, b, c] and tail is [].
\end{verbatim}
}

\end{frame}

% ------------------------------------------------------------------------
%
\begin{frame}[containsverbatim]
\frametitle{Lists (cont)}

Actually, list can be coded by means of tuples; all what is needed is
a special atom for representing the empty list. Tradition notes it
\texttt{[]}. Then
{\small
\begin{verbatim}
[a, b, c]              % List made of `a', `b' and `c'
.(a, .(b, .(c, [])))   % Same
\end{verbatim}
} 
Remember that lists can be \textbf{heterogeneous}, i.e. elements can
be of any kind (see \texttt{[a, [b, c]]} above).

\end{frame}

% ------------------------------------------------------------------------
%
\begin{frame}[containsverbatim]
\frametitle{Lists/Membership}

Let us write a \Prolog program that checks if a given object belongs
to a given list. Let \texttt{member} be this binary relation.

\bigskip

For example, we want
{\small
\begin{verbatim}
?- member(b, [a,b,c]).        % True
?- member(b, [a,[b,c]]).      % False
?- member([b,c], [a,[b,c]]).  % True
\end{verbatim}
}
This suggests the recursive definition
{\it
\begin{quote}
\texttt{X} is a member of a list \texttt{L} if either
\begin{itemize}

  \item \texttt{X} is the head of \texttt{L}, or

  \item \texttt{X} is a member of the tail of \texttt{L}.

\end{itemize}
\end{quote}
}
\end{frame}

% ------------------------------------------------------------------------
%
\begin{frame}[containsverbatim]
\frametitle{Lists/Membership (cont)}

This informal definition can straightforwardly be translated to
\Prolog:
{\small
\begin{verbatim}
member(X, [X | Tail]).
member(X, [Head | Tail]) :- member(X, Tail).
\end{verbatim}
}
or

\PrologIn{member}

Remember that order matters for the procedural meaning!
 
\end{frame}

% ------------------------------------------------------------------------
%
\begin{frame}[containsverbatim]
\frametitle{Arithmetic}

Arithmetic operators are special functors. The following are available:

\bigskip

\begin{tabular}{cl}
  \texttt{+}   & Addition\\
  \texttt{-}   & Subtraction\\
  \texttt{*}   & Multiplication\\
  \texttt{/}   & Division\\
  \texttt{**}  & Power\\
  \texttt{//}  & Integer division\\
  \texttt{mod} & Modulo (remainder of integer division)
\end{tabular}

\end{frame}

% ------------------------------------------------------------------------
%
\begin{frame}[containsverbatim]
\frametitle{Arithmetic (cont)}

If the \Prolog interpreter is naively asked
{\small
\begin{verbatim}
?- X = 1 + 2.
X = 1+2
Yes
\end{verbatim}
}
This because \texttt{+} is a functor and functors trigger no
computation. \Prolog builds the tree
\begin{center}
\includegraphics[bb=72 695 103 721]{one_plus_two}
\end{center}
In order to force the arithmetic interpretation, we use
{\small
\begin{verbatim}
?- X is 1 + 2.
X = 3
\end{verbatim}
}

\end{frame}

% ------------------------------------------------------------------------
%
\begin{frame}[containsverbatim]
\frametitle{Arithmetic (cont)}

We have
{\small
\begin{verbatim}
?- X is 5/2, 
   Y is 5//2, 
   Z is 5 mod 2.
X = 2.5
Y = 2
Z = 1
Yes
\end{verbatim}
}

\end{frame}

% ------------------------------------------------------------------------
%
\begin{frame}[containsverbatim]
\frametitle{Arithmetic/Comparison}

The comparison operators force the evaluation of their argument, as
\texttt{is} does:
{\small
\begin{verbatim}
?- 5 * 3 > 2.
Yes
\end{verbatim}
}
Assume we have a relation \texttt{born} in a program, that relates
people's name to their birth years. We can find all the persons born
between 1980 and 1990 by the query
{\small
\begin{verbatim}
?- born(Name, Year),
   Year >= 1980,
   Year =< 1990.
\end{verbatim}
}

\end{frame}

% ------------------------------------------------------------------------
%
\begin{frame}
\frametitle{Arithmetic/Comparison (cont)}

The comparison operators are

\bigskip

\begin{tabular}{cl}
  \texttt{X > Y}   & \texttt{X} is greater than \texttt{Y}\\
  \texttt{X < Y}   & \texttt{X} is smaller than \texttt{Y}\\
  \texttt{X >= Y}  & \texttt{X} is greater than or equal to
                     \texttt{Y}\\ 
  \texttt{X =< Y}  & \texttt{X} is smaller than or equal to
                     \texttt{Y}\\
  \texttt{X =:= Y} & the values of \texttt{X} and \texttt{Y} are
                     equal\\
  \texttt{X ={\tt\char`\\}= Y} & the values of \texttt{X} and
  \texttt{Y} are not 
                     equal
\end{tabular}

\end{frame}


% ------------------------------------------------------------------------
%
\begin{frame}[containsverbatim]
\frametitle{Arithmetic/Comparison (cont)}

\textbf{Beware!} The goals \texttt{X = Y} (matching) and \texttt{X =:=
  Y} (arithmetic comparison) are completely different.

Consider
{\small
\begin{verbatim}
?- 1 + 2 =:= 2 + 1.
Yes
?- 1 + 2 = 2 + 1.
No
?- 1 + A = B + 2.
A = 2
B = 1
Yes
\end{verbatim}
}

\end{frame}

% ------------------------------------------------------------------------
%
\begin{frame}[containsverbatim]
\frametitle{Arithmetic/Example}

Let us define a relation \texttt{length} which associate a list and
its length.
{\small
\begin{verbatim}
length([],0).
length([_ | Tail],N) :- length(Tail,M), N is 1 + M.
?- length([a,b,[c,d],e],N).
N = 4
\end{verbatim}
}
Note that ``\texttt{N is 1 + M}'' \emph{must} be the second goal of the
body. And what if
{\small
\begin{verbatim}
length([],0).
length([_ | Tail],N) :- length(Tail,M), N = 1 + M.
?- length([a,b,[c,d],e],N).
N = ???
\end{verbatim}
}

\end{frame}

% ------------------------------------------------------------------------
%
\begin{frame}[containsverbatim]
\frametitle{Arithmetic/Example (bis)}

Answer:
{\small
\begin{verbatim}
?- length([a,b,[c,d],e],N).
N = 1 + (1 + (1 + 0))
\end{verbatim}
}
This, again, because \texttt{+} is just a functor. Therefore, we can
equivalently write
{\small
\begin{verbatim}
length([],0).
length([_ | Tail],N) :- N = 1 + M, length(Tail,M).
?- length([a,b,[c,d],e],N).
N = 1 + (1 + (1 + 0))
\end{verbatim}
}

\end{frame}

% ------------------------------------------------------------------------
%
\begin{frame}[containsverbatim]
\frametitle{Arithmetic/Example (bis)}

Or even shorter
{\small
\begin{verbatim}
length([],0).
length([_ | Tail],1 + M) :- length(Tail,M).

?- length([a,b,[c,d],e],N).
N = 1 + (1 + (1 + 0))
Yes

?- length([a,b,[c,d],e],N), Length is N.
N = 1 + (1 + (1 + 0))
Length = 3
Yes
\end{verbatim}
}

\end{frame}
